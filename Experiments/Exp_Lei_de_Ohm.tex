%%%%%%%%%%%%%%%%%%%%%%%%%%%%%%%%%%%%%%%%%%%%%%%%%%%%%%%%%%%%%%%%%%%%%%%%%%%%%%%
\chapter{Lei de Ohm} % Sem "Experiência 01" ou qualquer outro número
\label{Chap:LeiDeOhm}        % para poder trocar a ordem com facilidade
%%%%%%%%%%%%%%%%%%%%%%%%%%%%%%%%%%%%%%%%%%%%%%%%%%%%%%%%%%%%%%%%%%%%%%%%%%%%%%%

\begin{fullwidth}\it
	Que experimento faremos?
	Qual é o objetivo?
	O que veremos/revisaremos? (teoria física)
	Quais conceitos/técnicas de análise de dados utilizaremos?
\end{fullwidth}

%%%%%%%%%%%%%%%%%%%%%%%%%%%%%%%%%%%%%%%%%%%%%%%%%%%%%%%%%%%%%%%%%%%%%%%%%%%%%%%
\section{Física do experimento}
%%%%%%%%%%%%%%%%%%%%%%%%%%%%%%%%%%%%%%%%%%%%%%%%%%%%%%%%%%%%%%%%%%%%%%%%%%%%%%%

%%%%%%%%%%%%%%%%%%%%%%%%%%%%%%%%%%%%%%%%%%%%%
%\subsection{Se necessário, usar subsections}
%%%%%%%%%%%%%%%%%%%%%%%%%%%%%%%%%%%%%%%%%%%%%

%%%%%%%%%%%%%%%%%%%%%%%%%%%%%%%%%%%%%%%%%%%%%%%%%%%%%%%%%%%%%%%%%%%%%%%%%%%%%%%
\section{Experimento}
%%%%%%%%%%%%%%%%%%%%%%%%%%%%%%%%%%%%%%%%%%%%%%%%%%%%%%%%%%%%%%%%%%%%%%%%%%%%%%%

%%%%%%%%%%%%%%%%%%%%%%
\subsection{Objetivos}
%%%%%%%%%%%%%%%%%%%%%%

\begin{itemize}
	\item Verificar a relação entre tensão e corrente para resistores ôhmicos e não ôhmicos;
	\item Verificar a resistência de associações de resistores em série e em paralelo.
\end{itemize}

%%%%%%%%%%%%%%%%%%%%%%%%%%%%%%%%%%%%%%%%%%%%%%%%%%%%%%%%%%%%%%%%%%%%%%%%%%%%%%%
\section{Material Necessário}
%%%%%%%%%%%%%%%%%%%%%%%%%%%%%%%%%%%%%%%%%%%%%%%%%%%%%%%%%%%%%%%%%%%%%%%%%%%%%%%

\begin{itemize}
	\item Multímetro;
	\item Fonte de tensão ajustável;
	\item Protoboard e resistores;
	\item Lâmpadas incandescentes e soquetes; 
	\item Cabos de ligação.
\end{itemize}

%%%%%%%%%%%%%%%%%%%%%%%%%%%%%%%%%%%%%%%%%%%%%%%%%%%%%%%%%%%%%%%%%%%%%%%%%%%%%%%
\section{Procedimento Experimental}
%%%%%%%%%%%%%%%%%%%%%%%%%%%%%%%%%%%%%%%%%%%%%%%%%%%%%%%%%%%%%%%%%%%%%%%%%%%%%%%

%%%%%%%%%%%%%%%%%%%%%%%%%%%%%%%%%%%%%%%%%%%%%
\subsection{Resistores ôhmicos e não-ôhmicos}
%%%%%%%%%%%%%%%%%%%%%%%%%%%%%%%%%%%%%%%%%%%%%

Coleta de dados para um resistor:
\begin{enumerate}
	\item Ajuste a fonte de tensão regulável em seu valor mínimo de tensão e máximo de corrente; 
	\item Ajuste o multímetro na função de amperímetro, na escala de \np[mA]{200};
	\item Ligue um dos terminais da fonte ao multímetro, na porta marcada como \texttt{COM};
	\item Conecte um cabo ao terminal do multímetro marcado como \texttt{mA} e o ligue a um resistor;
	\item Ligue o outro terminal da fonte ao resistor;
	\item Ligue a fonte e ajuste a tensão para aproximadamente \np[V]{1,0} e anote o valor na Tabela~\ref{Tab:ResistoresOhmicosENaoOhmicos};
	\item Verifique o valor correspondente de corrente e o anote na Tabela~\ref{Tab:ResistoresOhmicosENaoOhmicos};
	\item Calcule o valor de potência dissipada e o anote na Tabela~\ref{Tab:ResistoresOhmicosENaoOhmicos}. \textbf{Atenção:} o valor de potência não deve exceder \np[W]{0,5} durante toda a tomada de dados, caso contrário, o componente poderá ficar muito quente e causar queimaduras e/ou ser danificado;
	\item Aumente o valor de tensão da fonte regulável em aproximadamente \np[V]{1,0} e repita os passos acima, completando a Tabela~\ref{Tab:ResistoresOhmicosENaoOhmicos} até \np[V]{10}, ou até que a potência exceda \np[W]{0,5}. 
\end{enumerate}

Coleta de dados para uma lâmpada incandescente:
\begin{enumerate}
	\item Ajuste a fonte de tensão regulável em seu valor mínimo de tensão e máximo de corrente; 
	\item Ajuste o multímetro na função de amperímetro, na escala de \np[A]{20};
	\item Ligue um dos terminais da fonte ao multímetro, na porta marcada como \texttt{COM};
	\item Conecte um cabo ao terminal do multímetro marcado como \texttt{20A} e o ligue aos cabos do soquete da lâmpada;
	\item Ligue o outro terminal da fonte ao resistor;
	\item Ligue a fonte e ajuste a tensão para aproximadamente \np[V]{2,0} e anote o valor na Tabela~\ref{Tab:ResistoresOhmicosENaoOhmicos};
	\item Verifique o valor correspondente de corrente e o anote na Tabela~\ref{Tab:ResistoresOhmicosENaoOhmicos};
	\item Calcule o valor de potência dissipada e o anote na Tabela~\ref{Tab:ResistoresOhmicosENaoOhmicos};
	\item Aumente o valor de tensão da fonte regulável em aproximadamente \np[V]{2,0} e repita os passos acima, completando a Tabela~\ref{Tab:ResistoresOhmicosENaoOhmicos} até \np[V]{20}. 
\end{enumerate}

%%%%%%%%%%%%%%%%%%%%%%%%%%%%%%%%%%%%%
\subsection{Associação de resistores}
%%%%%%%%%%%%%%%%%%%%%%%%%%%%%%%%%%%%%

%%%%%%%%%%%%%%%%%%%%%%%%%%%%%%%%%%%%%%%%%%%%%%%%%%%%%%%%%%%%%%%%%%%%%%%%%%%%%%%
%%%%%%%%%%%%%%%%%%%%%%%%%%%%%%%%%%%%%%%%%%%%%%%%%%%%%%%%%%%%%%%%%%%%%%%%%%%%%%%
%%%%%%%%%%%%%%%%%%%%%%%%%%%%%%%%%%%%%%%%%%%%%%%%%%%%%%%%%%%%%%%%%%%%%%%%%%%%%%%
%%%%%%%%%%%%%%%%%%%%%%%%%%%%%%%%%%%%%%%%%%%%%%%%%%%%%%%%%%%%%%%%%%%%%%%%%%%%%%%
\cleardoublepage

\noindent{}{\huge\textit{Lei de Ohm}}

\vspace{15mm}

\begin{fullwidth}
\noindent{}\makebox[0.6\linewidth]{Turma:\enspace\hrulefill}\makebox[0.4\textwidth]{  Data:\enspace\hrulefill}
\vspace{5mm}

\noindent{}\makebox[0.6\linewidth]{Aluno(a):\enspace\hrulefill}\makebox[0.4\textwidth]{  Matrícula:\enspace\hrulefill}

\noindent{}\makebox[0.6\linewidth]{Aluno(a):\enspace\hrulefill}\makebox[0.4\textwidth]{  Matrícula:\enspace\hrulefill}

\noindent{}\makebox[0.6\linewidth]{Aluno(a):\enspace\hrulefill}\makebox[0.4\textwidth]{  Matrícula:\enspace\hrulefill}

\noindent{}\makebox[0.6\linewidth]{Aluno(a):\enspace\hrulefill}\makebox[0.4\textwidth]{  Matrícula:\enspace\hrulefill}

\noindent{}\makebox[0.6\linewidth]{Aluno(a):\enspace\hrulefill}\makebox[0.4\textwidth]{  Matrícula:\enspace\hrulefill}
\end{fullwidth}

\vspace{5mm}

%%%%%%%%%%%%%%%%%%%%%%%%%%%%%%%%%%%%%%%%%%%%%%%%%%%%%%%%%%%%%%%%%%%%%%%%%%%%%%%
\section{Questionário}
%%%%%%%%%%%%%%%%%%%%%%%%%%%%%%%%%%%%%%%%%%%%%%%%%%%%%%%%%%%%%%%%%%%%%%%%%%%%%%%

\begin{question}[type={exam}]{2}
Preencha as colunas de dados experimentais das tabelas com o número adequado de algarismos significativos e unidades.
\end{question}

\begin{question}[type={exam}]{4}
\begin{enumerate}[label=\roman*.]
\item Faça um gráfico para os dados de corrente $i$ em função da tensão $V$ para o resistor.
\item Realize uma regressão linear dos dados experimentais contidos no gráfico.
\item Através dos coeficientes da regressão linear, determine a resistência.
\end{enumerate}
\end{question}

\begin{question}[type={exam}]{2}
Faça um gráfico para os dados de corrente $i$ em função da tensão $V$ para a lâmpada.
\end{question}


\vfill
%%%%%%%%%%%%%%%%%%%%%%%%%%%%%%%%%%%%%%%%%%%%%%%%%%%%%%%%%%%%%%%%%%%%%%%%%%%%%%%
\pagebreak
\section{Tabelas}
%%%%%%%%%%%%%%%%%%%%%%%%%%%%%%%%%%%%%%%%%%%%%%%%%%%%%%%%%%%%%%%%%%%%%%%%%%%%%%%

\begin{table*}
	\begin{center}
		\begin{tabular}{lp{20mm}p{20mm}p{20mm}lp{20mm}p{20mm}p{20mm}l}
		\toprule
		& \multicolumn{3}{l}{\textbf{Resistor}} & & \multicolumn{3}{l}{\textbf{Lâmpada}} \\
		\cmidrule{2-4} \cmidrule{6-8}
		& $V$ & $i$ & $P$ & & $V$ & $i$ & $P$ \\
		\cmidrule{2-4} \cmidrule{6-8}
		& \cellcolor[gray]{0.89} & \cellcolor[gray]{0.92} & \cellcolor[gray]{0.89} & & \cellcolor[gray]{0.89} & \cellcolor[gray]{0.92} & \cellcolor[gray]{0.89} & \\
		& \cellcolor[gray]{0.95} & \cellcolor[gray]{0.97} & \cellcolor[gray]{0.95} & & \cellcolor[gray]{0.95} & \cellcolor[gray]{0.97} & \cellcolor[gray]{0.95} & \\
		& \cellcolor[gray]{0.89} & \cellcolor[gray]{0.92} & \cellcolor[gray]{0.89} & & \cellcolor[gray]{0.89} & \cellcolor[gray]{0.92} & \cellcolor[gray]{0.89} & \\
		& \cellcolor[gray]{0.95} & \cellcolor[gray]{0.97} & \cellcolor[gray]{0.95} & & \cellcolor[gray]{0.95} & \cellcolor[gray]{0.97} & \cellcolor[gray]{0.95} & \\
		& \cellcolor[gray]{0.89} & \cellcolor[gray]{0.92} & \cellcolor[gray]{0.89} & & \cellcolor[gray]{0.89} & \cellcolor[gray]{0.92} & \cellcolor[gray]{0.89} & \\
		& \cellcolor[gray]{0.95} & \cellcolor[gray]{0.97} & \cellcolor[gray]{0.95} & & \cellcolor[gray]{0.95} & \cellcolor[gray]{0.97} & \cellcolor[gray]{0.95} & \\
		& \cellcolor[gray]{0.89} & \cellcolor[gray]{0.92} & \cellcolor[gray]{0.89} & & \cellcolor[gray]{0.89} & \cellcolor[gray]{0.92} & \cellcolor[gray]{0.89} & \\
		& \cellcolor[gray]{0.95} & \cellcolor[gray]{0.97} & \cellcolor[gray]{0.95} & & \cellcolor[gray]{0.95} & \cellcolor[gray]{0.97} & \cellcolor[gray]{0.95} & \\
		& \cellcolor[gray]{0.89} & \cellcolor[gray]{0.92} & \cellcolor[gray]{0.89} & & \cellcolor[gray]{0.89} & \cellcolor[gray]{0.92} & \cellcolor[gray]{0.89} & \\
		& \cellcolor[gray]{0.95} & \cellcolor[gray]{0.97} & \cellcolor[gray]{0.95} & & \cellcolor[gray]{0.95} & \cellcolor[gray]{0.97} & \cellcolor[gray]{0.95} & \\
		& \cellcolor[gray]{0.89} & \cellcolor[gray]{0.92} & \cellcolor[gray]{0.89} & & \cellcolor[gray]{0.89} & \cellcolor[gray]{0.92} & \cellcolor[gray]{0.89} & \\
		& \cellcolor[gray]{0.95} & \cellcolor[gray]{0.97} & \cellcolor[gray]{0.95} & & \cellcolor[gray]{0.95} & \cellcolor[gray]{0.97} & \cellcolor[gray]{0.95} & \\
		& \cellcolor[gray]{0.89} & \cellcolor[gray]{0.92} & \cellcolor[gray]{0.89} & & \cellcolor[gray]{0.89} & \cellcolor[gray]{0.92} & \cellcolor[gray]{0.89} & \\
		& \cellcolor[gray]{0.95} & \cellcolor[gray]{0.97} & \cellcolor[gray]{0.95} & & \cellcolor[gray]{0.95} & \cellcolor[gray]{0.97} & \cellcolor[gray]{0.95} & \\
\bottomrule
		\end{tabular}
	\caption[][2mm]{Dados obtidos para o comprimento do elástico e a massa correspondente.}\label{Tab:ResistoresOhmicosENaoOhmicos}
	\end{center}
\end{table*}

