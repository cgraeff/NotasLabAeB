%%%%%%%%%%%%%%%%%%%%%%%%%%%%%%%%%%%%%%%%%%%%%%%%%%%%%%%%%%%%%%%%%%%%%%%%%%%%%%%
\chapter{Pêndulo Físico} % Sem "Experiência 01" ou qualquer outro número
\label{Chap:PenduloFisico} % para poder trocar a ordem com facilidade
%%%%%%%%%%%%%%%%%%%%%%%%%%%%%%%%%%%%%%%%%%%%%%%%%%%%%%%%%%%%%%%%%%%%%%%%%%%%%%%

\begin{fullwidth}\it
	Realizaremos um experimento usando pêndulo físicos, isto é, pêndulos que não seguem a descrição idealizada do pêndulo simples. Nossos objetivos incluem verificar os efeitos da distribuição não puntual de massa e determinar o valor da aceleração da gravidade com maior precisão. Verificaremos o cálculo de momentos de inércia de alguns corpos, considerando suas distribuições de massa como distribuições contínuas. Para a coleta, interpretação, e análise de dados, utilizaremos algarismos significativos, erros percentuais, gráficos, e regressões lineares. 
\end{fullwidth}

%%%%%%%%%%%%%%%%%%%%%%%%%%%%%%%%%%%%%%%%%%%%%%%%%%%%%%%%%%%%%%%%%%%%%%%%%%%%%%%
\section{Pêndulo Físico}
%%%%%%%%%%%%%%%%%%%%%%%%%%%%%%%%%%%%%%%%%%%%%%%%%%%%%%%%%%%%%%%%%%%%%%%%%%%%%%%

\begin{marginfigure}[2cm]
\centering
\begin{tikzpicture}[>=Stealth]

    \draw[pattern = north west lines, rotate = -75] (0,-0.1) rectangle (3, 0.1);
    \draw[dashed] (0,0) coordinate (A) -- (-75:3.5) coordinate (B);
    \draw[dashed] (0,0) -- (0,-3.5) coordinate (C);
    \draw[fill = white, draw = black, rotate = -75] (1.5,0) coordinate (CM) circle (1 pt) node[above right]{CM};
    
    \pic[draw, "$\theta$", angle eccentricity = 1.05, angle radius = 3.25 cm]{angle = C--A--B};
    \draw[fill] (0,0) circle (1 pt);
    
    \draw[dotted] (0,0) -- (15:1.25);
    \draw[|<->|] (15:1.25) -- node[right]{$L$} +(-75:1.5) coordinate (D);
    \draw[dotted] (D) -- (CM);
    
\end{tikzpicture}
\caption{Oscilação de uma haste. \label{Fig:TeoremaTrabalhoEnergiaRotacaoHaste}}
\end{marginfigure}

Embora o pêndulo simples forneça uma descrição simples de um movimento oscilatório sujeito à força gravitacional, ele não é uma boa aproximação para um corpo extenso. Nesse caso, precisamos abordar o movimento do ponto de vista de rotações. Se tomarmos, por exemplo, uma haste que pode girar em torno de um eixo que passa por sua extremidade e que é perpendicular ao eixo da haste, ao a deslocarmos por um ângulo $\theta$ em relação ao ponto inferior e a liberarmos para que se mova, obteremos um movimento oscilatório. Tal movimento se deve ao torque gerado pela força peso, cujo valor pode ser determinado através de
\begin{align}
    |\tau_P| &= |\vec{r} \times \vec{P}| \\
    &= mg \ell \cdot \sen\theta.
\end{align}
%
Para ângulos pequenos, podemos expandir $\sen\theta$ em uma série de potências, obtendo
\begin{equation}
    \sen\theta = \theta - \frac{\theta^3}{3!} + \frac{\theta^5}{5!} + \dots,
\end{equation}
%
o que nos permite usar $\sen\theta \approx \theta$. Note ainda que o torque sempre tem o sentido oposto ao do deslocamento angular $\theta$, por isso devemos utilizar um sinal negativo e escrever
\begin{equation}
    \tau_P = - mg\ell \cdot \theta.
\end{equation}

Aplicando a Segunda Lei de Newton para a rotação, obtemos
\begin{align}
    \tau_{\text{Res}}^{\text{Ext}} &= I\alpha \\
    \tau_P &= I \frac{d^2}{d\theta^2} \\
    -mg\ell \cdot \theta &= I \frac{d^2}{d\theta^2} \\
    \frac{-mg\ell}{I} \cdot \theta &= \frac{d^2}{d\theta^2},
\end{align}
%
o que pode ser reescrito como
\begin{equation}
    \frac{d^2}{d\theta^2} + \frac{mg\ell}{I} \cdot \theta = 0.
\end{equation}

O resultado acima é análogo à Equação~\eqref{Eq:EquacaoDiferencialOHS}, que obtivemos no Experimento de Oscilações. Procedendo da mesma maneira, é possível verificar que a solução para a equação acima é
\begin{equation}
    \theta(t) = \theta_{\text{Max}} \sen(\omega t + \phi),
\end{equation}
%
onde $\theta_{\text{Max}}$ representa a amplitude de oscilação e
\begin{equation}
    \omega = \sqrt{\frac{mg\ell}{I}}.
\end{equation}
%
Utilizando a relação
\begin{equation}
    f = \frac{\omega}{2\pi} = \frac{1}{T},
\end{equation}
%
obtemos uma relação entre o período e o momento de inércia:
\begin{equation}
    T = 2 \pi \sqrt{\frac{I}{mg\ell}}. \mathnote{Relação para o período de um pêndulo físico.}
\end{equation}

Através do resultado acima, podemos verificar que o período de oscilação não depende apenas da distância entre o centro de massa e o eixo de rotação, mas da própria forma do objeto. Nas seções seguintes determinaremos as expressões para os momentos de inércia dos corpos que utilizaremos.

%%%%%%%%%%%%%%%%%%%%%%%%%%%%%%%%%%%%%%%%%%
\subsection{Cálculo do momento de inércia}
%%%%%%%%%%%%%%%%%%%%%%%%%%%%%%%%%%%%%%%%%%

Expr geral. Falar que a do disco e do tubo usadas na roda de maxwell podem ser usadas, referenciar.

O momento de inércia de uma distribuição contínua de massa pode ser determinada através da integral
\begin{equation}
    I = \int r_{\perp}^2 \;dm,
\end{equation}
%
onde $r_{\perp}$ representa a distância do diferencial de massa $dm$ ao eixo de rotação. A expressão acima é mais conveniente de se trabalhar ao realizarmos uma substituição utilizando uma densidade de massa linear $\lambda(x)$, superficial $\sigma(\vec{r})$, ou volumétrica $\rho(\vec{r})$. Assim,
\begin{align}
    dm &= \lambda(x) \; dx \\
    dm &= \sigma(\vec{r}) \; dA \\
    dm &= \rho(\vec{r}) \; dV.
\end{align}

Para a placa em formato de disco e para a placa circular, podemos utilizar as Expressões~\eqref{Eq:MomentoDeInerciaDisco} e~\eqref{Eq:MomentoDeInerciaTuboCilindrico} obtidos no Experimento da Roda de Maxwell. Restam, portanto, determinar os momento de inércia da placa retangular e da placa triangular.

%%%%%%%%%%%%%%%%%%%%%%%%%%%%
\paragraph{Placa retangular}
%%%%%%%%%%%%%%%%%%%%%%%%%%%%

\begin{marginfigure}[2cm]
\centering
\begin{tikzpicture}[>=Stealth]

    \draw[gray, pattern = north west lines, pattern color = lightgray] (0,0) rectangle (3, 1);
    
    \draw[dashed, ->] (-0.25,0.5) -- (3.75,0.5) node[below left]{$x$};
    \draw[dashed, ->] (1.5,-0.25) --(1.5, 1.75) node[below left]{$y$};
    \draw[draw = black, fill = white] (1.5,0.5) coordinate (O) circle (1pt);
    
    \draw[|<->|] (-0.5,0) -- node[left]{$b$} (-0.5, 1);
    \draw[|<->|] (0,-0.5) -- node[below]{$a$} (3, -0.5);
\end{tikzpicture}
\caption{Placa retangular. \label{Fig:PlacaRetangularMomentoInercia}}
\end{marginfigure}

Para uma placa retangular com densidade superficial de massa uniforme $\sigma(\vec{r}) \equiv \sigma$, o centro de massa se localiza em seu centro, devido às suas simetrias. Considerando que o eixo de rotação passa pelo centro de massa, perpendicular à face plana da placa, e que a origem de um sistema de coordenadas está localizada no centro de massa (veja a Figura~\ref{Fig:PlacaRetangularMomentoInercia}), podemos determinar o momento de inércia através de
\begin{align}
    I &= \int r_{\perp}^2 \;dm \\
    &= \sigma \int_{-\nicefrac{a}{2}}^{\nicefrac{a}{2}} \int_{-\nicefrac{b}{2}}^{\nicefrac{b}{2}} r_\perp^2 \; dy\;dx.  
\end{align}
%
Note que $\vec{r}_\perp \equiv \vec{r}$, uma vez que o vetor $\vec{r}$ descreve a posição do diferencial de massa $dm$ em relação ao eixo de rotação, pois escolhemos a origem no mesmo local por onde passa o eixo de rotação. Assim,
\begin{equation}
    r_\perp^2 = r^2 = x^2 + y^2.
\end{equation}
%
Substituindo a expressão acima na integral e realizando a integração, obtemos
\begin{align}
    I &= \sigma \int_{-\nicefrac{a}{2}}^{\nicefrac{a}{2}} \int_{-\nicefrac{b}{2}}^{\nicefrac{b}{2}} r_\perp^2 \; dy\;dx. \\
    &= \sigma \frac{a^3b + b^3a}{12}.
\end{align}
%
Finalmente, usando $\sigma = M/(ab)$, obtemos
\begin{equation}
    I = M \frac{a^2 + b^2}{12}. \mathnote{Momento de inércia de uma placa em torno de um eixo que passa pelo centro de massa, perpendicularmente à face plana.}
\end{equation}


%%%%%%%%%%%%%%%%%%%%%%%%%%%%
\paragraph{Placa triangular}
%%%%%%%%%%%%%%%%%%%%%%%%%%%%

\begin{marginfigure}[4cm]
\centering
\begin{tikzpicture}[>=Stealth]

    \draw[gray, pattern = north east lines, pattern color = lightgray] (0,0) -- (3,0) -- (0,2) -- cycle;
    
    \draw[dashed, ->] (-0.25,0) -- (3.75,0) node[below left]{$x$};
    \draw[dashed, ->] (0,-0.25) --(0, 2.5) node[below left]{$y$};
    
    \draw[|<->|] (-0.5,0) -- node[left]{$b$} (-0.5, 2);
    \draw[|<->|] (0,-0.5) -- node[below]{$a$} (3, -0.5);
\end{tikzpicture}
\caption{Placa triangular. \label{Fig:PlacaTriangularMomentoInercia}}
\end{marginfigure}

Para uma placa triangular, precisamos primeiramente determinar a posição do centro de massa, possibilitando que o teorema dos eixos paralelos seja usado para outros eixos perpendiculares à placa. Considerando o eixo $x$ (veja a Figura~\ref{Fig:PlacaTriangularMomentoInercia}), temos
\begin{align}
    x_{\text{CM}} &= \frac{1}{M} \int x \; dm \\
    &=\frac{1}{M} \int x\cdot\sigma \; dA \\
    &= \frac{\sigma}{M} \int_0^a \int_0^{y(x)} x \; dy\;dx.
\end{align}
%
Para determinar a função $y(x)$, basta verificarmos que temos uma reta
\begin{equation}
    y(x) = A + B \cdot x,
\end{equation}

\begin{marginfigure}
\centering
\begin{tikzpicture}[>=Stealth]

    \draw[gray, pattern = north east lines, pattern color = lightgray] (0,0) -- (3,0) -- (0,2) -- cycle;
    
    \draw[dashed, ->] (-0.25,0) -- (3.75,0) node[below left]{$x$};
    \draw[dashed, ->] (0,-0.25) --(0, 2.75) node[below left]{$y$};
    
    \draw[|<->|] (3.25,0) -- node[right]{$b$} (3.25, 2);
    \draw[|<->|] (0,2.25) -- node[above]{$a$} (3, 2.25);
    
    \draw[thick] (0,2) coordinate (A2) -- (3,0) coordinate (A1);
    \draw[dotted] (0,2) -- (3,2) coordinate (A3) -- (3,0);
    \pic[draw, "$\gamma$", angle eccentricity = 1.5, angle radius = 6mm]{angle = A1--A2--A3};
    
\end{tikzpicture}
\caption{Determinação da função $y(x)$ para o limite superior de integração no eixo $y$. \label{Fig:PlacaTriangularMomentoInercia2}}
\end{marginfigure}

\noindent{}que $y(0) = b$, que $|B| = \tan \gamma = b/a$, e que $B < 0$, pois temos uma reta decrescente. Assim,
\begin{align}
    x_{\text{CM}} &= \frac{\sigma}{M} \int_0^a \int_0^{y(x)} x \; dy\;dx \\
    &= \frac{\sigma}{M} \int_0^a \int_0^{b(1-x/a)} x \; dy\;dx \\
    &= \frac{a^2b\sigma}{6M}.
\end{align}
%
Como $\sigma = M / (ab/2)$, obtemos finalmente
\begin{equation}
    x_{\text{CM}} = \frac{a}{3}.
\end{equation}
%
A determinação da posição $y_{\text{CM}}$ do centro de massa no eixo vertical segue o mesmo raciocínio,\footnote{Se renomearmos as medidas como $a = b'$ e $b = a'$, e também os eixos como $x = y'$ e $y = x'$, fica claro que o resultado no eixo vertical seria $x'_{\text{CM}} = a'/3$, o que corresponde a $y_{\text{CM}} = b/3$.} resultando em
\begin{equation}
    y_{\text{CM}} = \frac{b}{3}.
\end{equation}

\begin{marginfigure}[1cm]
\centering
\begin{tikzpicture}[>=Stealth]

    \draw[gray, pattern = north east lines, pattern color = lightgray] (0,0) -- (3,0) -- (0,2) -- cycle;
    
    \draw[dashed, ->] (-0.25,0) -- (3.75,0) node[below left]{$x$};
    \draw[dashed, ->] (0,-0.25) --(0, 2.5) node[below left]{$y$};
    
    \draw[|<->|] (-0.5,0) -- node[left]{$b/3$} (-0.5, 0.667);
    \draw[|<->|] (0,-0.5) -- node[below]{$a/3$} (1, -0.5);
    \draw[thick, dotted] (1,0.667) +(-0.5,0) -- +(0.5,0);
    \draw[thick, dotted] (1,0.667) +(0,-0.5) -- +(0,0.5);
    \draw[draw = black, fill = white] (1,0.667) node[below right]{CM} circle (1pt);
    
\end{tikzpicture}
\caption{Posição do centro de massa da placa triangular. \label{Fig:PlacaTriangularMomentoInercia3}}
\end{marginfigure}

Podemos realizar a determinação do momento de inércia em relação ao eixo que passa no centro de massa sem alterar o sistema de coordenadas ao notar que a posição em relação à origem (que será usada como variável de integração) pode ser escrita como (veja a Figura~\ref{Fig:PlacaTriangularMomentoInercia4})
\begin{equation}
    \vec{r} = \vec{r}_{\text{CM}} + \vec{r}_\perp,
\end{equation}
%
de onde podemos escrever
\begin{align}
    \vec{r}_\perp &= \vec{r} - \vec{r}_{\text{CM}} \\
    &= (x \,\versi + y\,\versj) - \left(\frac{a}{3} \versi + \frac{b}{3} \versj \right) \\
    &= \left(x - \frac{a}{3}\right)\versi + \left(y - \frac{b}{3}\right)\versj.
\end{align}

\begin{marginfigure}[1cm]
\centering
\begin{tikzpicture}[>=Stealth, scale = 1.5]

    \draw[fill] (0.65, 1.5) coordinate (P) circle (0.75pt);
    \draw[->] (1,0.667) coordinate (CM) -- node[above right]{$\vec{r}_\perp$} (P);
    \draw[->] (0,0) coordinate (O) -- node[below right]{$\vec{r}_{\text{CM}}$} (CM);
    \draw[->] (O) -- node[left]{$\vec{r}$} (P);
    \draw[draw = black, fill = white] (CM) node[below right]{CM} circle (0.75pt);
    
\end{tikzpicture}
\caption{Relação entre os vetores $\vec{r}$, $\vec{r}_{\text{CM}}$, e $\vec{r}_\perp$. \label{Fig:PlacaTriangularMomentoInercia4}}
\end{marginfigure}

Podemos agora determinar o momento de inércia através da definição:
\begin{align}
    I &= \int r_{\perp}^2 \;dm \\
    &= \sigma \int_{0}^{a} \int_{0}^{y(x)} r_\perp^2 \; dy\;dx.
\end{align}
%
Note que
\begin{align}
    \vec{r}_\perp^2 &\equiv \vec{r}_\perp \cdot \vec{r}_\perp \\
    &= \left(x - \frac{a}{3}\right)^2 + \left(y - \frac{b}{3}\right)^2,
\end{align}
%
o que resulta em 
\begin{align}
    I &= \sigma \int_{0}^{a} \int_{0}^{y(x)} r_\perp^2 \; dy\;dx \\
    &= \sigma \int_{0}^{a} \int_{0}^{b(1-x/a)} \left(x - \frac{a}{3}\right)^2 + \left(y - \frac{b}{3}\right)^2 \; dy\;dx \\
    &= \sigma \frac{ab(a^2 + b^2)}{36}.
\end{align}
%
Finalmente, substituindo $\sigma = M / (ab/2)$, obtemos
\begin{equation}
    I = M \frac{a^2 + b^2}{18}.
\end{equation}

%%%%%%%%%%%%%%%%%%%%%%%%%%%%%%%%%%%%%%%%%%%%%%%%%%%%%%%%%%%%%%%%%%%%%%%%%%%%%%%
\section{Experimento}
%%%%%%%%%%%%%%%%%%%%%%%%%%%%%%%%%%%%%%%%%%%%%%%%%%%%%%%%%%%%%%%%%%%%%%%%%%%%%%%

%%%%%%%%%%%%%%%%%%%%%%
\subsection{Objetivos}
\label{Sec:ObjetivosPenduloFisico}
%%%%%%%%%%%%%%%%%%%%%%

\begin{itemize}
	\item Determinar o erro percentual entre os valores teórico e experimental do período de oscilação para os diferentes corpos/eixos de rotação;
	\item Verificar a dependência do período de oscilação no momento de inércia;
	\item Determinar o valor da aceleração da gravidade através do período de oscilação de um corpo extenso.
\end{itemize}

%%%%%%%%%%%%%%%%%%%%%%%%%%%%%%%%%%%%%%%%%%%%%%%%%%%%%%%%%%%%%%%%%%%%%%%%%%%%%%%
\section{Material Necessário}
%%%%%%%%%%%%%%%%%%%%%%%%%%%%%%%%%%%%%%%%%%%%%%%%%%%%%%%%%%%%%%%%%%%%%%%%%%%%%%%

\begin{itemize}
	\item Aparato de pêndulo físico (suporte vertical, eixo com rolamentos, placas perfuradas);
	\item Régua;
	\item Paquímetro;
	\item Cronômetro;
	\item Balança digital.
\end{itemize}

%%%%%%%%%%%%%%%%%%%%%%%%%%%%%%%%%%%%%%%%%%%%%%%%%%%%%%%%%%%%%%%%%%%%%%%%%%%%%%%
\section{Procedimento Experimental}
%%%%%%%%%%%%%%%%%%%%%%%%%%%%%%%%%%%%%%%%%%%%%%%%%%%%%%%%%%%%%%%%%%%%%%%%%%%%%%%

%%%%%%%%%%%%%%%%%%%%%
\subsection{Medidas} % Se necessário
%%%%%%%%%%%%%%%%%%%%%
\begin{enumerate}
	\item Para cada uma das placas, determine as dimensões e anote na Tabela~\ref{DadosPenduloFisico};
	\item Afira as medidas de massa das placas e anote na mesma tabela em que foram anotadas as dimensões.
\end{enumerate}

%%%%%%%%%%%%%%%%%%%%%
\subsection{Medidas do período de oscilação} % Se necessário
%%%%%%%%%%%%%%%%%%%%%
\begin{enumerate}
    \item Escolha uma placa e um eixo de rotação. Anote na coluna C/E da Tabela~\ref{DadosPenduloFisico} qual das placas e eixo foi escolhido.\footnote{Por exemplo: retângulo/extremidade, retângulo/intermediário, triângulo/vér-\\tice, triângulo/aresta, etc.}
    \item Determine as dimensões e massa da placa e anote na na Tabela~\ref{DadosPenduloFisico}
	\item Determine a distância do centro de massa até o eixo de rotação. Anote o valor obtido na coluna $\ell$ da Tabela~\ref{DadosPenduloFisico};
	\item Instale a placa no eixo de oscilação e a disponha na posição de equilíbrio;
	\item Desloque a placa por um ângulo inferior a \degree{10} e cronometre o tempo de 10 oscilações completas. Anote o valor obtido na Tabela~\ref{DadosPenduloFisico};
	\item Repita os passos acima para as demais placas e eixos, anotando os resultados na Tabela~\ref{DadosPenduloFisico}.
\end{enumerate}

%%%%%%%%%%%%%%%%%%%%%%%%%%%%%%%%%%%%%%%%%%%%%%%%%%%%%%%%%%%%%%%%%%%%%%%%%%%%%%%
%%%%%%%%%%%%%%%%%%%%%%%%%%%%%%%%%%%%%%%%%%%%%%%%%%%%%%%%%%%%%%%%%%%%%%%%%%%%%%%
%%%%%%%%%%%%%%%%%%%%%%%%%%%%%%%%%%%%%%%%%%%%%%%%%%%%%%%%%%%%%%%%%%%%%%%%%%%%%%%
%%%%%%%%%%%%%%%%%%%%%%%%%%%%%%%%%%%%%%%%%%%%%%%%%%%%%%%%%%%%%%%%%%%%%%%%%%%%%%%
\cleardoublepage

\noindent{}{\huge\textit{Pêndulo Físico}}

\vspace{15mm}

\begin{fullwidth}
\noindent{}\makebox[0.6\linewidth]{Turma:\enspace\hrulefill}\makebox[0.4\textwidth]{  Data:\enspace\hrulefill}
\vspace{5mm}

\noindent{}\makebox[0.6\linewidth]{Aluno(a):\enspace\hrulefill}\makebox[0.4\textwidth]{  Matrícula:\enspace\hrulefill}

\noindent{}\makebox[0.6\linewidth]{Aluno(a):\enspace\hrulefill}\makebox[0.4\textwidth]{  Matrícula:\enspace\hrulefill}

\noindent{}\makebox[0.6\linewidth]{Aluno(a):\enspace\hrulefill}\makebox[0.4\textwidth]{  Matrícula:\enspace\hrulefill}

\noindent{}\makebox[0.6\linewidth]{Aluno(a):\enspace\hrulefill}\makebox[0.4\textwidth]{  Matrícula:\enspace\hrulefill}

\noindent{}\makebox[0.6\linewidth]{Aluno(a):\enspace\hrulefill}\makebox[0.4\textwidth]{  Matrícula:\enspace\hrulefill}
\end{fullwidth}

\vspace{5mm}

%%%%%%%%%%%%%%%%%%%%%%%%%%%%%%%%%%%%%%%%%%%%%%%%%%%%%%%%%%%%%%%%%%%%%%%%%%%%%%%
\section{Questionário}
%%%%%%%%%%%%%%%%%%%%%%%%%%%%%%%%%%%%%%%%%%%%%%%%%%%%%%%%%%%%%%%%%%%%%%%%%%%%%%%
\emph{Nas questões seguintes, apresente os cálculos requisitados de maneira clara e sucinta, para que o professor possa acompanhar o raciocínio desenvolvido.}
\vspace{5mm}

\begin{question}[type={exam}]{2}
Preencha as colunas de dados experimentais das tabelas com o número adequado de algarismos significativos e unidades.
\end{question}

\begin{question}[type={exam}]{2}
Determine os momentos de inércia teóricos $I_T$ e os períodos teóricos $T_T$ de cada corpo e complete a Tabela~\ref{DadosPenduloFisico}. Utilize o número adequado de algarismos significativos. Determine o erro percentual entre os valores experimental e teórico do período.
\end{question}

\begin{question}[type={exam}]{2}
\begin{enumerate}[label=\roman*.]
    \item Elabore um gráfico de $T^2 \times I/(m\ell)$ com os dados obtidos.
    \item Faça uma regressão linear e a adicione ao gráfico.
\end{enumerate}
\end{question}

\begin{question}[type={exam}]{2}
\begin{enumerate}[label=\roman*.]
    \item Determine o significado físico do coeficiente angular, isto é, determine o que tal coeficiente representa.
    \item Determine o valor da aceleração da gravidade usando o coeficiente angular e calcule o erro percentual em relação ao valor de referência.
\end{enumerate}
\end{question}

\begin{question}[type={exam}]{2}
Considerando os objetivos do experimento, listados na Seção~\ref{Sec:ObjetivosPenduloFisico}, e os resultados obtidos nas questões anteriores, discuta quais objetivos foram atingidos com sucesso, justificando suas conclusões. Se algum objetivo não foi atingido, discuta quais são os possíveis motivos do insucesso e que providências podem ser tomadas para que eles sejam alcançados.
\end{question}

\vfill
%%%%%%%%%%%%%%%%%%%%%%%%%%%%%%%%%%%%%%%%%%%%%%%%%%%%%%%%%%%%%%%%%%%%%%%%%%%%%%%
\pagebreak
\section{Tabelas}
%%%%%%%%%%%%%%%%%%%%%%%%%%%%%%%%%%%%%%%%%%%%%%%%%%%%%%%%%%%%%%%%%%%%%%%%%%%%%%%

\begin{table*}[!ht]
    \centering
    \begin{tabular}{lp{30mm}p{19mm}p{19mm}p{19mm}p{19mm}p{19mm}p{19mm}l}
    \toprule
        & \multicolumn{6}{l}{\textbf{Dimensões e massas das placas}} \\
    \cmidrule{2-6}
        &\multicolumn{2}{l}{\textbf{Retângulo}} & & \multicolumn{2}{l}{\textbf{Disco}} & \\
        \cmidrule{2-3} \cmidrule{5-6}
        & $a$ \cellcolor[gray]{0.89} & \cellcolor[gray]{0.92} & & $R$ \cellcolor[gray]{0.89} & \cellcolor[gray]{0.92} \\
        & $b$ \cellcolor[gray]{0.95} & \cellcolor[gray]{0.97} & & $m$ \cellcolor[gray]{0.95} & \cellcolor[gray]{0.97} \\
        & $m$ \cellcolor[gray]{0.89} & \cellcolor[gray]{0.92} & & \\
        \cmidrule{2-3} \cmidrule{5-6}
        \\
        &\multicolumn{2}{l}{\textbf{Triângulo}} & & \multicolumn{2}{l}{\textbf{Círculo}} \\
        \cmidrule{2-3} \cmidrule{5-6}
        & $a$ \cellcolor[gray]{0.89} & \cellcolor[gray]{0.92} & & $R_e$ \cellcolor[gray]{0.89} & \cellcolor[gray]{0.92} \\
        & $b$ \cellcolor[gray]{0.95} & \cellcolor[gray]{0.97} & & $R_i$\cellcolor[gray]{0.95} & \cellcolor[gray]{0.97} \\
        & $m$ \cellcolor[gray]{0.89} & \cellcolor[gray]{0.92} & & $m$ \cellcolor[gray]{0.89} & \cellcolor[gray]{0.92} \\
        \cmidrule{2-3} \cmidrule{5-6}
        \\
        \\
        &\multicolumn{4}{l}{\textbf{Dados Experimentais}} \\
        \cmidrule{2-8}
        & C/E & $T_{10}$ & $T$ & $\ell$ & $I_T$ & $T_T$ & $E_\%$ & \\
        \cmidrule{2-8}
        & \cellcolor[gray]{0.89} & \cellcolor[gray]{0.92} & \cellcolor[gray]{0.89} & \cellcolor[gray]{0.92} & \cellcolor[gray]{0.89} & \cellcolor[gray]{0.92} & \cellcolor[gray]{0.89} \\
        & \cellcolor[gray]{0.95} & \cellcolor[gray]{0.97} & \cellcolor[gray]{0.95} & \cellcolor[gray]{0.97} & \cellcolor[gray]{0.95} & \cellcolor[gray]{0.97} & \cellcolor[gray]{0.95} \\
        & \cellcolor[gray]{0.89} & \cellcolor[gray]{0.92} & \cellcolor[gray]{0.89} & \cellcolor[gray]{0.92} & \cellcolor[gray]{0.89} & \cellcolor[gray]{0.92} & \cellcolor[gray]{0.89} \\
        & \cellcolor[gray]{0.95} & \cellcolor[gray]{0.97} & \cellcolor[gray]{0.95} & \cellcolor[gray]{0.97} & \cellcolor[gray]{0.95} & \cellcolor[gray]{0.97} & \cellcolor[gray]{0.95} \\
        & \cellcolor[gray]{0.89} & \cellcolor[gray]{0.92} & \cellcolor[gray]{0.89} & \cellcolor[gray]{0.92} & \cellcolor[gray]{0.89} & \cellcolor[gray]{0.92} & \cellcolor[gray]{0.89} \\
        & \cellcolor[gray]{0.95} & \cellcolor[gray]{0.97} & \cellcolor[gray]{0.95} & \cellcolor[gray]{0.97} & \cellcolor[gray]{0.95} & \cellcolor[gray]{0.97} & \cellcolor[gray]{0.95} \\
        & \cellcolor[gray]{0.89} & \cellcolor[gray]{0.92} & \cellcolor[gray]{0.89} & \cellcolor[gray]{0.92} & \cellcolor[gray]{0.89} & \cellcolor[gray]{0.92} & \cellcolor[gray]{0.89} \\
        & \cellcolor[gray]{0.95} & \cellcolor[gray]{0.97} & \cellcolor[gray]{0.95} & \cellcolor[gray]{0.97} & \cellcolor[gray]{0.95} & \cellcolor[gray]{0.97} & \cellcolor[gray]{0.95} \\
        & \cellcolor[gray]{0.89} & \cellcolor[gray]{0.92} & \cellcolor[gray]{0.89} & \cellcolor[gray]{0.92} & \cellcolor[gray]{0.89} & \cellcolor[gray]{0.92} & \cellcolor[gray]{0.89} \\
        & \cellcolor[gray]{0.95} & \cellcolor[gray]{0.97} & \cellcolor[gray]{0.95} & \cellcolor[gray]{0.97} & \cellcolor[gray]{0.95} & \cellcolor[gray]{0.97} & \cellcolor[gray]{0.95} \\
        \cmidrule{2-8}
    \bottomrule
    \end{tabular}
    \caption[][5mm]{Dados para o pêndulo físico}
    \label{DadosPenduloFisico}
\end{table*}
