%%%%%%%%%%%%%%%%%%%%%%%%%%%%%%%%%%%%%%%%%%%%%%%%%%%%%%%%%%%%%%%%%%%%%%%%%%%%%%%
\chapter{Pêndulo Físico} % Sem "Experiência 01" ou qualquer outro número
\label{Chap:PenduloFisico} % para poder trocar a ordem com facilidade
%%%%%%%%%%%%%%%%%%%%%%%%%%%%%%%%%%%%%%%%%%%%%%%%%%%%%%%%%%%%%%%%%%%%%%%%%%%%%%%

\begin{fullwidth}\it
	Que experimento faremos?
	Qual é o objetivo?
	O que veremos/revisaremos? (teoria física)
	Quais conceitos/técnicas de análise de dados utilizaremos?
\end{fullwidth}

%%%%%%%%%%%%%%%%%%%%%%%%%%%%%%%%%%%%%%%%%%%%%%%%%%%%%%%%%%%%%%%%%%%%%%%%%%%%%%%
\section{Pêndulo Físico}
%%%%%%%%%%%%%%%%%%%%%%%%%%%%%%%%%%%%%%%%%%%%%%%%%%%%%%%%%%%%%%%%%%%%%%%%%%%%%%%

\begin{marginfigure}[2cm]
\centering
\begin{tikzpicture}[>=Stealth]

    \draw[pattern = north west lines, rotate = -75] (0,-0.1) rectangle (3, 0.1);
    \draw[dashed] (0,0) coordinate (A) -- (-75:3.5) coordinate (B);
    \draw[dashed] (0,0) -- (0,-3.5) coordinate (C);
    \draw[fill = white, draw = black, rotate = -75] (1.5,0) circle (1 pt) node[above right]{CM};
    
    \pic[draw, "$\theta$", angle eccentricity = 1.05, angle radius = 3.25 cm]{angle = C--A--B};
    \draw[fill] (0,0) circle (1 pt);
    
    \draw[dotted] (0,0) -- (15:1.25);
    \draw[|<->|] (15:1.25) -- node[right]{$L$} +(-75:3) coordinate (D);
    \draw[dotted] (D) -- (-75:3);
    
\end{tikzpicture}
\caption{Oscilação de uma haste. \label{Fig:TeoremaTrabalhoEnergiaRotacaoHaste}}
\end{marginfigure}

Embora o pêndulo simples forneça uma descrição simples de um movimento oscilatório sujeito à força gravitacional, ele não é uma boa aproximação para um corpo extenso. Nesse caso, precisamos abordar o movimento do ponto de vista de rotações. Se tomarmos, por exemplo, uma haste que pode girar em torno de um eixo que passa por sua extremidade e que é perpendicular ao eixo da haste, ao a deslocarmos por um ângulo $\theta$ em relação ao ponto inferior e a liberarmos para que se mova, obteremos um movimento oscilatório. Tal movimento se deve ao torque gerado pela força peso, cujo valor pode ser determinado através de
\begin{align}
    |\tau_{\text{Res}}^{\text{Ext}} | &= |\vec{r} \times \vec{F}| \\
    &= mg \cdot \frac{L}{2} \cdot \sen\theta.
\end{align}

Aplicando a Segunda Lei de Newton para a rotação

%%%%%%%%%%%%%%%%%%%%%%%%%%%%%%%%%%%%%%%%%%%%%
%\subsection{Se necessário, usar subsections}
%%%%%%%%%%%%%%%%%%%%%%%%%%%%%%%%%%%%%%%%%%%%%

%%%%%%%%%%%%%%%%%%%%%%%%%%%%%%%%%%%%%%%%%%%%%%%%%%%%%%%%%%%%%%%%%%%%%%%%%%%%%%%
\section{Experimento}
%%%%%%%%%%%%%%%%%%%%%%%%%%%%%%%%%%%%%%%%%%%%%%%%%%%%%%%%%%%%%%%%%%%%%%%%%%%%%%%

%%%%%%%%%%%%%%%%%%%%%%
\subsection{Objetivos}
%%%%%%%%%%%%%%%%%%%%%%

\begin{itemize}
	\item Resultados concretos que devemos conseguir;
	\item Observar a relação de tal coisa com outra coisa;
	\item Calcular a constante $x$.
\end{itemize}

%%%%%%%%%%%%%%%%%%%%%%%%%%%%%%%%%%%%%%%%%%%%%%%%%%%%%%%%%%%%%%%%%%%%%%%%%%%%%%%
\section{Material Necessário}
%%%%%%%%%%%%%%%%%%%%%%%%%%%%%%%%%%%%%%%%%%%%%%%%%%%%%%%%%%%%%%%%%%%%%%%%%%%%%%%

\begin{itemize}
	\item Item 1;
	\item Item 2.
\end{itemize}

%%%%%%%%%%%%%%%%%%%%%%%%%%%%%%%%%%%%%%%%%%%%%%%%%%%%%%%%%%%%%%%%%%%%%%%%%%%%%%%
\section{Procedimento Experimental}
%%%%%%%%%%%%%%%%%%%%%%%%%%%%%%%%%%%%%%%%%%%%%%%%%%%%%%%%%%%%%%%%%%%%%%%%%%%%%%%

%%%%%%%%%%%%%%%%%%%%%
%\subsection{Parte A} % Se necessário
%%%%%%%%%%%%%%%%%%%%%
\begin{enumerate}
	\item Passo 1;
	\item Passo 2;
	\item Passo 3.
\end{enumerate}

%%%%%%%%%%%%%%%%%%%%%%%%%%%%%%%%%%%%%%%%%%%%%%%%%%%%%%%%%%%%%%%%%%%%%%%%%%%%%%%
%%%%%%%%%%%%%%%%%%%%%%%%%%%%%%%%%%%%%%%%%%%%%%%%%%%%%%%%%%%%%%%%%%%%%%%%%%%%%%%
%%%%%%%%%%%%%%%%%%%%%%%%%%%%%%%%%%%%%%%%%%%%%%%%%%%%%%%%%%%%%%%%%%%%%%%%%%%%%%%
%%%%%%%%%%%%%%%%%%%%%%%%%%%%%%%%%%%%%%%%%%%%%%%%%%%%%%%%%%%%%%%%%%%%%%%%%%%%%%%
\cleardoublepage

\noindent{}{\huge\textit{Pêndulo Físico}}

\vspace{15mm}

\begin{fullwidth}
\noindent{}\makebox[0.6\linewidth]{Turma:\enspace\hrulefill}\makebox[0.4\textwidth]{  Data:\enspace\hrulefill}
\vspace{5mm}

\noindent{}\makebox[0.6\linewidth]{Aluno(a):\enspace\hrulefill}\makebox[0.4\textwidth]{  Matrícula:\enspace\hrulefill}

\noindent{}\makebox[0.6\linewidth]{Aluno(a):\enspace\hrulefill}\makebox[0.4\textwidth]{  Matrícula:\enspace\hrulefill}

\noindent{}\makebox[0.6\linewidth]{Aluno(a):\enspace\hrulefill}\makebox[0.4\textwidth]{  Matrícula:\enspace\hrulefill}

\noindent{}\makebox[0.6\linewidth]{Aluno(a):\enspace\hrulefill}\makebox[0.4\textwidth]{  Matrícula:\enspace\hrulefill}

\noindent{}\makebox[0.6\linewidth]{Aluno(a):\enspace\hrulefill}\makebox[0.4\textwidth]{  Matrícula:\enspace\hrulefill}
\end{fullwidth}

\vspace{5mm}

%%%%%%%%%%%%%%%%%%%%%%%%%%%%%%%%%%%%%%%%%%%%%%%%%%%%%%%%%%%%%%%%%%%%%%%%%%%%%%%
\section{Questionário}
%%%%%%%%%%%%%%%%%%%%%%%%%%%%%%%%%%%%%%%%%%%%%%%%%%%%%%%%%%%%%%%%%%%%%%%%%%%%%%%
\emph{Nas questões seguintes, apresente os cálculos requisitados de maneira clara e sucinta, para que o professor possa acompanhar o raciocínio desenvolvido.}
\vspace{5mm}

\begin{question}[type={exam}]{2}
Lorem ipsum dolor sit amet, consectetuer adi-
piscing elit. Ut purus elit, vestibulum ut, placerat ac, adipiscing vitae,
felis. Curabitur dictum gravida mauris. Nam arcu libero, nonummy
eget, consectetuer id, vulputate a, magna. Donec vehicula augue
eu neque. Pellentesque habitant morbi tristique senectus et netus
et malesuada fames ac turpis egestas. Mauris ut leo. Cras viverra
metus rhoncus sem. Nulla et lectus vestibulum urna fringilla ultrices.
\end{question}

\begin{question}[type={exam}]{2}
Lorem ipsum dolor sit amet, consectetuer adi-
piscing elit. Ut purus elit, vestibulum ut, placerat ac, adipiscing vitae,
felis. Curabitur dictum gravida mauris. Nam arcu libero, nonummy
eget, consectetuer id, vulputate a, magna. Donec vehicula augue
eu neque. Pellentesque habitant morbi tristique senectus et netus
et malesuada fames ac turpis egestas. Mauris ut leo. Cras viverra
metus rhoncus sem. Nulla et lectus vestibulum urna fringilla ultrices.
\end{question}

\begin{question}[type={exam}]{2}
Lorem ipsum dolor sit amet, consectetuer adi-
piscing elit. Ut purus elit, vestibulum ut, placerat ac, adipiscing vitae,
felis. Curabitur dictum gravida mauris. Nam arcu libero, nonummy
eget, consectetuer id, vulputate a, magna. Donec vehicula augue
eu neque. Pellentesque habitant morbi tristique senectus et netus
et malesuada fames ac turpis egestas. Mauris ut leo. Cras viverra
metus rhoncus sem. Nulla et lectus vestibulum urna fringilla ultrices.
\end{question}

\begin{question}[type={exam}]{2}
Lorem ipsum dolor sit amet, consectetuer adi-
piscing elit. Ut purus elit, vestibulum ut, placerat ac, adipiscing vitae,
felis. Curabitur dictum gravida mauris. Nam arcu libero, nonummy
eget, consectetuer id, vulputate a, magna. Donec vehicula augue
eu neque. Pellentesque habitant morbi tristique senectus et netus
et malesuada fames ac turpis egestas. Mauris ut leo. Cras viverra
metus rhoncus sem. Nulla et lectus vestibulum urna fringilla ultrices.
\end{question}

\begin{question}[type={exam}]{2}
Lorem ipsum dolor sit amet, consectetuer adi-
piscing elit. Ut purus elit, vestibulum ut, placerat ac, adipiscing vitae,
felis. Curabitur dictum gravida mauris. Nam arcu libero, nonummy
eget, consectetuer id, vulputate a, magna. Donec vehicula augue
eu neque. Pellentesque habitant morbi tristique senectus et netus
et malesuada fames ac turpis egestas. Mauris ut leo. Cras viverra
metus rhoncus sem. Nulla et lectus vestibulum urna fringilla ultrices.
\end{question}
\vfill
%%%%%%%%%%%%%%%%%%%%%%%%%%%%%%%%%%%%%%%%%%%%%%%%%%%%%%%%%%%%%%%%%%%%%%%%%%%%%%%
\pagebreak
\section{Tabelas}
%%%%%%%%%%%%%%%%%%%%%%%%%%%%%%%%%%%%%%%%%%%%%%%%%%%%%%%%%%%%%%%%%%%%%%%%%%%%%%%

