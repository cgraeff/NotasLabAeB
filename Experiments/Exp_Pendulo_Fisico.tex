%%%%%%%%%%%%%%%%%%%%%%%%%%%%%%%%%%%%%%%%%%%%%%%%%%%%%%%%%%%%%%%%%%%%%%%%%%%%%%%
\chapter{Pêndulo Físico} % Sem "Experiência 01" ou qualquer outro número
\label{Chap:PenduloFisico} % para poder trocar a ordem com facilidade
%%%%%%%%%%%%%%%%%%%%%%%%%%%%%%%%%%%%%%%%%%%%%%%%%%%%%%%%%%%%%%%%%%%%%%%%%%%%%%%

\begin{fullwidth}\it
	Que experimento faremos?
	Qual é o objetivo?
	O que veremos/revisaremos? (teoria física)
	Quais conceitos/técnicas de análise de dados utilizaremos?
\end{fullwidth}

%%%%%%%%%%%%%%%%%%%%%%%%%%%%%%%%%%%%%%%%%%%%%%%%%%%%%%%%%%%%%%%%%%%%%%%%%%%%%%%
\section{Pêndulo Físico}
%%%%%%%%%%%%%%%%%%%%%%%%%%%%%%%%%%%%%%%%%%%%%%%%%%%%%%%%%%%%%%%%%%%%%%%%%%%%%%%

\begin{marginfigure}[2cm]
\centering
\begin{tikzpicture}[>=Stealth]

    \draw[pattern = north west lines, rotate = -75] (0,-0.1) rectangle (3, 0.1);
    \draw[dashed] (0,0) coordinate (A) -- (-75:3.5) coordinate (B);
    \draw[dashed] (0,0) -- (0,-3.5) coordinate (C);
    \draw[fill = white, draw = black, rotate = -75] (1.5,0) coordinate (CM) circle (1 pt) node[above right]{CM};
    
    \pic[draw, "$\theta$", angle eccentricity = 1.05, angle radius = 3.25 cm]{angle = C--A--B};
    \draw[fill] (0,0) circle (1 pt);
    
    \draw[dotted] (0,0) -- (15:1.25);
    \draw[|<->|] (15:1.25) -- node[right]{$L$} +(-75:1.5) coordinate (D);
    \draw[dotted] (D) -- (CM);
    
\end{tikzpicture}
\caption{Oscilação de uma haste. \label{Fig:TeoremaTrabalhoEnergiaRotacaoHaste}}
\end{marginfigure}

Embora o pêndulo simples forneça uma descrição simples de um movimento oscilatório sujeito à força gravitacional, ele não é uma boa aproximação para um corpo extenso. Nesse caso, precisamos abordar o movimento do ponto de vista de rotações. Se tomarmos, por exemplo, uma haste que pode girar em torno de um eixo que passa por sua extremidade e que é perpendicular ao eixo da haste, ao a deslocarmos por um ângulo $\theta$ em relação ao ponto inferior e a liberarmos para que se mova, obteremos um movimento oscilatório. Tal movimento se deve ao torque gerado pela força peso, cujo valor pode ser determinado através de
\begin{align}
    |\tau_P| &= |\vec{r} \times \vec{P}| \\
    &= mg \ell \cdot \sen\theta.
\end{align}
%
Para ângulos pequenos, podemos expandir $\sen\theta$ em uma série de potências, obtendo
\begin{equation}
    \sen\theta = \theta - \frac{\theta^3}{3!} + \frac{\theta^5}{5!} + \dots,
\end{equation}
%
o que nos permite usar $\sen\theta \approx \theta$. Note ainda que o torque sempre tem o sentido oposto ao do deslocamento angular $\theta$, por isso devemos utilizar um sinal negativo e escrever
\begin{equation}
    \tau_P = - mg\ell \cdot \theta.
\end{equation}

Aplicando a Segunda Lei de Newton para a rotação, obtemos
\begin{align}
    \tau_{\text{Res}}^{\text{Ext}} &= I\alpha \\
    \tau_P &= I \frac{d^2}{d\theta^2} \\
    -mg\ell \cdot \theta &= I \frac{d^2}{d\theta^2} \\
    \frac{-mg\ell}{I} \cdot \theta &= \frac{d^2}{d\theta^2},
\end{align}
%
o que pode ser reescrito como
\begin{equation}
    \frac{d^2}{d\theta^2} + \frac{mg\ell}{I} \cdot \theta = 0.
\end{equation}

O resultado acima é análogo à Equação~\eqref{Eq:EquacaoDiferencialOHS}, que obtivemos no Experimento de Oscilações. Procedendo da mesma maneira, é possível verificar que a solução para a equação acima é
\begin{equation}
    \theta(t) = \theta_{\text{Max}} \sen(\omega t + \phi),
\end{equation}
%
onde $\theta_{\text{Max}}$ representa a amplitude de oscilação e
\begin{equation}
    \omega = \sqrt{\frac{mg\ell}{I}}.
\end{equation}
%
Utilizando a relação
\begin{equation}
    f = \frac{\omega}{2\pi} = \frac{1}{T},
\end{equation}
%
obtemos uma relação entre o período e o momento de inércia:
\begin{equation}
    T = 2 \pi \sqrt{\frac{I}{mg\ell}}. \mathnote{Relação para o período de um pêndulo físico.}
\end{equation}

Através do resultado acima, podemos verificar que o período de oscilação não depende apenas da distância entre o centro de massa e o eixo de rotação, mas da própria forma do objeto. Nas seções seguintes determinaremos as expressões para os momentos de inércia dos corpos que utilizaremos.

%%%%%%%%%%%%%%%%%%%%%%%%%%%%%%%%%%%%%%%%%%
\subsection{Cálculo do momento de inércia}
%%%%%%%%%%%%%%%%%%%%%%%%%%%%%%%%%%%%%%%%%%

Expr geral. Falar que a do disco e do tubo usadas na roda de maxwell podem ser usadas, referenciar.

O momento de inércia de uma distribuição contínua de massa pode ser determinada através da integral
\begin{equation}
    I = \int r_{\perp}^2 \;dm,
\end{equation}
%
onde $r_{\perp}$ representa a distância do diferencial de massa $dm$ ao eixo de rotação. A expressão acima é mais conveniente de se trabalhar ao realizarmos uma substituição utilizando uma densidade de massa linear $\lambda(x)$, superficial $\sigma(\vec{r})$, ou volumétrica $\rho(\vec{r})$. Assim,
\begin{align}
    dm &= \lambda(x) \; dx \\
    dm &= \sigma(\vec{r}) \; dA \\
    dm &= \rho(\vec{r}) \; dV.
\end{align}

Para a placa em formato de disco e para a placa circular, podemos utilizar as Expressões~\eqref{Eq:MomentoDeInerciaDisco} e~\eqref{Eq:MomentoDeInerciaTuboCilindrico} obtidos no Experimento da Roda de Maxwell. Restam, portanto, determinar os momento de inércia da placa retangular e da placa triangular.

%%%%%%%%%%%%%%%%%%%%%%%%%%%%
\paragraph{Placa retangular}
%%%%%%%%%%%%%%%%%%%%%%%%%%%%

\begin{marginfigure}[2cm]
\centering
\begin{tikzpicture}[>=Stealth]

    \draw[gray, pattern = north west lines, pattern color = gray] (0,0) rectangle (3, 1);
    
    \draw[dashed, ->] (-0.25,0.5) -- (3.75,0.5) node[below left]{$x$};
    \draw[dashed, ->] (1.5,-0.25) --(1.5, 1.75) node[below left]{$y$};
    \draw[fill] (1.5,0.5) coordinate (O) circle (1pt);
    
    \draw[|<->|] (-0.5,0) -- node[left]{$b$} (-0.5, 1);
    \draw[|<->|] (0,-0.5) -- node[below]{$a$} (3, -0.5);
\end{tikzpicture}
\caption{Placa retangular. \label{Fig:PlacaRetangularMomentoInercia}}
\end{marginfigure}

Para uma placa retangular com densidade superficial de massa uniforme $\sigma(\vec{r}) \equiv \sigma$, o centro de massa se localiza em seu centro, devido às suas simetrias. Considerando que o eixo de rotação passa pelo centro de massa, perpendicular à face plana da placa, e que a origem de um sistema de coordenadas está localizada no centro de massa (veja a Figura~\ref{Fig:PlacaRetangularMomentoInercia}), podemos determinar o momento de inércia através de
\begin{align}
    I &= \int r_{\perp}^2 \;dm \\
    &= \sigma \int_{-\nicefrac{a}{2}}^{\nicefrac{a}{2}} \int_{-\nicefrac{b}{2}}^{\nicefrac{b}{2}} r_\perp^2 \; dy\;dx.  
\end{align}
%
Note que $\vec{r}_\perp \equiv \vec{r}$, uma vez que o vetor $\vec{r}$ descreve a posição do diferencial de massa $dm$ em relação ao eixo de rotação, pois escolhemos a origem no mesmo local por onde passa o eixo de rotação. Assim,
\begin{equation}
    r_\perp^2 = r^2 = x^2 + y^2.
\end{equation}
%
Substituindo a expressão acima na integral e realizando a integração, obtemos
\begin{align}
    I &= \sigma \int_{-\nicefrac{a}{2}}^{\nicefrac{a}{2}} \int_{-\nicefrac{b}{2}}^{\nicefrac{b}{2}} r_\perp^2 \; dy\;dx. \\
    &= \sigma \frac{a^3b + b^3a}{12}.
\end{align}
%
Finalmente, usando $\sigma = M/(ab)$, obtemos
\begin{equation}
    I = M \frac{a^2 + b^2}{12}. \mathnote{Momento de inércia de uma placa em torno de um eixo que passa pelo centro de massa, perpendicularmente à face plana.}
\end{equation}


%%%%%%%%%%%%%%%%%%%%%%%%%%%%
\paragraph{Placa triangular}
%%%%%%%%%%%%%%%%%%%%%%%%%%%%


%%%%%%%%%%%%%%%%%%%%%%%%%%%%%%%%%%%%%%%%%%%%%%%%%%%%%%%%%%%%%%%%%%%%%%%%%%%%%%%
\section{Experimento}
%%%%%%%%%%%%%%%%%%%%%%%%%%%%%%%%%%%%%%%%%%%%%%%%%%%%%%%%%%%%%%%%%%%%%%%%%%%%%%%

%%%%%%%%%%%%%%%%%%%%%%
\subsection{Objetivos}
%%%%%%%%%%%%%%%%%%%%%%

\begin{itemize}
	\item Resultados concretos que devemos conseguir;
	\item Observar a relação de tal coisa com outra coisa;
	\item Calcular a constante $x$.
\end{itemize}

%%%%%%%%%%%%%%%%%%%%%%%%%%%%%%%%%%%%%%%%%%%%%%%%%%%%%%%%%%%%%%%%%%%%%%%%%%%%%%%
\section{Material Necessário}
%%%%%%%%%%%%%%%%%%%%%%%%%%%%%%%%%%%%%%%%%%%%%%%%%%%%%%%%%%%%%%%%%%%%%%%%%%%%%%%

\begin{itemize}
	\item Item 1;
	\item Item 2.
\end{itemize}

%%%%%%%%%%%%%%%%%%%%%%%%%%%%%%%%%%%%%%%%%%%%%%%%%%%%%%%%%%%%%%%%%%%%%%%%%%%%%%%
\section{Procedimento Experimental}
%%%%%%%%%%%%%%%%%%%%%%%%%%%%%%%%%%%%%%%%%%%%%%%%%%%%%%%%%%%%%%%%%%%%%%%%%%%%%%%

%%%%%%%%%%%%%%%%%%%%%
%\subsection{Parte A} % Se necessário
%%%%%%%%%%%%%%%%%%%%%
\begin{enumerate}
	\item Passo 1;
	\item Passo 2;
	\item Passo 3.
\end{enumerate}

%%%%%%%%%%%%%%%%%%%%%%%%%%%%%%%%%%%%%%%%%%%%%%%%%%%%%%%%%%%%%%%%%%%%%%%%%%%%%%%
%%%%%%%%%%%%%%%%%%%%%%%%%%%%%%%%%%%%%%%%%%%%%%%%%%%%%%%%%%%%%%%%%%%%%%%%%%%%%%%
%%%%%%%%%%%%%%%%%%%%%%%%%%%%%%%%%%%%%%%%%%%%%%%%%%%%%%%%%%%%%%%%%%%%%%%%%%%%%%%
%%%%%%%%%%%%%%%%%%%%%%%%%%%%%%%%%%%%%%%%%%%%%%%%%%%%%%%%%%%%%%%%%%%%%%%%%%%%%%%
\cleardoublepage

\noindent{}{\huge\textit{Pêndulo Físico}}

\vspace{15mm}

\begin{fullwidth}
\noindent{}\makebox[0.6\linewidth]{Turma:\enspace\hrulefill}\makebox[0.4\textwidth]{  Data:\enspace\hrulefill}
\vspace{5mm}

\noindent{}\makebox[0.6\linewidth]{Aluno(a):\enspace\hrulefill}\makebox[0.4\textwidth]{  Matrícula:\enspace\hrulefill}

\noindent{}\makebox[0.6\linewidth]{Aluno(a):\enspace\hrulefill}\makebox[0.4\textwidth]{  Matrícula:\enspace\hrulefill}

\noindent{}\makebox[0.6\linewidth]{Aluno(a):\enspace\hrulefill}\makebox[0.4\textwidth]{  Matrícula:\enspace\hrulefill}

\noindent{}\makebox[0.6\linewidth]{Aluno(a):\enspace\hrulefill}\makebox[0.4\textwidth]{  Matrícula:\enspace\hrulefill}

\noindent{}\makebox[0.6\linewidth]{Aluno(a):\enspace\hrulefill}\makebox[0.4\textwidth]{  Matrícula:\enspace\hrulefill}
\end{fullwidth}

\vspace{5mm}

%%%%%%%%%%%%%%%%%%%%%%%%%%%%%%%%%%%%%%%%%%%%%%%%%%%%%%%%%%%%%%%%%%%%%%%%%%%%%%%
\section{Questionário}
%%%%%%%%%%%%%%%%%%%%%%%%%%%%%%%%%%%%%%%%%%%%%%%%%%%%%%%%%%%%%%%%%%%%%%%%%%%%%%%
\emph{Nas questões seguintes, apresente os cálculos requisitados de maneira clara e sucinta, para que o professor possa acompanhar o raciocínio desenvolvido.}
\vspace{5mm}

\begin{question}[type={exam}]{2}
Lorem ipsum dolor sit amet, consectetuer adi-
piscing elit. Ut purus elit, vestibulum ut, placerat ac, adipiscing vitae,
felis. Curabitur dictum gravida mauris. Nam arcu libero, nonummy
eget, consectetuer id, vulputate a, magna. Donec vehicula augue
eu neque. Pellentesque habitant morbi tristique senectus et netus
et malesuada fames ac turpis egestas. Mauris ut leo. Cras viverra
metus rhoncus sem. Nulla et lectus vestibulum urna fringilla ultrices.
\end{question}

\begin{question}[type={exam}]{2}
Lorem ipsum dolor sit amet, consectetuer adi-
piscing elit. Ut purus elit, vestibulum ut, placerat ac, adipiscing vitae,
felis. Curabitur dictum gravida mauris. Nam arcu libero, nonummy
eget, consectetuer id, vulputate a, magna. Donec vehicula augue
eu neque. Pellentesque habitant morbi tristique senectus et netus
et malesuada fames ac turpis egestas. Mauris ut leo. Cras viverra
metus rhoncus sem. Nulla et lectus vestibulum urna fringilla ultrices.
\end{question}

\begin{question}[type={exam}]{2}
Lorem ipsum dolor sit amet, consectetuer adi-
piscing elit. Ut purus elit, vestibulum ut, placerat ac, adipiscing vitae,
felis. Curabitur dictum gravida mauris. Nam arcu libero, nonummy
eget, consectetuer id, vulputate a, magna. Donec vehicula augue
eu neque. Pellentesque habitant morbi tristique senectus et netus
et malesuada fames ac turpis egestas. Mauris ut leo. Cras viverra
metus rhoncus sem. Nulla et lectus vestibulum urna fringilla ultrices.
\end{question}

\begin{question}[type={exam}]{2}
Lorem ipsum dolor sit amet, consectetuer adi-
piscing elit. Ut purus elit, vestibulum ut, placerat ac, adipiscing vitae,
felis. Curabitur dictum gravida mauris. Nam arcu libero, nonummy
eget, consectetuer id, vulputate a, magna. Donec vehicula augue
eu neque. Pellentesque habitant morbi tristique senectus et netus
et malesuada fames ac turpis egestas. Mauris ut leo. Cras viverra
metus rhoncus sem. Nulla et lectus vestibulum urna fringilla ultrices.
\end{question}

\begin{question}[type={exam}]{2}
Lorem ipsum dolor sit amet, consectetuer adi-
piscing elit. Ut purus elit, vestibulum ut, placerat ac, adipiscing vitae,
felis. Curabitur dictum gravida mauris. Nam arcu libero, nonummy
eget, consectetuer id, vulputate a, magna. Donec vehicula augue
eu neque. Pellentesque habitant morbi tristique senectus et netus
et malesuada fames ac turpis egestas. Mauris ut leo. Cras viverra
metus rhoncus sem. Nulla et lectus vestibulum urna fringilla ultrices.
\end{question}
\vfill
%%%%%%%%%%%%%%%%%%%%%%%%%%%%%%%%%%%%%%%%%%%%%%%%%%%%%%%%%%%%%%%%%%%%%%%%%%%%%%%
\pagebreak
\section{Tabelas}
%%%%%%%%%%%%%%%%%%%%%%%%%%%%%%%%%%%%%%%%%%%%%%%%%%%%%%%%%%%%%%%%%%%%%%%%%%%%%%%

