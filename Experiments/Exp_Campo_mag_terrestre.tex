%%%%%%%%%%%%%%%%%%%%%%%%%%%%%%%%%%%%%%%%%%%%%%%%%%%%%%%%%%%%%%%%%%%%%%%%%%%%%%%
\chapter{Campo magnético terrestre} % Sem "Experiência 01" ou qualquer outro número
\label{Chap:CampoMagTerrestre}        % para poder trocar a ordem com facilidade
%%%%%%%%%%%%%%%%%%%%%%%%%%%%%%%%%%%%%%%%%%%%%%%%%%%%%%%%%%%%%%%%%%%%%%%%%%%%%%%

\begin{fullwidth}\it
	Que experimento faremos?
	Qual é o objetivo?
	O que veremos/revisaremos? (teoria física)
	Quais conceitos/técnicas de análise de dados utilizaremos?
\end{fullwidth}

%%%%%%%%%%%%%%%%%%%%%%%%%%%%%%%%%%%%%%%%%%%%%%%%%%%%%%%%%%%%%%%%%%%%%%%%%%%%%%%
\section{Campo magnético terrestre}
%%%%%%%%%%%%%%%%%%%%%%%%%%%%%%%%%%%%%%%%%%%%%%%%%%%%%%%%%%%%%%%%%%%%%%%%%%%%%%%

fazer uma introdução curta

%%%%%%%%%%%%%%%%%%%%%%%%%%%%%%%%%%%%%%%%%%%%%%%%
\subsection{Campo magnético devido a uma espira}
%%%%%%%%%%%%%%%%%%%%%%%%%%%%%%%%%%%%%%%%%%%%%%%%

% Determinar o campo magnético de uma espira ao longo do eixo $z$.

Podemos determinar o campo magnético de uma espira circular ao longo do eixo $z$ que passa por seu centro, perpendicularmente ao plano que contém a espira, através da lei de Biot-Savart:
\begin{equation}
    d\vec{B} = \frac{\mu_0}{4\pi} \frac{I\, d\vec{\ell} \times \hat{r}}{r^2}.
\end{equation}
%
No caso particular da espira circular, a equação acima descreve o campo $d\vec{B}$ gerado por um segmento de comprimento $d\vec{\ell}$ do condutor que porta uma corrente $I$.

Note que devido à simetria rotacional do problema, verificamos que sobre o eixo $z$ as componentes $B_x$ e $B_y$ do campo magnético são nulas, restando somente a componente $B_z$. Por essa razão, vamos nos preocupar somente com a determinação de $dB_z$ de agora em diante. Esta componente pode ser deteminada ao projetarmos o campo $dB$ na direção do eixo $z$ através de 
\begin{equation}
    dB_z = dB \sen\theta.
\end{equation}
%
Podemos determinar o módulo do campo magnético ao tomarmos a lei de Biot-Savart em módulo:
\begin{equation}
    dB = \frac{\mu_0}{4\pi} \frac{I\, |d\vec{\ell} \times \hat{r}|}{r^2}.
\end{equation}
%
Como $d\vec{\ell} \perp \hat{r}$, o módulo do produto vetorial entre esses dois vetores resulta em
\begin{equation}
    |d\vec{\ell} \times \hat{r}| = d\ell.
\end{equation}
%
Além disso, ainda que a distância $r$ entre o ponto onde estamos calculando campo e a posição do segmento $d\vec{\ell}$ pode ser escrita em termos de $z$ e do raio $R$ da espira:
\begin{equation}
    r = \sqrt{z^2 + R^2}.
\end{equation}
%
Assim,
\begin{align}
    dB_z &= dB \cdot \sin\theta \\
    &= \left(\frac{\mu_0}{4\pi} \cdot \frac{I \,d\ell}{z^2 + R^2}\right) \cdot \sen\theta \\
    &= \left(\frac{\mu_0}{4\pi} \cdot \frac{I \,d\ell}{z^2 + R^2}\right) \cdot \left(\frac{R}{\sqrt{z^2+R^2}}\right) \\
    &= \frac{\mu_0}{4\pi} \cdot \frac{IR \,d\ell}{(z^2 + R^2)^{\nicefrac{3}{2}}}.
\end{align}
%
Para determinarmos o campo total na direção do eixo $z$, basta agora realizarmos a integração ao longo do caminho formado pela espira:
\begin{align}
    B_z(z) &= \oint \frac{\mu_0}{4\pi} \cdot \frac{IR \,d\ell}{(z^2 + R^2)^{\nicefrac{3}{2}}} \\
    &= \frac{\mu_0}{4\pi} \cdot \frac{IR}{(z^2 + R^2)^{\nicefrac{3}{2}}} \oint d\ell \\
    &= \frac{\mu_0}{4\pi} \cdot \frac{IR}{(z^2 + R^2)^{\nicefrac{3}{2}}} \cdot 2\pi R \\
    &= \frac{\mu_0}{2} \cdot \frac{IR^2}{(z^2 + R^2)^{\nicefrac{3}{2}}}. \label{Eq:CampoMagneticoEspiraCircular}
\end{align}

No caso de um grupo de $N$ espiras de mesmas dimensões, próximas umas das outras, podemos escrever o campo magnético em termos da corrente $i$ de uma espira como
\begin{equation}\label{Eq:CampoMagneticoGrupoEspiras}
    B_z(z) = \frac{\mu_0}{2} \cdot \frac{iNR^2}{(z^2 + R^2)^{\nicefrac{3}{2}}}.
\end{equation}
    

%%%%%%%%%%%%%%%%%%%%%%%%%%%%%%%%%
\subsection{Bobinas de Helmholtz}
%%%%%%%%%%%%%%%%%%%%%%%%%%%%%%%%%

As bobinas de Helmholtz são uma configuração particular de dois grupos paralelos de espiras de mesmas dimensões, separadas por uma distância $h = R$. A corrente $i$ nas espiras e seu sentido  devem ser iguais nos dois grupos de espiras. Nesse caso, o campo na região central dos grupos de espiras é bastante uniforme, tornando esse tipo de aparato útil para diversos tipos de aplicação. 

Para determinarmos o campo na região central, basta somarmos o campo $B_1$ ---~devido ao primeiro grupo de espiras~--- e o campo $B_2$ ---~devido ao segundo grupo~---. Note que a Equação~\eqref{Eq:CampoMagneticoGrupoEspiras} foi deduzida considerando que a origem do sistema de coordenadas estava localizada no centro da espira. Agora gostaríamos que a posição central entre as espiras fosse a origem de um eixo $z'$ e que as bobinas estivessem localizadas nas posições $\pm R/2$. Nesse caso, temos que as variáveis $z_1$ e $z_2$ das expressões originais podem ser escritas em termos de $z'$ como
\begin{align}
    z_1 &= z' + R/2 \\
    z_2 &= z' - R/2,
\end{align}
%
para as bobinas localizadas nas posições $z'_1 = -R/2$ e $z'_2 = R/2$. Assim,
\begin{align}
    B_H(z') &= B_1 + B_2 \\
    &= \frac{\mu_0 N i}{2} \cdot \frac{R^2}{[(z' + R/2)^2 + R^2]^{\nicefrac{3}{2}}} + \frac{\mu_0 N i}{2} \cdot \frac{R^2}{[(z' - R/2)^2 + R^2]^{\nicefrac{3}{2}}} \\
    &= \frac{\mu_0 N i}{2} \cdot \left(\frac{R^2}{[(z' + R/2)^2 + R^2]^{\nicefrac{3}{2}}} + \frac{R^2}{[(z' - R/2)^2 + R^2]^{\nicefrac{3}{2}}}\right). \label{Eq:CampoMagneticoBobinasDeHelmholtz}
\end{align}

\begin{marginfigure}
\centering
\begin{tikzpicture}[>=Stealth, yscale = 1.5, xscale = 1.25, extended line/.style={shorten >=-#1,shorten <=-#1},
 extended line/.default=3mm]] 
    \draw[->] (0,0) -- (0,1.25) node[below left] {$B_z$};
	\draw[->] (-1.75,0) -- (1.75,0) node[below left] {$z$};

    % Desenhar função:
    \def\R{1}
    \draw[smooth, name path=plot,samples=1000,domain=-1.5:1.5]
    plot(\x,{pow(\R, 2)/(2 * pow(pow(\x - \R/2, 2) + pow(\R, 2), 3/2)) + pow(\R, 2)/(2 * pow(pow(\x + \R/2, 2) + pow(\R, 2), 3/2))});

    \draw[smooth, dashed, name path=plot2,samples=1000,domain=-1.5:1.5]
    plot(\x,{pow(\R, 2)/(2 * pow(pow(\x - \R/2, 2) + pow(\R, 2), 3/2))});
    
    \draw[smooth, dashed, name path=plot3,samples=1000,domain=-1.5:1.5]
    plot(\x,{pow(\R, 2)/(2 * pow(pow(\x + \R/2, 2) + pow(\R, 2), 3/2))});
    
    % Note que o 1 abaixo na verdade corresponde ao valor de \R acima
    \draw (0, 0.0) -- +(0,-0.1) node[below]{$0$};
    \draw (1, 0.0) -- +(0,-0.1) node[below]{$R$};
    \draw (-1, 0.0) -- +(0,-0.1) node[below]{$-R$};
    
	\end{tikzpicture}
\caption{Intensidade do campo magnético no espaço entre as bobinas. As linhas tracejadas denotam os campos devidos às duas bobinas separadamente. Note que o as intensidades máximas dessas duas curvas correspondem às posições $\pm R/2$, que são as posições das bobinas no eixo $z$. A linha cheia denota o campo total devido às duas bobinas. Note que entre $z = -R/2$ e $z = R/2$ temos que a intensidade é aproximadamente contante. \label{Fig:CampoMagneticoBobinasDeHelmholtz}}
\end{marginfigure}

Na Figura~\ref{Fig:CampoMagneticoBobinasDeHelmholtz} temos um gráfico da intensidade do campo magnético na região central entre as bobinas e podemos verificar que seu valor é aproximadamente constante. Para determinar tal valor, basta fazermos $z' = 0$ na Equação~\eqref{Eq:CampoMagneticoBobinasDeHelmholtz}:
\begin{align}
    B_H(z'=0) &=  \frac{\mu_0 N i}{2} \cdot \left(\frac{R^2}{[(z' + R/2)^2 + R^2]^{\nicefrac{3}{2}}} + \frac{R^2}{[(z' - R/2)^2 + R^2]^{\nicefrac{3}{2}}}\right) \\
    &= \frac{\mu_0 N i}{2} \cdot \left(\frac{R^2}{[(0 + R/2)^2 + R^2]^{\nicefrac{3}{2}}} + \frac{R^2}{[(0 - R/2)^2 + R^2]^{\nicefrac{3}{2}}}\right) \\
    &=  \frac{\mu_0 N i}{2} \cdot \left(\frac{R^2}{[R^2/4 + R^2]^{\nicefrac{3}{2}}} + \frac{R^2}{[R^2/4)^2 + R^2]^{\nicefrac{3}{2}}}\right) \\
    &= \frac{\mu_0 N i}{2} \cdot \left(\frac{R^2}{[5R^2/4]^{\nicefrac{3}{2}}} + \frac{R^2}{[5R^2/4)^2]^{\nicefrac{3}{2}}}\right).
\end{align}
%
Note que os dois termos dentro dos parêntesis são iguais, logo,
\begin{align}
    B_H(z'=0) &= \frac{\mu_0 N i}{2} \cdot \left(2\frac{R^2}{[5R^2/4]^{\nicefrac{3}{2}}}\right) \\
    &= \frac{\mu_0 N i}{2} \cdot \left(2\frac{R^2}{R^3[5/4]^{\nicefrac{3}{2}}}\right) \\
    &= \frac{\mu_0 N i}{R} \cdot \left(\frac{1}{[5/4]^{\nicefrac{3}{2}}}\right) \\
    &= \frac{\mu_0 N i}{R} \cdot \left(\frac{4}{5}\right)^{\rlap{\nicefrac{3}{2}}}.
\end{align}

%%%%%%%%%%%%%%%%%%%%%%%%%%%%%%%%%%%%%%%%%%%%%%%%%%%%%%
\subsection{Determinação do campo magnético terrestre}
%%%%%%%%%%%%%%%%%%%%%%%%%%%%%%%%%%%%%%%%%%%%%%%%%%%%%%

Podemos utilizar uma bússola e as bobinas de Helmholtz para determinar a intensidade da componente horizontal do campo magnético da Terra. Isso é possível pois uma bússola na verdade simplesmente aponta na direção do campo magnético total no ponto onde está localizada, não necessariamente para o verdadeiro norte magnético.

\begin{marginfigure}
\centering
\begin{tikzpicture}[>=Stealth] 
    \draw[->] (0,0) -- (0,3) node[below left] {$x'$};
	\draw[->] (0,0) -- (4,0) node[below left] {$z'$};

    \draw[thick, ->] (0,0) -- (0,2) node[below left] {$\vec{B}_H$};
	\draw[thick, ->] (0,0) -- (2.5,0) node[below left] {$\vec{B}_T$};
	
	\draw[thick, ->] (0,0) -- (2.5,2) node[right] {$\vec{B}$};
	
	\draw[dotted] (0,2) -- (2.5,2) -- (2.5,0);
	
	\draw[->] (1,0) arc (0:38.66:1) node[midway, right]{$\phi$};

\end{tikzpicture}
\caption{O campo total $\vec{B}$ é dado pela soma vetorial dos campos da Terra $\vec{B}_T$ e das bobinas $\vec{B}_H$. \label{Fig:CampoMagneticoTotalSobreBussola}}
\end{marginfigure}

Nesse caso, se dispusermos a bússola e as bobinas de forma que:
\begin{itemize}
    \item A bússola esteja localizada na região central entre as bobinas e aponte inicialmente para o norte;
    \item O eixo das bobinas (denotado por $z'$ na discussão acima) aponte perpendicularmente à direção da bússola;
\end{itemize}
%
teremos uma situação como a mostrada na Figura~\ref{Fig:CampoMagneticoTotalSobreBussola}. Note que ao controlarmos a corrente que passa pelas bobinas, determinamos o campo magnético $\vec{B}_H$ e a direção do vetor $\vec{B}$, uma vez que o campo $\vec{B}_T$ é constante.

Para determinarmos o campo magnético da Terra, verificamos que ---~usando a definição da função tangente~---
\begin{equation}
    \tan \phi = \frac{B_H}{B_T}.
\end{equation}
%
Consequentemente, podemos escrever a relação
\begin{equation}
    B_H = B_T \cdot \tan \phi.
\end{equation}
%
Portanto, se fizermos medidas da deflexão da agulha da bússola e do campo magnético $B_H$ correspondente, podemos determinar o campo magnético $B_T$ da Terra ao fazermos uma regressão linear dos dados de um gráfico $B_H \times \tan\phi$.

Note que podemos inverter o sentido da corrente nas bobinas, com uma consequente alteração no sentido do campo $\vec{B}_H$, o que causará uma deflexão da agulha da bússola no sentido oposto ao registrado com o sentido inicial para a corrente. Podemos usar desse artifício para obter mais dados experimentais e melhorar o resultado obtido para o valor do campo magnético $B_T$. Para isso, vamos registrar os dados para o ângulo $\phi$ contidos no quarto quadrante como ângulos negativos, medidos no sentido horário a partir do eixo $z'$. Correspondentemente, vamos considerar os valores de $B_H$ que apontam no sentido negativo de $x'$ como negativos.


%%%%%%%%%%%%%%%%%%%%%%%%%%%%%%%%%%%%%%%%%%%%%%%%%%%%%%%%%%%%%%%%%%%%%%%%%%%%%%%
\section{Experimento}
%%%%%%%%%%%%%%%%%%%%%%%%%%%%%%%%%%%%%%%%%%%%%%%%%%%%%%%%%%%%%%%%%%%%%%%%%%%%%%%

%%%%%%%%%%%%%%%%%%%%%%
\subsection{Objetivos}
%%%%%%%%%%%%%%%%%%%%%%

\begin{itemize}
	\item Determinar a componente horizontal do campo magnético da Terra.
\end{itemize}

%%%%%%%%%%%%%%%%%%%%%%%%%%%%%%%%%%%%%%%%%%%%%%%%%%%%%%%%%%%%%%%%%%%%%%%%%%%%%%%
\section{Material Necessário}
%%%%%%%%%%%%%%%%%%%%%%%%%%%%%%%%%%%%%%%%%%%%%%%%%%%%%%%%%%%%%%%%%%%%%%%%%%%%%%%

\begin{itemize}
	\item Bobinas de Helmholtz;
	\item Fonte de tensão regulável;
	\item Multímetro;
	\item Cabos para ligação.
\end{itemize}

%%%%%%%%%%%%%%%%%%%%%%%%%%%%%%%%%%%%%%%%%%%%%%%%%%%%%%%%%%%%%%%%%%%%%%%%%%%%%%%
\section{Procedimento Experimental}
%%%%%%%%%%%%%%%%%%%%%%%%%%%%%%%%%%%%%%%%%%%%%%%%%%%%%%%%%%%%%%%%%%%%%%%%%%%%%%%

\begin{marginfigure}[2cm]
\centering
\begin{circuitikz}[american]
	\draw (0,1) to[battery1, v=$V$, i=$i$] (0,3) to[switch] (2,3) to[L, l=$L$] (2,1) to[L, l=$L$] (2,-1) to[smeter, t=A] (0,-1) to[R, l=$R$] (0,1);
	%\draw (2,3) -- (3.5,3)  (3.5,0) -- (2,0);
\end{circuitikz}
\caption{Circuito para as bobinas de Helmholtz.}
\end{marginfigure}

%%%%%%%%%%%%%%%%%%%%%
%\subsection{Parte A} % Se necessário
%%%%%%%%%%%%%%%%%%%%%
\begin{enumerate}
	\item Passo 1;
	\item Passo 2;
	\item Passo 3.
\end{enumerate}

%%%%%%%%%%%%%%%%%%%%%%%%%%%%%%%%%%%%%%%%%%%%%%%%%%%%%%%%%%%%%%%%%%%%%%%%%%%%%%%
%%%%%%%%%%%%%%%%%%%%%%%%%%%%%%%%%%%%%%%%%%%%%%%%%%%%%%%%%%%%%%%%%%%%%%%%%%%%%%%
%%%%%%%%%%%%%%%%%%%%%%%%%%%%%%%%%%%%%%%%%%%%%%%%%%%%%%%%%%%%%%%%%%%%%%%%%%%%%%%
%%%%%%%%%%%%%%%%%%%%%%%%%%%%%%%%%%%%%%%%%%%%%%%%%%%%%%%%%%%%%%%%%%%%%%%%%%%%%%%
\cleardoublepage

\noindent{}{\huge\textit{Campo magnético terrestre}}

\vspace{15mm}

\begin{fullwidth}
\noindent{}\makebox[0.6\linewidth]{Turma:\enspace\hrulefill}\makebox[0.4\textwidth]{  Data:\enspace\hrulefill}
\vspace{5mm}

\noindent{}\makebox[0.6\linewidth]{Aluno(a):\enspace\hrulefill}\makebox[0.4\textwidth]{  Matrícula:\enspace\hrulefill}

\noindent{}\makebox[0.6\linewidth]{Aluno(a):\enspace\hrulefill}\makebox[0.4\textwidth]{  Matrícula:\enspace\hrulefill}

\noindent{}\makebox[0.6\linewidth]{Aluno(a):\enspace\hrulefill}\makebox[0.4\textwidth]{  Matrícula:\enspace\hrulefill}

\noindent{}\makebox[0.6\linewidth]{Aluno(a):\enspace\hrulefill}\makebox[0.4\textwidth]{  Matrícula:\enspace\hrulefill}

\noindent{}\makebox[0.6\linewidth]{Aluno(a):\enspace\hrulefill}\makebox[0.4\textwidth]{  Matrícula:\enspace\hrulefill}
\end{fullwidth}

\vspace{5mm}

%%%%%%%%%%%%%%%%%%%%%%%%%%%%%%%%%%%%%%%%%%%%%%%%%%%%%%%%%%%%%%%%%%%%%%%%%%%%%%%
\section{Questionário}
%%%%%%%%%%%%%%%%%%%%%%%%%%%%%%%%%%%%%%%%%%%%%%%%%%%%%%%%%%%%%%%%%%%%%%%%%%%%%%%

\begin{question}[type={exam}]{1}
Apresente os resultados de maneira clara e organizada. Mostre os cálculos requisitados de maneira clara e sucinta, evidenciando o raciocínio desenvolvido.
\end{question}

\begin{question}[type={exam}]{1}
Preencha as colunas de dados experimentais das tabelas com o número adequado de algarismos significativos e unidades.
\end{question}

\begin{question}[type={exam}]{2}
Lorem ipsum dolor sit amet, consectetuer adi-
piscing elit. Ut purus elit, vestibulum ut, placerat ac, adipiscing vitae,
felis. Curabitur dictum gravida mauris. Nam arcu libero, nonummy
eget, consectetuer id, vulputate a, magna. Donec vehicula augue
eu neque. Pellentesque habitant morbi tristique senectus et netus
et malesuada fames ac turpis egestas. Mauris ut leo. Cras viverra
metus rhoncus sem. Nulla et lectus vestibulum urna fringilla ultrices.
\end{question}

\begin{question}[type={exam}]{2}
Lorem ipsum dolor sit amet, consectetuer adi-
piscing elit. Ut purus elit, vestibulum ut, placerat ac, adipiscing vitae,
felis. Curabitur dictum gravida mauris. Nam arcu libero, nonummy
eget, consectetuer id, vulputate a, magna. Donec vehicula augue
eu neque. Pellentesque habitant morbi tristique senectus et netus
et malesuada fames ac turpis egestas. Mauris ut leo. Cras viverra
metus rhoncus sem. Nulla et lectus vestibulum urna fringilla ultrices.
\end{question}

\begin{question}[type={exam}]{2}
Lorem ipsum dolor sit amet, consectetuer adi-
piscing elit. Ut purus elit, vestibulum ut, placerat ac, adipiscing vitae,
felis. Curabitur dictum gravida mauris. Nam arcu libero, nonummy
eget, consectetuer id, vulputate a, magna. Donec vehicula augue
eu neque. Pellentesque habitant morbi tristique senectus et netus
et malesuada fames ac turpis egestas. Mauris ut leo. Cras viverra
metus rhoncus sem. Nulla et lectus vestibulum urna fringilla ultrices.
\end{question}

\begin{question}[type={exam}]{2}
Lorem ipsum dolor sit amet, consectetuer adi-
piscing elit. Ut purus elit, vestibulum ut, placerat ac, adipiscing vitae,
felis. Curabitur dictum gravida mauris. Nam arcu libero, nonummy
eget, consectetuer id, vulputate a, magna. Donec vehicula augue
eu neque. Pellentesque habitant morbi tristique senectus et netus
et malesuada fames ac turpis egestas. Mauris ut leo. Cras viverra
metus rhoncus sem. Nulla et lectus vestibulum urna fringilla ultrices.
\end{question}

\begin{question}[type={exam}]{2}
Lorem ipsum dolor sit amet, consectetuer adi-
piscing elit. Ut purus elit, vestibulum ut, placerat ac, adipiscing vitae,
felis. Curabitur dictum gravida mauris. Nam arcu libero, nonummy
eget, consectetuer id, vulputate a, magna. Donec vehicula augue
eu neque. Pellentesque habitant morbi tristique senectus et netus
et malesuada fames ac turpis egestas. Mauris ut leo. Cras viverra
metus rhoncus sem. Nulla et lectus vestibulum urna fringilla ultrices.
\end{question}
\vfill
%%%%%%%%%%%%%%%%%%%%%%%%%%%%%%%%%%%%%%%%%%%%%%%%%%%%%%%%%%%%%%%%%%%%%%%%%%%%%%%
\pagebreak
\section{Tabelas}
%%%%%%%%%%%%%%%%%%%%%%%%%%%%%%%%%%%%%%%%%%%%%%%%%%%%%%%%%%%%%%%%%%%%%%%%%%%%%%%

