%%%%%%%%%%%%%%%%%%%%%%%%%%%%%%%%%%%%%%%%%%%%%%%%%%%%%%%%%%%%%%%%%%%%%%%%%%%%%%%
\chapter{Roda de Maxwell} % Sem "Experiência 01" ou qualquer outro número
\label{Chap:RodaDeMaxwell}        % para poder trocar a ordem com facilidade
%%%%%%%%%%%%%%%%%%%%%%%%%%%%%%%%%%%%%%%%%%%%%%%%%%%%%%%%%%%%%%%%%%%%%%%%%%%%%%%

\tikzset{
    partial ellipse/.style args={#1:#2:#3}{
        insert path={+ (#1:#3) arc (#1:#2:#3)}
    }
}

\begin{fullwidth}\it
	Vamos realizar um experimento que envolve um movimento combinado de rotação e de translação. Verificaremos as características do movimento de rotação, buscando determinar o momento de inércia de um corpo rígido experimentalmente, verificando se o seu valor corresponde ao determinado teoricamente. Revisaremos a Segunda Lei de Newton para a rotação e as técnicas para a obtenção do momento de inércia. \textbf{Quais conceitos/técnicas de análise de dados utilizaremos?}
\end{fullwidth}

%%%%%%%%%%%%%%%%%%%%%%%%%%%%%%%%%%%%%%%%%%%%%%%%%%%%%%%%%%%%%%%%%%%%%%%%%%%%%%%
\section{Física do experimento}
%%%%%%%%%%%%%%%%%%%%%%%%%%%%%%%%%%%%%%%%%%%%%%%%%%%%%%%%%%%%%%%%%%%%%%%%%%%%%%%

\begin{marginfigure}[5cm]
\centering
\begin{tikzpicture}[>=Stealth, scale = 1.4,
     interface/.style={
        % superfície
        postaction={draw,decorate,decoration={border,angle=-45,
                    amplitude=0.2cm,segment length=2mm}}},
    ]

%%% Figura superior

\draw (0,0) ellipse (1.25 and 0.5);

\draw (-1.25,0) -- (-1.25,-0.4);
\draw (1.25,-0.4) -- (1.25,0);  

\draw (-1.25,-0.4) arc (180:360:1.25 and 0.5);
\draw[densely dotted] (-1.25,-0.4) arc (180:360:1.25 and -0.5);

\draw[dashdotted,->] (0,0) -- (0,1.5) node[below left]{$z$};
\draw[dashdotted] (0,-1.5) -- (0,-0.9);
\draw[dotted] (0,-0.9) -- (0,0);
\draw[fill] (0,0) circle (1pt);

\draw[dashdotted] (15:-2) -- (0,0);

% raio
\draw[|-|] (-0.05,0.15) -- node[above]{$r_\perp^i$} +(15:-0.8);
\draw[fill] (15:-0.8) circle (1pt);

% força
\draw[thick, ->] (15:-0.8) -- +(-65:1) node[below]{$\vec{F}_j^i$};

% projeção radial
\draw[dashed] (15:-0.8) ++(-65:1) -- (15:-1.6); 
\draw[->] (15:-0.8) -- (15:-1.6) node[above]{$F_{j,r}^i$};

% projeção tangencial
\draw[dashed] (15:-0.8) +(-30:1.6) -- +(-30:-0.3);
\draw[dashed] (15:-0.8) ++(-65:1) -- +(18:0.8);
\draw[->] (15:-0.8) -- +(-30:1.35) node[above]{$F_{j,t}^i$};

% phi
\draw (15:-1.2) arc (150:117:1.25 and -0.5);
\node (theta) at (30:-1.1) {$\phi_j^i$};

\end{tikzpicture}
\caption{Sobre cada uma das partículas que compõe um corpo rígido, age uma grande quantidade de forças internas e externas. Na figura tomamos uma partícula $i$ qualquer e analisamos o efeito de uma força $\vec{F}_j^i$ que atua sobre ela, obtendo as componentes nas direções radial e tangencial.\label{Fig:CalculoSegundaLeiRotacoes}}
\end{marginfigure}

Para verificar a relação entre o torque e aceleração, podemos analisar um objeto como o da Figura~\ref{Fig:CalculoSegundaLeiRotacoes}. Vamos supor que sobre cada partícula do corpo sejam exercidas forças internas e externas, sendo que cada força que atua sobre a $i$-ésima partícula é representada por $F_j^i$, onde $j$ é um índice que enumera cada uma das forças que atuam sobre tal partícula. Se analisarmos o torque devido a $j$-ésima força que atua sobre a $i$-ésima partícula, temos
\begin{align}
	\tau_j^i &= r_\perp^i F_j^i \sen\phi_j^i \\
	&= r_\perp^i F_{j,t}^i
\end{align}
%
mas
\begin{equation}
	F_{j,t}^{i} = m_i a_{j,t}^{i},
\end{equation}
%
onde $a_{j,t}^i$ é a aceleração  tangencial que a $i$-ésima partícula teria se fosse submetida à $j$-ésima força isoladamente.
%
Logo,
\begin{equation}
	\tau_j^i = m_i a_{j,t}^{i} r_\perp^i.
\end{equation}
%
Se somarmos todos os torques que atuam sobre todas as partículas que compõe o corpo, é possível mostrar que
\begin{equation}
	\sum_i \sum_j \tau_j^i = \sum_{i}[m_i r_\perp^i a_{t}^i].
\end{equation}

Podemos utilizar a relação $a_t = \alpha r$ na expressão acima para escrever $a_{t}^i$ como $\alpha r_\perp^i$, o que resulta em
\begin{equation}
	\sum_i \sum_j \tau_j^i = \sum_{i}[m_i (r_\perp^i)^2 \alpha].
\end{equation}
%
No resultado acima, a aceleração angular $\alpha$ é comum a todos os pontos do corpo rígido, por isso podemos escrever
\begin{equation}
	\sum_i \sum_j \tau_j^i =  \left[\sum_{i} m_i (r_\perp^i)^2\right] \alpha.
\end{equation}

Podemos observar que a soma à esquerda da igualdade inclui todos os torques realizados sobre todas as partículas por todas as forças que atuam no sistema, sejam elas internas ou externas. No entanto, como as forças internas estão presentes aos pares, sendo que cada força do par têm o mesmo módulo e direção, porém sentidos contrários, verificamos que as forças de ação e reação geram torques que se equilibram. Assim, podemos escrever a soma dos torques como a \emph{soma dos torques externos}, ou seja, o \emph{torque resultante externo}:
\begin{equation}\label{Eq:ProtoSegundaLeiRotacoes}
    \tau_R^{\rm{ext}} = \left[\sum_{i} m_i (r_\perp^i)^2\right] \alpha.
\end{equation}
% 

Na expressão acima, o termo entre colchetes é denominado como \emph{momento de inércia}:
\begin{equation}\label{Eq:MomentoInerciaConjPart}
    I = \left[\sum_{i = 1}^N m_i (r_\perp^i)^2\right], \mathnote{Momento de inércia para um conjunto de partículas}
\end{equation}
%
e suas unidades são $[I] = \rm{M}\cdot\rm{L}^2$, ou, dentro do SI, $[I] = \rm{kg}\cdot\rm{m}^2$. A expressão acima é análoga à Segunda Lei de Newton para a translação, porém descrevendo o caso da rotação, e a denominamos como \emph{segunda Lei de Newton para a Rotação}:\footnote[][-5mm]{Na verdade, essa relação foi determinada por Leonard Euler.}
\begin{equation}
    \tau_R^{\rm{ext}} = I \alpha. \mathnote{Segunda Lei de Newton para a rotação}
\end{equation}

É importante notar que o momento de inércia, definido pela Equação~\eqref{Eq:MomentoInerciaConjPart}, tem uma dependência na \emph{forma} do objeto: devido ao fato de que existe uma dependência no quadrado da distância $r_i$ entre a partícula e o eixo de rotação, temos uma dependência na forma como a massa está distribuída. Se, por exemplo, temos um disco girando em torno de um eixo que passa por seu centro de massa, perpendicularmente à sua face plana, temos partículas distribuídas por toda a distância entre o eixo e a borda externa do disco. Se tomarmos um aro sustentado por raios finos, de forma que a massa e o diâmetro sejam iguais aos do disco, devemos ter um momento de inércia maior para uma rotação em torno de um eixo que também passe pelo centro de massa, perpendicularmente à face plana. Isso se deve ao fato de que a maior parte das partículas se localizará a uma distância maior do eixo de rotação.

%%%%%%%%%%%%%%%%%%%%%%%%%%%%%%%%%%%%%%%
\section{Cálculo do momento de inércia}
%%%%%%%%%%%%%%%%%%%%%%%%%%%%%%%%%%%%%%%

Para que possamos aplicar a Segunda Lei de Newton para a rotação, é fundamental que saibamos qual é o momento de inércia do corpo em questão para o eixo em torno do qual a rotação ocorre. Como é possível verificar na Expressão~\ref{Eq:MomentoInerciaConjPart}, o momento de inércia depende da distância de cada uma das partículas em relação ao eixo de rotação. Isso nos indica que é importante determinar uma série de características particulares a cada situação que estivermos analisando.

\begin{marginfigure}[-2cm]
\centering
\begin{tikzpicture}[>=Stealth, rotate = -90,
     interface/.style={
        % superfície
        postaction={draw,decorate,decoration={border,angle=-45,
                    amplitude=0.2cm,segment length=2mm}}},
    ]

%%% Disco

\draw (0,0) ellipse (1.25 and 0.75);

\draw (-1.25,0) -- (-1.25,-0.4);
\draw (1.25,-0.4) -- (1.25,0);  

\draw (-1.25,-0.4) arc (180:360:1.25 and 0.75);
\draw[dotted] (-1.25,-0.4) arc (180:360:1.25 and -0.75);

\draw[dashdotted,<-] (-2,-0.2) node[below left]{$z$} -- (-1.25,-0.2);
\draw[loosely dotted] (-1.25,-0.2) -- (1.25,-0.2);
\draw[dashdotted] (1.25,-0.2) -- (1.75, -0.2);

\draw[->] (-1.5,-0.2) [partial ellipse=-120:150:0.125cm and 0.3cm];

%%% aro

\draw (4,0) ellipse (1.25 and 0.75);
\draw (4,0) ellipse (1 and 0.55);
\draw[dotted] (4,-0.4) ellipse (1 and 0.55);
\draw (4,-0.4) [partial ellipse=22:158:1cm and 0.55cm];

\draw (2.75,0) -- (2.75,-0.4);
\draw (5.25,-0.4) -- (5.25,0);  

\draw (2.75,-0.4) arc (180:360:1.25 and 0.75);
\draw[dotted] (2.75,-0.4) arc (180:360:1.25 and -0.75);

\draw[dashdotted,<-] (2,-0.2) node[below left]{$z$} -- (2.75,-0.2);
\draw[dotted] (2.75,-0.2) -- (3.15,-0.2);
\draw[dashdotted] (3.15,-0.2) -- (4.9,-0.2);
\draw[loosely dotted] (4.9,-0.2) -- (5.25,-0.2);
\draw[dashdotted] (5.25,-0.2) -- (5.75, -0.2);

\draw[->] (2.5,-0.2) [partial ellipse=-120:150:0.125cm and 0.3cm];

\end{tikzpicture}
\caption{Um disco e um anel de mesma massa e raio têm momentos de inércia diferentes em virtude das diferentes \emph{distribuições} de massa.\label{Fig:CompMomInerciaDiscoAnel}}
\end{marginfigure}

Dentre as principais características do momento de inércia, temos as seguintes dependências, cuja origem é a própria dependência na distância entre o eixo de rotação e a posição de cada partícula do corpo:
\begin{description}
    \item[Dependência na forma do objeto:] Se tomarmos um disco e um anel de mesma massa e raio, como na Figura~\ref{Fig:CompMomInerciaDiscoAnel}, o momento de inércia em torno do eixo $z$ mostrado acima é maior para o anel. Podemos compreender esse fato ao verificarmos que no caso do disco uma parte da massa, localizada na região central, se encontra próxima ao eixo de rotação. No caso do anel, essa massa está localizada a uma distância maior do eixo de rotação. Note que para que ambos os corpos possam ter a mesma massa e o mesmo raio, o material de que é feito o anel deve ser mais denso que o material do disco.
    
\begin{marginfigure}[-1.0cm]
\centering
\begin{tikzpicture}[>=Stealth, scale = 1.2,
     interface/.style={
        % superfície
        postaction={draw,decorate,decoration={border,angle=-45,
                    amplitude=0.2cm,segment length=2mm}}},
    ]

\draw[fill] (0,0) circle (0.6pt);
\draw[fill] (0,-0.2) circle (1pt) node[above left] {CM};
\draw (0,0) ellipse (1.25 and 0.5);
\draw[fill] (0,-0.4) circle (0.6pt);

\draw (-1.25,0) -- (-1.25,-0.4);
\draw (1.25,-0.4) -- (1.25,0);  

\draw (-1.25,-0.4) arc (180:360:1.25 and 0.5);
\draw[densely dotted] (-1.25,-0.4) arc (180:360:1.25 and -0.5);

% Eixo z
\draw[dashdotted,->] (0,0) -- (0,1.5) node[below left]{$z$};
\draw[dashdotted] (0,-1.5) -- (0,-0.9);
\draw[loosely dotted] (0,-0.9) -- (0,0);

\draw[->] (0,1.05) [partial ellipse=-225:60:0.3cm and 0.125cm];

% Eixo x
\draw[dashdotted, ->] (0,-0.2)++(15:-1.05) -- +(15:-0.75) node[below right]{$x$};
\draw[fill] (0,-0.2)+(15:-1.05) circle (0.6pt);
\draw[loosely dotted] (0,-0.2)+(15:-1.05) -- (0,-0.2) -- +(15:1.05);
\draw[fill] (0,-0.2) +(15:1.05) circle (0.6pt);

\draw[dashdotted] (0,-0.2) ++(15:1.25) -- +(15:0.75);

\draw[->, rotate=15] (0,-0.2)++(0:-1.5) [partial ellipse=-130:160:0.105cm and 0.3cm];

% Eixo w
\draw[dashdotted, <-] (0,-0.2)+(60:-1.38) node[below right]{$w$} -- +(60:-0.78);
\draw[fill] (0,-0.2)+(60:-0.78) circle (0.6pt);
\draw[loosely dotted] (0,-0.2)+(60:-0.78) -- +(60:0.78);
\draw[fill] (0,-0.2)+(60:0.78) circle (0.6pt);
\draw[dashdotted] (0,-0.2)+(60:0.78) -- +(60:1.18);

\draw[->, rotate=60] (0,-0.2)++(-4:-1.2) [partial ellipse=-150:150:0.15cm and 0.3cm];

\end{tikzpicture}
\caption{A orientação do eixo de rotação em relação ao objeto determina a distância entre as partículas e o próprio eixo, o que faz com que o momento de inércia seja diferente para cada orientação. Na figura, os pequenos círculos pretos mostram pontos onde os eixos saem/entram no objeto. \label{Fig:MomInerciaDiscoVariosEixos}}
\end{marginfigure}

    \item[Dependência no eixo em que ocorre a rotação:] Mesmo para um corpo só, podemos ter diversos momentos de inércia diferentes. Na Figura~\ref{Fig:MomInerciaDiscoVariosEixos}, temos três eixos que atravessam um disco de maneiras diferentes, e que passam pelo centro de massa do corpo. Como a orientação do eixo muda a distância entre as partículas que compõe o corpo e o eixo de rotação, ocorre uma mudança do momento de inércia. Dentre os três eixos mostrados, o momento de inércia é maior em relação ao eixo $z$, uma vez que ele minimiza a quantidade de massa próxima ao eixo.
    
\begin{marginfigure}[2cm]
\centering
\begin{tikzpicture}[>=Stealth, scale = 1.1,
     interface/.style={
        % superfície
        postaction={draw,decorate,decoration={border,angle=-45,
                    amplitude=0.2cm,segment length=2mm}}},
    ]

%%% Figura superior

\draw[fill] (0,0) circle (0.6pt);
\draw[fill] (0,-0.2) circle (1pt) node[above left] {CM};
\draw (0,0) ellipse (1.25 and 0.5);
\draw[fill] (0,-0.4) circle (0.6pt);

\draw (-1.25,0) -- (-1.25,-0.4);
\draw (1.25,-0.4) -- (1.25,0);  

\draw (-1.25,-0.4) arc (180:360:1.25 and 0.5);
\draw[densely dotted] (-1.25,-0.4) arc (180:360:1.25 and -0.5);

% Eixo z
\draw[dashdotted,->] (0,0) -- (0,1.5) node[below left]{$z$};
\draw[dashdotted] (0,-1.5) -- (0,-0.9);
\draw[dotted] (0,-0.9) -- (0,0);

\draw[->] (0,1.05) [partial ellipse=-225:60:0.3cm and 0.125cm];

% Eixos z', z''

\draw[dashdotted, ->] (-1.25,-1.5) -- (-1.25,1.5) node[below left]{$z'$};
\draw[->] (-1.25,1.05) [partial ellipse=-225:60:0.3cm and 0.125cm];

\draw[dashdotted, ->] (-2.5,-1.5) -- (-2.5,1.5) node[below left]{$z''$};
\draw[->] (-2.5,1.05) [partial ellipse=-225:60:0.3cm and 0.125cm];

\end{tikzpicture}
\caption{Mesmo que a orientação de diversos eixos em relação a um corpo seja a mesma, a distância em relação ao corpo também determina a distância das partículas em relação ao eixo de rotação. Veja que quando um corpo faz parte de um conjunto mais complexo, podemos ter uma rotação em relação a um eixo que não o atravessa. \label{Fig:MomInerciaRotEixosParalelos}}
\end{marginfigure}

    \item[Dependência na distância ao eixo de rotação:] Mesmo que tenhamos o cuidado de escolher eixos cuja orientação em relação ao corpo é a mesma, a distância entre o eixo de rotação e o próprio corpo é um fator determinante no cálculo do momento de inércia, já que isso causa uma variação da distância entre as partículas e o eixo. Na Figura~\ref{Fig:MomInerciaRotEixosParalelos} temos três eixos distintos, paralelos uns aos outros, em torno dos quais o corpo pode efetuar uma rotação. Devido a alteração das distâncias entre as partículas e o eixo, cada um deles terá um momento de inércia diferente. Devemos destacar em especial o eixo $z''$, pois ele não atravessa o corpo. Esse tipo de situação é relativamente comum quando analisamos corpos compostos de diversas partes, e tem grande influência no valor do momento de inércia. Verificaremos adiante que os momentos de inércia em eixos paralelos estão relacionados de forma bastante simples e essa relação deixará evidente que o momento de inércia mínimo é aquele associado ao eixo que passa pelo centro de massa do corpo.
    
\begin{marginfigure}[2cm]
\centering
\begin{tikzpicture}[>=Stealth]

%%% Figura superior

\draw[fill] (0,0) circle (0.6pt);
\draw[fill] (0,-0.2) circle (1pt) node[right] {CM};
\draw (0,0) ellipse (1.25 and 0.5);
\draw[fill] (0,-0.4) circle (0.6pt);

\draw (-1.25,0) -- (-1.25,-0.4);
\draw (1.25,-0.4) -- (1.25,0);  

%% arco inferior
\draw (-1.25,-0.4) arc (180:219:1.25 and 0.5);
\draw (1.25,-0.4) arc (0:-135:1.25 and 0.5);
\draw[densely dotted] (-1.25,-0.4) arc (180:360:1.25 and -0.5);

% Eixo z
\draw[dashdotted,->] (0,0) -- (0,1.5) node[below left]{$z$};
\draw[dashdotted] (0,-1.5) -- (0,-0.9);
\draw[dotted] (0,-0.9) -- (0,0);

\draw[->] (0,1.05) [partial ellipse=-225:60:0.3cm and 0.125cm];

% haste
\draw (0,-0.2)+(30:-2) ellipse (0.45mm and 0.5mm);
\draw (0,-0.25)+(30:-0.82) arc[start angle = -90, end angle = 90, radius = 0.5mm];
\draw (0,-0.146)+(30:-2) -- +(30:-0.82);
\draw (0,-0.254)+(30:-2) -- +(30:-0.82);

\end{tikzpicture}
\caption{Para um corpo complexo, podemos determinar o momento de inércia separando-o em partes simples, cujo momento de inércia sabemos determinar. \label{Fig:AditividadeMomInercia}}
\end{marginfigure}

    \item[Aditividade do momento de inércia:] Através da Equação~\ref{Eq:MomentoInerciaConjPart}, podemos mostrar que o momento de inércia de um corpo extenso é aditivo: sabemos que
\begin{equation}
    I = \sum_i m_i (r_\perp^i)^2,
\end{equation}
%
ou, se separarmos a soma em diferentes regiões,
\begin{align}
    I &= \sum_{i}^{R_1} m_i (r_\perp^i)^2 + \sum_{i}^{R_1} m_i (r_\perp^i)^2 + \sum_{i}^{R_1} m_i (r_\perp^i)^2 + \dots \\
    &= I_1 + I_2 + I_3 + \dots \\
    &= \sum_i I_i.
\end{align}

Isso significa que podemos dividir o corpo em regiões cujo momento de inércia sabemos calcular, e, após isso, determinamos o momento de inércia total simplesmente somando os resultados. É importante destacar que a posição das regiões em relação ao eixo de rotação não muda, isto é, apesar de separarmos o corpo de diversas partes, a posição que elas ocupam no corpo altera o valor obtido para o momento de inércia. No caso da Figura~\ref{Fig:AditividadeMomInercia}, por exemplo, podemos calcular separadamente o momento de inércia do disco e da haste, mas devemos levar em conta que a haste gira em torno de um eixo que está distante da extremidade.
\end{description}

%%%%%%%%%%%%%%%%%%%%%%%%%%%%%%%%%%%%%%%%%%%%%%%%%%%%%%%%%%%%%%%%%%%%%
\subsection{Momento de inércia de uma distribuição contínua de massa}
%%%%%%%%%%%%%%%%%%%%%%%%%%%%%%%%%%%%%%%%%%%%%%%%%%%%%%%%%%%%%%%%%%%%%

\begin{marginfigure}
\centering
\begin{tikzpicture}[scale = 1.2,
    interface/.style={
        % superfície
        postaction={draw,decorate,decoration={border,angle=-45,
                    amplitude=0.2cm,segment length=2mm}}},
    ]
    
% topo
\draw (0,0) ellipse (1.3 and 0.55);
\draw (-1.3, 0) -- (-1.3,-0.25);
\draw (1.3, 0) -- (1.3, -0.25);
\draw (-1.3,-0.25) arc (180:360:1.3 and 0.55);
\draw[dotted] (-1.3,-0.25) arc (180:360:1.3 and -0.55);

%\draw[<->] (0,0) -- node[above]{$R$} (1.25,0);

% laterais
\draw[densely dotted] (-1.3,-0.25) arc[start angle = 90, end angle = -14, radius = 0.25];
\draw[densely dashed] (-1.3,-0.75) arc[start angle = -90, end angle = -14, radius = 0.25];

\draw[densely dotted] (1.3,-0.25) arc[start angle = 90, end angle = 194, radius = 0.25];
\draw[densely dashed] (1.3,-0.75) arc[start angle = -90, end angle = -166, radius = 0.25];

% fundo
\draw (-1.3,-0.75) arc (180:153:1.3 and 0.55);
\draw (1.3,-0.75) arc (0:27:1.3 and 0.55);
\draw (-1.3,-0.75) arc (180:360:1.3 and 0.55);
\draw[dotted] (-1.3,-0.75) arc (180:360:1.3 and -0.55);
\draw (-1.3,-0.75) -- (-1.3,-1);
\draw (1.3,-0.75) -- (1.3,-1);
\draw (-1.3,-1) arc (180:360:1.3 and 0.55);
\draw[dotted] (-1.3,-1) arc (180:360:1.3 and -0.55);
\draw[fill] (0,-1) circle (0.6pt);

% Eixo z
\draw[fill] (0,0) circle (0.6pt);
\draw[dashdotted,-Stealth] (0,0) -- (0,1.5) node[below left]{$z$};
\draw[densely dotted] (0,0) -- (0,-1.55);
\draw[dashdotted] (0,-1.55) -- (0,-2);

\draw[fill] (0,-0.5) circle (1pt);

\draw[-Stealth] (0,1.05) [partial ellipse=-225:60:0.3cm and 0.125cm];

% dm
\begin{scope}[scale = 0.15, rotate around y = -7, shift = {(-5, -1)}]

    \draw[fill, lightgray, draw = black] (1,1,1) -- (1,1,0) -- (0,1,0) -- (0,1,1) -- cycle;
    \draw[fill, lightgray, draw = black] (1,1,1) -- (1,1,0) -- (1,0,0) -- (1,0,1) -- cycle;
    \draw[fill, lightgray, draw = black] (1,1,1) -- (0,1,1) -- (0,0,1) -- (1,0,1) -- cycle;

\end{scope}

\draw[-Stealth] (0,-0.18) coordinate (brcmc) -- node[above]{$r_\perp$} (-0.68,-0.09) coordinate (cmc);
\draw[fill] (brcmc) circle (0.3pt);
\draw[fill] (cmc) circle (0.6pt);
\coordinate (O) at (0,0);

\pic [draw, "$\cdot$", angle radius = 1.7mm, angle eccentricity = 0.5] {angle = O--brcmc--cmc};

% Visão lateral

\draw (-1.3,-3) rectangle (1.3,-3.25);
\draw (-1.3,-3.25) arc[start angle = 90, end angle = -90, radius = 0.25];
\draw (1.3,-3.25) arc[start angle = 90, end angle = 270, radius = 0.25];
\draw (-1.3,-4) rectangle (1.3, -3.75);

\draw[dashdotted, -Stealth] (0,-3) -- +(0,0.75) node[below left]{$z$};
\draw[dotted] (0,-3) -- +(0,-1);
\draw[dashdotted] (0,-4) -- +(0,-0.75);

\draw[fill] (0,-3.5) circle (1pt);

\end{tikzpicture}
\caption{Roldana formada por um cilindro com uma ``calha'' semi-circular.}
\end{marginfigure}

A determinação do momento de inércia através da Equação \eqref{Eq:MomentoInerciaConjPart} para um corpo extenso não é possível devido ao grande número de partículas que compõe um corpo rígido. Para contornar esse problema, vamos dividimos o corpo em uma série de regiões cúbicas e vamos considerar que toda a massa está localizada no centro de massa do cubo. Assim, temos que
\begin{equation}
    I = \sum_{i = 1}^N M_{R_i} (r_\perp^i)^2.
\end{equation}
%
Como no caso do centro de massa, a divisão em cubos pode não ser muito precisa para um corpo qualquer cujos limites não se alinhem com os cubos, porém sempre podemos melhorar a precisão ao aumentar o número de cubos, diminuindo assim o volume de cada um deles. Ao tomarmos o limite de infinitos cubos, recaímos na definição de uma integral em duas ou três dimensões:\footnote[][3cm]{O caso bidimensional é um caso especial do tridimensional, e se aplica aos casos onde temos objetos em que uma das dimensões é constante para todo o objeto --~ou seja, uma figura plana, como uma placa metálica, por exemplo~--.}
\begin{align}
    I &= \lim_{N \to \infty} \sum_{i = 1}^N M_{R_i} (r_\perp^i)^2 \\
    &= \int r_\perp^2 dm. \label{Eq:MomInerciaDistContinua}
\end{align}
%
A interpretação da expressão acima é mais simples se escrevermos $dm$ em termos de densidades, como fizemos na determinação da posição do centro de massa:
\begin{align}
    dm &= \lambda(x) \; dx \\
    dm &= \sigma(\vec{r}) \;dA \\
    dm &= \rho(\vec{r}) \; dV. \label{Eq:DiferencialDeMassaVolume}
\end{align}

Mais uma vez, o uso de expressões integrais como a Equação~\eqref{Eq:MomInerciaDistContinua} acima depende de podermos descrever a forma de um corpo matematicamente, o que só pode ser feito facilmente para formas simples. Nesse caso, as expressões resultantes para o momento de inércia dos corpos serão diferentes para cada tipo de corpo, porém são razoavelmente simples.

%%%%%%%%%%%%%%%%%%%%%%%%%%%%%%%%%%%%%%%%%%%%%%%%%%%
\paragraph{Momento de inércia de um disco/cilindro}
%%%%%%%%%%%%%%%%%%%%%%%%%%%%%%%%%%%%%%%%%%%%%%%%%%%

Para determinarmos o momento de inércia de um disco/cilindro de massa $M$, raio $R$ e altura $L$, em torno do eixo que passa pelo centro de massa, perpendicularmente à face plana, basta empregarmos a Equação~\eqref{Eq:MomInerciaDistContinua} e escrever $dm$ de acordo com a Equação~\eqref{Eq:DiferencialDeMassaVolume}:
%
\begin{marginfigure}
\centering
\begin{tikzpicture}[>=Stealth]
    
% topo
\draw (0,0) ellipse (1.5 and 0.7);

\draw[<->] (0,0) -- node[above]{$R$} (-20:1.25);

% laterais
\draw (-1.5,0) -- (-1.5,-1.5);
\draw[|<->|] (-1.75,0) -- node[left]{$L$} (-1.75,-1.5);
\draw (1.5,-1.5) -- (1.5,0);  

% fundo
\draw[dotted] (-1.5,-1.5) arc (180:360:1.5 and -0.7);
\draw (-1.5,-1.5) arc (180:360:1.5 and 0.7);

% Eixo z
\draw[dashdotted,->] (0,-0.5) -- (0,1.5) node[below left]{$z$};
\draw[dotted] (0,-0.5) -- (0,-2.2);
\draw[dashdotted] (0,-2.2) -- (0,-2.9);

\draw[fill] (0,-0.8) circle (1pt) node[below right] {CM};
\draw[fill] (0,0) circle (0.5pt);
\draw[fill] (0, -1.5) circle (0.5pt);

\draw[->] (0,1.05) [partial ellipse=-225:60:0.3cm and 0.125cm];

\end{tikzpicture}
\caption{Cilíndro de massa $M$, raio $R$, e altura $L$. \label{Fig:MomInerciaDiscoCilindro}}
\end{marginfigure}
%
\begin{equation}
    I = \iiint_V r_\perp^2 \, \rho(\vec{r}) \, dV.
\end{equation}
%
Se a densidade do corpo é homogênea, podemos escrever $\rho(\vec{r}) \equiv \rho$. O volume $V$ sobre o qual a integração é realizada é o próprio volume determinado pelo cilindro. Claramente uma integração em coordenadas cilíndricas é mais conveniente, logo, reescrevemos $dV$ como $dV = r \, dr \, d\theta \, dz$ e utilizamos os limites $r = (0, R)$, $\theta = (0, 2\pi)$ e $z = (0,L)$. Note ainda que a coordenada $r$ das coordenadas cilíndricas é a própria distância ao eixo de rotação e podemos omitir o índice $\perp$. Assim:
\begin{align}
    I &= \int_0^L \int_0^{2\pi} \int_0^R r^3 \, \rho \, dr \, d\theta \, dz \\
    &= 2\pi \cdot L \cdot \rho \cdot \frac{R^4}{4} \\
    &= [\pi R^2 \cdot L \cdot \rho] \cdot \frac{R^2}{2}.
\end{align}
%
Note que o termo entre colchetes nada mais é que o volume do cilindro multiplicado pela sua densidade, ou seja, corresponde à sua massa. Consequentemente, podemos escrever:
\begin{equation}
    I = \frac{MR^2}{2}.\mathnote{Momento de inércia de um disco/cilíndro em torno de um eixo perpendicular à face plana e que passa pelo centro de massa.}
\end{equation}

%%%%%%%%%%%%%%%%%%%%%%%%%%%%%%%%%%%%%%%%%%%%%%%%%%%%
\paragraph{Momento de inércia de um tubo cilíndrico}
%%%%%%%%%%%%%%%%%%%%%%%%%%%%%%%%%%%%%%%%%%%%%%%%%%%%

\begin{marginfigure}[8cm]
\centering
\begin{tikzpicture}[>=Stealth]
    
% topo
\draw (0,0) ellipse (1.25 and 0.5);
\draw (0,0) ellipse (1.5 and 0.7);

\draw[<->] (0,0) -- node[above]{$R_i$} (-1.25,0);
\draw[<->] (0,0) -- node[above]{$R_e$} (-20:1.25);

% laterais
\draw[dotted] (-1.25,0) -- (-1.25,-1.5);
\draw[dotted] (1.25,-1.5) -- (1.25,0);
\draw (-1.5,0) -- (-1.5,-1.5);
\draw[|<->|] (-1.75,0) -- node[left]{$L$} (-1.75,-1.5);
\draw (1.5,-1.5) -- (1.5,0);  

% fundo
\draw[dotted] (-1.25,-1.5) arc (180:360:1.25 and 0.5);
\draw[dotted] (-1.25,-1.5) arc (180:360:1.25 and -0.5);
\draw[dotted] (-1.5,-1.5) arc (180:360:1.5 and -0.7);
\draw (-1.5,-1.5) arc (180:360:1.5 and 0.7);

% Eixo z
\draw[dashdotted,->] (0,-0.5) -- (0,1.5) node[below left]{$z$};
\draw[dotted] (0,-0.5) -- (0,-2.2);
\draw[dashdotted] (0,-2.2) -- (0,-2.9);

\draw[fill] (0,-0.8) circle (1pt) node[below right] {CM};
\draw[fill] (0,0) circle (0.5pt);
\draw[fill] (0, -1.5) circle (0.5pt);

\draw[->] (0,1.05) [partial ellipse=-225:60:0.3cm and 0.125cm];

\end{tikzpicture}
\caption{Tubo cilíndrico de massa $M$, altura $L$ e raios interno e externo $R_i$ e $R_e$, respectivamente. \label{Fig:MomInerciaTuboCilindrico}}
\end{marginfigure}

Para um tubo cilíndrico de massa $M$, altura $L$ e raios interno e externo $R_i$ e $R_e$, e que gira em torno do eixo que passa pelo centro de massa, coincidindo com o eixo do tubo, podemos seguir o mesmo procedimento que o caso de um cilíndro. Será necessário, no entanto, substituir os limites de integração em $r$ por $r = (R_i, R_e)$:
\begin{align}
    I &= \iiint_V r_\perp^2 \, \rho(\vec{r}) \, dV \\
    &= \int_0^L \int_0^{2\pi} \int_{R_i}^{R_e} r^3 \, \rho \, dr \, d\theta \, dz \\
    &= 2\pi \cdot L \cdot \rho \cdot \frac{R_e^4 - R_i^4}{4}.
\end{align}
%
Notando que 
\begin{equation}
    R_e^4 - R_i^4 = (R_e^2 - R_i^2)\cdot(R_e^2 + R_i^2),
\end{equation}
%
podemos reecrever a expressão para o momento de inércia como
\begin{equation}
    I = [\pi \cdot (R_e^2 - R_i^2) \cdot L \cdot \rho] \cdot \frac{R_e^2 + R_i^2}{2}.
\end{equation}
%
Como no caso do cilíndro, verificamos que o termo entre colchetes corresponde ao produto do volume do tubo e de sua densidade, equivalendo à massa do corpo. Consequentemente, podemos escrever para o momento de inércia
\begin{equation}
    I = M \; \frac{R_e^2 + R_i^2}{2}. \mathnote{Momento de inércia de um tubo cilíndrico em torno do eixo central do tubo.}
\end{equation}

%%%%%%%%%%%%%%%%%%%%%%%%%
\section{Roda de Maxwell}
%%%%%%%%%%%%%%%%%%%%%%%%%

% Pra essa figura ficar boa, precisaria fazer
% os traços sobre uma foto, assim a perspectiva
% ficaria correta
\begin{marginfigure}
\centering
\begin{tikzpicture}[>=Stealth, rotate = -95,
     interface/.style={
        % superfície
        postaction={draw,decorate,decoration={border,angle=-45,
                    amplitude=0.2cm,segment length=2mm}}},
    ]

%%% Teto

    \draw[interface] (-2.6,2.25) -- (-2.45,-2.5);
    
%%% Disco

\draw (0,0) ellipse (1.25 and 0.75);
\draw (0,0) ellipse (1 and 0.5);
\begin{scope}
    \clip (0,0) ellipse (1 and 0.5);
    \draw (-1,-0.1) arc (180:360:1 and -0.5);
\end{scope}

\draw (-1.25,0) -- (-1.25,-0.4);
\draw (1.25,-0.4) -- (1.25,0);  

\draw (-1.25,-0.4) arc (180:360:1.25 and 0.75);
%\draw[dotted] (-1.25,-0.4) arc (180:360:1.25 and -0.75);

%%% Semieixo direito
\fill[white] (0.125,-0.075) rectangle (-0.125,2);
\draw (0.125,-0.075) -- +(0,2.075);
\draw (-0.125,-0.075) -- +(0,2.075);
\draw (0,2) ellipse (0.125 and 0.075);
\draw (-0.125,-0.075) arc[start angle = 180, end angle = 360, x radius = 0.125, y radius = 0.075];

% Corda
\draw (-0.125,1.75) arc[start angle = 180, end angle = 360, x radius = 0.125, y radius = 0.075];
\draw (-0.125,1.70) arc[start angle = 180, end angle = 360, x radius = 0.125, y radius = 0.075];
\draw (-0.125,1.65) arc[start angle = 180, end angle = 360, x radius = 0.125, y radius = 0.075];
\draw (-0.125,1.60) arc[start angle = 180, end angle = 360, x radius = 0.125, y radius = 0.075];
\draw (-0.125,1.55) arc[start angle = 180, end angle = 360, x radius = 0.125, y radius = 0.075];
\draw (-0.125,1.50) arc[start angle = 180, end angle = 360, x radius = 0.125, y radius = 0.075];
\draw (-0.125,1.45) arc[start angle = 180, end angle = 360, x radius = 0.125, y radius = 0.075];
\draw (-0.125,1.40) arc[start angle = 180, end angle = 360, x radius = 0.125, y radius = 0.075];
\draw (-0.125,1.8) -- (-2.58,1.55);


%%% Semieixo esquerdo
%\draw[dotted] (0,-0.4) ellipse (0.125 and 0.075);
%\draw[dotted] (0.125, -0.4) -- (0.125,-1.15);
%\draw[dotted] (-0.125, -0.4) -- (-0.125,-1.15);
\draw (0.125, -1.15) -- (0.125,-2.13);
\draw (-0.125, -1.15) -- (-0.125,-2.13);
\draw (-0.125,-2.13) arc[start angle = 180, end angle = 360, x radius = 0.125, y radius = 0.075];
% Corda
\draw (-0.125,-1.85) arc[start angle = 180, end angle = 360, x radius = 0.125, y radius = 0.075];
\draw (-0.125,-1.80) arc[start angle = 180, end angle = 360, x radius = 0.125, y radius = 0.075];
\draw (-0.125,-1.75) arc[start angle = 180, end angle = 360, x radius = 0.125, y radius = 0.075];
\draw (-0.125,-1.70) arc[start angle = 180, end angle = 360, x radius = 0.125, y radius = 0.075];
\draw (-0.125,-1.65) arc[start angle = 180, end angle = 360, x radius = 0.125, y radius = 0.075];
\draw (-0.125,-1.60) arc[start angle = 180, end angle = 360, x radius = 0.125, y radius = 0.075];
\draw (-0.125,-1.55) arc[start angle = 180, end angle = 360, x radius = 0.125, y radius = 0.075];
\draw (-0.125,-1.50) arc[start angle = 180, end angle = 360, x radius = 0.125, y radius = 0.075];
\draw (-0.125,-1.9) -- (-2.45,-2.1);

\end{tikzpicture}
\caption{Roda de Maxwell.\label{Fig:RodaDeMaxwell}}
\end{marginfigure}

\begin{marginfigure}[1.5cm]
\centering
\begin{tikzpicture}[>=Stealth,
     interface/.style={
        % superfície
        postaction={draw,decorate,decoration={border,angle=-45,
                    amplitude=0.2cm,segment length=2mm}}},
    ]

    \draw (0,0) circle (2cm);
    \draw (0,0) circle (1.5cm);
    \draw (0,0) circle (0.25cm);
    
    \draw (-0.25,0) -- (-0.25,3);
    \draw[interface] (1,3) -- (-1,3);
    
    \draw[dotted] (0,0.25) -- (1,0.25);
    \draw[dotted] (0,-0.25) -- (1,-0.25);
    \draw[|<->|] (1,0.25) -- node[right]{$2r$} (1,-0.25);
    
    \draw[dotted] (0,2) -- (2.5,2);
    \draw[dotted] (0,-2) -- (2.5,-2);
    \draw[|<->|] (2.5,2) -- node[right]{$2R$} (2.5,-2);

\end{tikzpicture}
\caption{Roda de Maxwell, visão lateral.\label{Fig:RodaDeMaxwellVisaoLateral}}
\end{marginfigure}

Um dos sistemas mais simples que exibe rotação e translação é o que denominamos como ``Roda de Maxwell'', e que consiste em um disco ligado a um eixo suspenso por cordas. As cordas são enroladas no eixo e o sistema é liberado a partir do repouso, deixando o disco descer e desenrolar a corda, de maneira similar a um ioiô. Apesar de o disco exibir uma rotação e uma translação, esses movimentos não são independentes, uma vez que a distância percorrida pelo centro de massa está ligada ao comprimento de corda que é desenrolado do eixo. A relação entre o deslocamento do centro de massa e o deslocamento angular, que determina a quantidade $\ell$ de corda desenrolada, é
\begin{equation}
    \Delta y_{\text{CM}} \equiv \ell = r \Delta \theta.
\end{equation}
%
A partir dessa relação, também temos
\begin{align}
    v_{\text{CM}} &= r \omega \\
    a_{\text{CM}} &= r \alpha.
\end{align}
%
Note que as relações acima podem ou não ter um sinal negativo, dependendo da escolha do sistema de referência. Se adotarmos um eixo vertical apontando para cima, as expressões acima estão adequadas. %\textbf{Se tirar a máquina de atwood lá do começo, vou precisar explicar a questão dos sinais aqui.}

Aplicando a Segunda Lei de Newton para translação ao centro de massa, obtemos
\begin{description}
\item[Eixo $y$:]
\begin{align}
    F_R^y &= m_t a_{\text{CM}, y} \\
    T - P &= m_t a,
\end{align}
\end{description}
%
onde adotamos $a \equiv a_{\text{CM}, y}$ e $m_t$ representa a massa total do sistema (soma da massa do disco e dos dois semieixos).

Aplicando a Segunda Lei de Newton para a rotação, temos
\begin{align}
    \tau &= I\alpha \\
    -Tr &= I\alpha.
\end{align}
%
O momento de inércia total é dado pela soma do momento de inércia do disco central e dos dois semieixos:
\begin{equation}
    I = \frac{1}{2} MR^2 + 2\frac{1}{2} m r^2.
\end{equation}

Através dos resultados obtidos acima, podemos montar um sistema de equações
\begin{equation}
\begin{system}
    T - P &= m_t a \\
    -Tr &= I\alpha \\
    a &= r \alpha,
\end{system}
\end{equation}
%
cuja solução é
\begin{align}
    a &= \frac{m_t g}{m_t + I / r^2} \label{Eq:AcelRodaDeMaxwellGenerica}\\
    &= \frac{m_tg}{m_t + M(R/r)^2/2 + m} \\
    &= \frac{M + 2m}{M + 2m + M(R/r)^2/2 + m} \cdot g \\
    &= \frac{M + 2m}{3m + M \cdot [1 + (R/r)^2/2]} \cdot g \\
    &= \frac{2 + M/m}{3 + (M/m) \cdot [1 + (R/r)^2/2]} \cdot g
\end{align}

\begin{marginfigure}[1.5cm]
\centering
\begin{tikzpicture}[>=Stealth, scale = 0.95,
     interface/.style={
        % superfície
        postaction={draw,decorate,decoration={border,angle=-45,
                    amplitude=0.2cm,segment length=2mm}}},
    ]

    \def\hL{0.5};
    \def\l{2};
    \def\R{2};
    \def\r{0.15};
    \def\d{0.3}
    \draw (-\hL,-\R) rectangle (\hL,\R);
    \draw (-\hL,\r) rectangle +(-\l,-\d);
    \draw (\hL,-\r) rectangle +(\l,\d);
    
    \draw[|<->|] (-\hL, 2.2) -- node[above]{$L$} (\hL, 2.2);
    \draw[|<->] (-\hL, 0.35) +(-\l,0) -- node[above]{$\ell$} (-\hL,0.35);
    \draw[|<->] (\hL,0.35) +(\l,0) -- node[above]{$\ell$} (\hL,0.35);
    
\end{tikzpicture}
\caption{Roda de Maxwell, visão frontal.\label{Fig:RodaDeMaxwellVisaoFrontal}}
\end{marginfigure}

Se assumirmos que os eixos laterais e o disco central são feitos do mesmo material, podemos escrever as massas como
\begin{align}
    M &= \rho V_d \\
    &= \rho \pi R^2 L \\
    m &= \rho V_e \\
    &= \rho \pi r^2 \ell,
\end{align}
%
onde $L$ e $\ell$ representam a espessura do disco e o comprimento dos semieixos, respectivamente, e $\rho$ representa a densidade do material. Calculando a razão $M/m$, obtemos
\begin{align}
    \frac{M}{m} &= \frac{\rho \pi R^2 L}{\rho \pi r^2 \ell} \\
    &= \left(\frac{R}{r}\right)^2\frac{L}{\ell}.
\end{align}
%
Se substituirmos os resultados acima na expressão para a aceleração, podemos escrever
\begin{equation}
    a = \frac{2 + \lambda^2\gamma}{3 + (1 + \lambda^2/2) \lambda^2 \gamma} \cdot g,
\end{equation}
%
onde definimos
\begin{align}
    \lambda &\equiv R/r \\
    \gamma &\equiv L/\ell.
\end{align}

Um caso particular da expressão obtida acima para a aceleração é aquele em que $\lambda = R/r \to 1$, ou seja, temos um simples cilindro. Nesse caso temos que
\begin{align}
    a &\to \frac{2 + \gamma}{3 + (1 + 1/2)\gamma} \cdot g \\
    &\to \frac{2 + \gamma}{3 + 3\gamma/2} \cdot g \\
    &\to \frac{2}{3} \cdot \frac{ 1 + \gamma/2}{1 + \gamma/2} \cdot g \\
    &\to \frac{2}{3} \cdot g.
\end{align}
%
Note que esse é o mesmo resultado que se obtém ao usar $I = MR^2 / 2$, com $M = m_t$ e $R = r$ na Equação~\eqref{Eq:AcelRodaDeMaxwellGenerica}.

%%%%%%%%%%%%%%%%%%%%%%%%%%%%%%%%%%%%%%%%%%%%%%%%%%%%%%%%%%%%%%%%%%%%%%%%%%%%%%%
\section{Experimento}
%%%%%%%%%%%%%%%%%%%%%%%%%%%%%%%%%%%%%%%%%%%%%%%%%%%%%%%%%%%%%%%%%%%%%%%%%%%%%%%

%%%%%%%%%%%%%%%%%%%%%%
\subsection{Objetivos}
%%%%%%%%%%%%%%%%%%%%%%

\begin{itemize}
	\item Resultados concretos que devemos conseguir;
	\item Observar a relação de tal coisa com outra coisa;
	\item Calcular a constante $x$.
\end{itemize}

%%%%%%%%%%%%%%%%%%%%%%%%%%%%%%%%%%%%%%%%%%%%%%%%%%%%%%%%%%%%%%%%%%%%%%%%%%%%%%%
\section{Material Necessário}
%%%%%%%%%%%%%%%%%%%%%%%%%%%%%%%%%%%%%%%%%%%%%%%%%%%%%%%%%%%%%%%%%%%%%%%%%%%%%%%

\begin{itemize}
	\item Item 1;
	\item Item 2.
\end{itemize}

%%%%%%%%%%%%%%%%%%%%%%%%%%%%%%%%%%%%%%%%%%%%%%%%%%%%%%%%%%%%%%%%%%%%%%%%%%%%%%%
\section{Procedimento Experimental}
%%%%%%%%%%%%%%%%%%%%%%%%%%%%%%%%%%%%%%%%%%%%%%%%%%%%%%%%%%%%%%%%%%%%%%%%%%%%%%%

%%%%%%%%%%%%%%%%%%%%%
%\subsection{Parte A} % Se necessário
%%%%%%%%%%%%%%%%%%%%%
\begin{enumerate}
	\item Passo 1;
	\item Passo 2;
	\item Passo 3.
\end{enumerate}

%%%%%%%%%%%%%%%%%%%%%%%%%%%%%%%%%%%%%%%%%%%%%%%%%%%%%%%%%%%%%%%%%%%%%%%%%%%%%%%
%%%%%%%%%%%%%%%%%%%%%%%%%%%%%%%%%%%%%%%%%%%%%%%%%%%%%%%%%%%%%%%%%%%%%%%%%%%%%%%
%%%%%%%%%%%%%%%%%%%%%%%%%%%%%%%%%%%%%%%%%%%%%%%%%%%%%%%%%%%%%%%%%%%%%%%%%%%%%%%
%%%%%%%%%%%%%%%%%%%%%%%%%%%%%%%%%%%%%%%%%%%%%%%%%%%%%%%%%%%%%%%%%%%%%%%%%%%%%%%
\cleardoublepage

\noindent{}{\huge\textit{Nome do Experimento}}

\vspace{15mm}

\begin{fullwidth}
\noindent{}\makebox[0.6\linewidth]{Turma:\enspace\hrulefill}\makebox[0.4\textwidth]{  Data:\enspace\hrulefill}
\vspace{5mm}

\noindent{}\makebox[0.6\linewidth]{Aluno(a):\enspace\hrulefill}\makebox[0.4\textwidth]{  Matrícula:\enspace\hrulefill}

\noindent{}\makebox[0.6\linewidth]{Aluno(a):\enspace\hrulefill}\makebox[0.4\textwidth]{  Matrícula:\enspace\hrulefill}

\noindent{}\makebox[0.6\linewidth]{Aluno(a):\enspace\hrulefill}\makebox[0.4\textwidth]{  Matrícula:\enspace\hrulefill}

\noindent{}\makebox[0.6\linewidth]{Aluno(a):\enspace\hrulefill}\makebox[0.4\textwidth]{  Matrícula:\enspace\hrulefill}

\noindent{}\makebox[0.6\linewidth]{Aluno(a):\enspace\hrulefill}\makebox[0.4\textwidth]{  Matrícula:\enspace\hrulefill}
\end{fullwidth}

\vspace{5mm}

%%%%%%%%%%%%%%%%%%%%%%%%%%%%%%%%%%%%%%%%%%%%%%%%%%%%%%%%%%%%%%%%%%%%%%%%%%%%%%%
\section{Questionário}
%%%%%%%%%%%%%%%%%%%%%%%%%%%%%%%%%%%%%%%%%%%%%%%%%%%%%%%%%%%%%%%%%%%%%%%%%%%%%%%

\begin{question}[type={exam}]{1}
Apresente os resultados de maneira clara e organizada. Mostre os cálculos requisitados de maneira clara e sucinta, evidenciando o raciocínio desenvolvido.
\end{question}

\begin{question}[type={exam}]{1}
Preencha as colunas de dados experimentais das tabelas com o número adequado de algarismos significativos e unidades.
\end{question}

\begin{question}[type={exam}]{2}
Lorem ipsum dolor sit amet, consectetuer adi-
piscing elit. Ut purus elit, vestibulum ut, placerat ac, adipiscing vitae,
felis. Curabitur dictum gravida mauris. Nam arcu libero, nonummy
eget, consectetuer id, vulputate a, magna. Donec vehicula augue
eu neque. Pellentesque habitant morbi tristique senectus et netus
et malesuada fames ac turpis egestas. Mauris ut leo. Cras viverra
metus rhoncus sem. Nulla et lectus vestibulum urna fringilla ultrices.
\end{question}

\begin{question}[type={exam}]{2}
Lorem ipsum dolor sit amet, consectetuer adi-
piscing elit. Ut purus elit, vestibulum ut, placerat ac, adipiscing vitae,
felis. Curabitur dictum gravida mauris. Nam arcu libero, nonummy
eget, consectetuer id, vulputate a, magna. Donec vehicula augue
eu neque. Pellentesque habitant morbi tristique senectus et netus
et malesuada fames ac turpis egestas. Mauris ut leo. Cras viverra
metus rhoncus sem. Nulla et lectus vestibulum urna fringilla ultrices.
\end{question}

\begin{question}[type={exam}]{2}
Lorem ipsum dolor sit amet, consectetuer adi-
piscing elit. Ut purus elit, vestibulum ut, placerat ac, adipiscing vitae,
felis. Curabitur dictum gravida mauris. Nam arcu libero, nonummy
eget, consectetuer id, vulputate a, magna. Donec vehicula augue
eu neque. Pellentesque habitant morbi tristique senectus et netus
et malesuada fames ac turpis egestas. Mauris ut leo. Cras viverra
metus rhoncus sem. Nulla et lectus vestibulum urna fringilla ultrices.
\end{question}

\begin{question}[type={exam}]{2}
Lorem ipsum dolor sit amet, consectetuer adi-
piscing elit. Ut purus elit, vestibulum ut, placerat ac, adipiscing vitae,
felis. Curabitur dictum gravida mauris. Nam arcu libero, nonummy
eget, consectetuer id, vulputate a, magna. Donec vehicula augue
eu neque. Pellentesque habitant morbi tristique senectus et netus
et malesuada fames ac turpis egestas. Mauris ut leo. Cras viverra
metus rhoncus sem. Nulla et lectus vestibulum urna fringilla ultrices.
\end{question}

\begin{question}[type={exam}]{2}
Lorem ipsum dolor sit amet, consectetuer adi-
piscing elit. Ut purus elit, vestibulum ut, placerat ac, adipiscing vitae,
felis. Curabitur dictum gravida mauris. Nam arcu libero, nonummy
eget, consectetuer id, vulputate a, magna. Donec vehicula augue
eu neque. Pellentesque habitant morbi tristique senectus et netus
et malesuada fames ac turpis egestas. Mauris ut leo. Cras viverra
metus rhoncus sem. Nulla et lectus vestibulum urna fringilla ultrices.
\end{question}
\vfill
%%%%%%%%%%%%%%%%%%%%%%%%%%%%%%%%%%%%%%%%%%%%%%%%%%%%%%%%%%%%%%%%%%%%%%%%%%%%%%%
\pagebreak
\section{Tabelas}
%%%%%%%%%%%%%%%%%%%%%%%%%%%%%%%%%%%%%%%%%%%%%%%%%%%%%%%%%%%%%%%%%%%%%%%%%%%%%%%

