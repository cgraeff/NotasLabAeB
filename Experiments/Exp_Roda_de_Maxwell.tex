%%%%%%%%%%%%%%%%%%%%%%%%%%%%%%%%%%%%%%%%%%%%%%%%%%%%%%%%%%%%%%%%%%%%%%%%%%%%%%%
\chapter{Roda de Maxwell} % Sem "Experiência 01" ou qualquer outro número
\label{Chap:RodaDeMaxwell}        % para poder trocar a ordem com facilidade
%%%%%%%%%%%%%%%%%%%%%%%%%%%%%%%%%%%%%%%%%%%%%%%%%%%%%%%%%%%%%%%%%%%%%%%%%%%%%%%

\tikzset{
    partial ellipse/.style args={#1:#2:#3}{
        insert path={+ (#1:#3) arc (#1:#2:#3)}
    }
}

\begin{fullwidth}\it
	Vamos realizar um experimento que envolve um movimento combinado de rotação e de translação. Verificaremos as características do movimento de rotação, buscando determinar o momento de inércia de um corpo rígido experimentalmente, verificando se o seu valor corresponde ao determinado teoricamente. Revisaremos a Segunda Lei de Newton para a rotação e as técnicas para a obtenção do momento de inércia. \textbf{Quais conceitos/técnicas de análise de dados utilizaremos?}
\end{fullwidth}

%%%%%%%%%%%%%%%%%%%%%%%%%%%%%%%%%%%%%%%%%%%%%%%%%%%%%%%%%%%%%%%%%%%%%%%%%%%%%%%
\section{Física do experimento}
%%%%%%%%%%%%%%%%%%%%%%%%%%%%%%%%%%%%%%%%%%%%%%%%%%%%%%%%%%%%%%%%%%%%%%%%%%%%%%%

\begin{marginfigure}[5cm]
\centering
\begin{tikzpicture}[>=Stealth, scale = 1.4,
     interface/.style={
        % superfície
        postaction={draw,decorate,decoration={border,angle=-45,
                    amplitude=0.2cm,segment length=2mm}}},
    ]

%%% Figura superior

\draw (0,0) ellipse (1.25 and 0.5);

\draw (-1.25,0) -- (-1.25,-0.4);
\draw (1.25,-0.4) -- (1.25,0);  

\draw (-1.25,-0.4) arc (180:360:1.25 and 0.5);
\draw[densely dotted] (-1.25,-0.4) arc (180:360:1.25 and -0.5);

\draw[dashdotted,->] (0,0) -- (0,1.5) node[below left]{$z$};
\draw[dashdotted] (0,-1.5) -- (0,-0.9);
\draw[dotted] (0,-0.9) -- (0,0);
\draw[fill] (0,0) circle (1pt);

\draw[dashdotted] (15:-2) -- (0,0);

% raio
\draw[|-|] (-0.05,0.15) -- node[above]{$r_\perp^i$} +(15:-0.8);
\draw[fill] (15:-0.8) circle (1pt);

% força
\draw[thick, ->] (15:-0.8) -- +(-65:1) node[below]{$\vec{F}_j^i$};

% projeção radial
\draw[dashed] (15:-0.8) ++(-65:1) -- (15:-1.6); 
\draw[->] (15:-0.8) -- (15:-1.6) node[above]{$F_{j,r}^i$};

% projeção tangencial
\draw[dashed] (15:-0.8) +(-30:1.6) -- +(-30:-0.3);
\draw[dashed] (15:-0.8) ++(-65:1) -- +(18:0.8);
\draw[->] (15:-0.8) -- +(-30:1.35) node[above]{$F_{j,t}^i$};

% phi
\draw (15:-1.2) arc (150:117:1.25 and -0.5);
\node (theta) at (30:-1.1) {$\phi_j^i$};

\end{tikzpicture}
\caption{Sobre cada uma das partículas que compõe um corpo rígido, age uma grande quantidade de forças internas e externas. Na figura tomamos uma partícula $i$ qualquer e analisamos o efeito de uma força $\vec{F}_j^i$ que atua sobre ela, obtendo as componentes nas direções radial e tangencial.\label{Fig:CalculoSegundaLeiRotacoes}}
\end{marginfigure}

Para verificar a relação entre o torque e aceleração, podemos analisar um objeto como o da Figura~\ref{Fig:CalculoSegundaLeiRotacoes}. Vamos supor que sobre cada partícula do corpo sejam exercidas forças internas e externas, sendo que cada força que atua sobre a $i$-ésima partícula é representada por $F_j^i$, onde $j$ é um índice que enumera cada uma das forças que atuam sobre tal partícula. Se analisarmos o torque devido a $j$-ésima força que atua sobre a $i$-ésima partícula, temos
\begin{align}
	\tau_j^i &= r_\perp^i F_j^i \sen\phi_j^i \\
	&= r_\perp^i F_{j,t}^i.
\end{align}
%
Note que
\begin{equation}
	F_{j,t}^{i} = m_i a_{j,t}^{i},
\end{equation}
%
onde $a_{j,t}^i$ é a aceleração tangencial que a $i$-ésima partícula teria se fosse submetida à $j$-ésima força isoladamente.
%
Logo,
\begin{equation}
	\tau_j^i = m_i a_{j,t}^{i} r_\perp^i.
\end{equation}
%
Se somarmos todos os torques que atuam sobre todas as partículas que compõe o corpo, é possível mostrar que
\begin{equation}
	\sum_i \sum_j \tau_j^i = \sum_{i}[m_i r_\perp^i a_{t}^i].
\end{equation}

Podemos utilizar a relação $a_t = \alpha r$ na expressão acima para escrever $a_{t}^i$ como $\alpha r_\perp^i$, o que resulta em
\begin{equation}
	\sum_i \sum_j \tau_j^i = \sum_{i}[m_i (r_\perp^i)^2 \alpha].
\end{equation}
%
No resultado acima, a aceleração angular $\alpha$ é comum a todos os pontos do corpo rígido, por isso podemos escrever
\begin{equation}
	\sum_i \sum_j \tau_j^i =  \left[\sum_{i} m_i (r_\perp^i)^2\right] \alpha.
\end{equation}

Podemos observar que a soma à esquerda da igualdade inclui todos os torques realizados sobre todas as partículas por todas as forças que atuam no sistema, sejam elas internas ou externas. No entanto, como as forças internas estão presentes aos pares, sendo que cada força do par têm o mesmo módulo e direção, porém sentidos contrários, verificamos que as forças de ação e reação geram torques que se equilibram. Assim, podemos escrever a soma dos torques como a \emph{soma dos torques externos}, ou seja, o \emph{torque resultante externo}:
\begin{equation}\label{Eq:ProtoSegundaLeiRotacoes}
    \tau_R^{\rm{ext}} = \left[\sum_{i} m_i (r_\perp^i)^2\right] \alpha.
\end{equation}
% 

Na expressão acima, o termo entre colchetes é denominado como \emph{momento de inércia}:
\begin{equation}\label{Eq:MomentoInerciaConjPart}
    I = \left[\sum_{i = 1}^N m_i (r_\perp^i)^2\right], \mathnote{Momento de inércia para um conjunto de partículas}
\end{equation}
%
e suas unidades são $[I] = \rm{M}\cdot\rm{L}^2$, ou, dentro do SI, $[I] = \rm{kg}\cdot\rm{m}^2$. A expressão acima é análoga à Segunda Lei de Newton para a translação, porém descrevendo o caso da rotação, e a denominamos como \emph{segunda Lei de Newton para a Rotação}:\footnote[][-5mm]{Na verdade, essa relação foi determinada por Leonard Euler.}
\begin{equation}
    \tau_R^{\rm{ext}} = I \alpha. \mathnote{Segunda Lei de Newton para a rotação}
\end{equation}

%%%%%%%%%%%%%%%%%%%%%%%%%%%%%%%%%%%%%%%
\section{Cálculo do momento de inércia}
%%%%%%%%%%%%%%%%%%%%%%%%%%%%%%%%%%%%%%%

Para que possamos aplicar a Segunda Lei de Newton para a rotação, é fundamental que saibamos qual é o momento de inércia do corpo em questão para o eixo em torno do qual a rotação ocorre. Como é possível verificar na Expressão~\ref{Eq:MomentoInerciaConjPart}, o momento de inércia depende da distância de cada uma das partículas em relação ao eixo de rotação. Isso nos indica que é importante determinar uma série de características particulares a cada situação que estivermos analisando.

Dentre as principais características do momento de inércia, temos as seguintes dependências, cuja origem é a própria dependência na distância entre o eixo de rotação e a posição de cada partícula do corpo (veja a Figura~\ref{Fig:MomInerciaPropriedades}):
\begin{itemize}
    \item Dependência na forma do objeto;
    \item Dependência no eixo em que ocorre a rotação;
    \item Dependência na distância ao eixo de rotação;
    \item Aditividade do momento de inércia.
\end{itemize}

\begin{figure*}[tb]\forcerectofloat
    \centering
    \begin{tikzpicture}[>=Stealth, scale = 1.1]
    
    \draw[fill] (0,0) circle (0.6pt);
    \draw[fill] (0,-0.2) circle (1pt) node[above left] {CM};
    \draw (0,0) ellipse (1.25 and 0.5);
    \draw[fill] (0,-0.4) circle (0.6pt);
    
    \draw (-1.25,0) -- (-1.25,-0.4);
    \draw (1.25,-0.4) -- (1.25,0);  
    
    \draw (-1.25,-0.4) arc (180:360:1.25 and 0.5);
    \draw[densely dotted] (-1.25,-0.4) arc (180:360:1.25 and -0.5);
    
    % Eixo z
    \draw[dashdotted,->] (0,0) -- (0,1.5) node[below left]{$z$};
    \draw[dashdotted] (0,-1.5) -- (0,-0.9);
    \draw[loosely dotted] (0,-0.9) -- (0,0);
    
    \draw[->] (0,1.05) [partial ellipse=-225:60:0.3cm and 0.125cm];
    
    % Eixo x
    \draw[dashdotted, ->] (0,-0.2)++(15:-1.05) -- +(15:-0.75) node[below right]{$x$};
    \draw[fill] (0,-0.2)+(15:-1.05) circle (0.6pt);
    \draw[loosely dotted] (0,-0.2)+(15:-1.05) -- (0,-0.2) -- +(15:1.05);
    \draw[fill] (0,-0.2) +(15:1.05) circle (0.6pt);
    
    \draw[dashdotted] (0,-0.2) ++(15:1.25) -- +(15:0.75);
    
    \draw[->, rotate=15] (0,-0.2)++(0:-1.5) [partial ellipse=-130:160:0.105cm and 0.3cm];
    
    % Eixo w
    \draw[dashdotted, <-] (0,-0.2)+(60:-1.38) node[below right]{$w$} -- +(60:-0.78);
    \draw[fill] (0,-0.2)+(60:-0.78) circle (0.6pt);
    \draw[loosely dotted] (0,-0.2)+(60:-0.78) -- +(60:0.78);
    \draw[fill] (0,-0.2)+(60:0.78) circle (0.6pt);
    \draw[dashdotted] (0,-0.2)+(60:0.78) -- +(60:1.18);
    
    \draw[->, rotate=60] (0,-0.2)++(-4:-1.2) [partial ellipse=-150:150:0.15cm and 0.3cm];
    
    \begin{scope}[shift={(7,0)}]
        %%% Figura superior
    
        \draw[fill] (0,0) circle (0.6pt);
        \draw[fill] (0,-0.2) circle (1pt) node[above left] {CM};
        \draw (0,0) ellipse (1.25 and 0.5);
        \draw[fill] (0,-0.4) circle (0.6pt);
        
        \draw (-1.25,0) -- (-1.25,-0.4);
        \draw (1.25,-0.4) -- (1.25,0);  
        
        \draw (-1.25,-0.4) arc (180:360:1.25 and 0.5);
        \draw[densely dotted] (-1.25,-0.4) arc (180:360:1.25 and -0.5);
        
        % Eixo z
        \draw[dashdotted,->] (0,0) -- (0,1.5) node[below left]{$z$};
        \draw[dashdotted] (0,-1.5) -- (0,-0.9);
        \draw[dotted] (0,-0.9) -- (0,0);
        
        \draw[->] (0,1.05) [partial ellipse=-225:60:0.3cm and 0.125cm];
        
        % Eixos z', z''
        
        \draw[dashdotted, ->] (-1.25,-1.5) -- (-1.25,1.5) node[below left]{$z'$};
        \draw[->] (-1.25,1.05) [partial ellipse=-225:60:0.3cm and 0.125cm];
        
        \draw[dashdotted, ->] (-2.5,-1.5) -- (-2.5,1.5) node[below left]{$z''$};
        \draw[->] (-2.5,1.05) [partial ellipse=-225:60:0.3cm and 0.125cm];
    \end{scope}

    \begin{scope}[shift={(12.5,0)}]
        %%% Figura superior
    
        \draw[fill] (0,0) circle (0.6pt);
        \draw[fill] (0,-0.2) circle (1pt) node[right] {CM};
        \draw (0,0) ellipse (1.25 and 0.5);
        \draw[fill] (0,-0.4) circle (0.6pt);
        
        \draw (-1.25,0) -- (-1.25,-0.4);
        \draw (1.25,-0.4) -- (1.25,0);  
        
        %% arco inferior
        \draw (-1.25,-0.4) arc (180:219:1.25 and 0.5);
        \draw (1.25,-0.4) arc (0:-135:1.25 and 0.5);
        \draw[densely dotted] (-1.25,-0.4) arc (180:360:1.25 and -0.5);
        
        % Eixo z
        \draw[dashdotted,->] (0,0) -- (0,1.5) node[below left]{$z$};
        \draw[dashdotted] (0,-1.5) -- (0,-0.9);
        \draw[dotted] (0,-0.9) -- (0,0);
        
        \draw[->] (0,1.05) [partial ellipse=-225:60:0.3cm and 0.125cm];
        
        % haste
        \draw (0,-0.2)+(30:-2) ellipse (0.45mm and 0.5mm);
        \draw (0,-0.25)+(30:-0.82) arc[start angle = -90, end angle = 90, radius = 0.5mm];
        \draw (0,-0.146)+(30:-2) -- +(30:-0.82);
        \draw (0,-0.254)+(30:-2) -- +(30:-0.82);
    \end{scope}

    \end{tikzpicture}
    \caption{A orientação do eixo de rotação em relação ao objeto e a distância entre o objeto e o eixo de rotação são aspectos importantes para a determinação do momento de inércia. Felizmente, podemos utilizar a aditividade do momento de inércia para dividir o corpo em partes cujos momentos de inércia são tabelados.\label{Fig:MomInerciaPropriedades}}
\end{figure*}

%%%%%%%%%%%%%%%%%%%%%%%%%%%%%%%%%%%%%%%%%%%%%%%%%%%%%%%%%%%%%%%%%%%%%
\subsection{Momento de inércia de uma distribuição contínua de massa}
%%%%%%%%%%%%%%%%%%%%%%%%%%%%%%%%%%%%%%%%%%%%%%%%%%%%%%%%%%%%%%%%%%%%%

A determinação do momento de inércia através da Equação \eqref{Eq:MomentoInerciaConjPart} para um corpo extenso não é possível devido ao grande número de partículas que compõe um corpo rígido. Para contornar esse problema, vamos dividimos o corpo em uma série de regiões cúbicas e vamos considerar que toda a massa está localizada no centro de massa do cubo. Assim, temos que
\begin{equation}
    I = \sum_{i = 1}^N M_{R_i} (r_\perp^i)^2.
\end{equation}
%
Como no caso do centro de massa, a divisão em cubos pode não ser muito precisa para um corpo qualquer cujos limites não se alinhem com os cubos, porém sempre podemos melhorar a precisão ao aumentar o número de cubos, diminuindo assim o volume de cada um deles. Ao tomarmos o limite de infinitos cubos, recaímos na definição de uma integral em duas ou três dimensões:\footnote{O caso bidimensional é um caso especial do tridimensional, e se aplica aos casos onde temos objetos em que uma das dimensões é constante para todo o objeto --~ou seja, uma figura plana, como uma placa metálica, por exemplo~--.}
\begin{align}
    I &= \lim_{N \to \infty} \sum_{i = 1}^N M_{R_i} (r_\perp^i)^2 \\
    &= \int r_\perp^2 dm. \label{Eq:MomInerciaDistContinua}
\end{align}
%
A interpretação da expressão acima é mais simples se escrevermos $dm$ em termos de densidades, como fizemos na determinação da posição do centro de massa:
\begin{align}
    dm &= \lambda(x) \; dx \\
    dm &= \sigma(\vec{r}) \;dA \\
    dm &= \rho(\vec{r}) \; dV. \label{Eq:DiferencialDeMassaVolume}
\end{align}

Mais uma vez, o uso de expressões integrais como a Equação~\eqref{Eq:MomInerciaDistContinua} acima depende de podermos descrever a forma de um corpo matematicamente, o que só pode ser feito facilmente para formas simples. Nesses casos, as expressões resultantes para os momentos de inércia dos corpos serão diferentes para cada tipo de corpo, porém são razoavelmente simples e são frequentemente sumarizados em tabelas.

%%%%%%%%%%%%%%%%%%%%%%%%%%%%%%%%%%%%%%%%%%%%%%%%%%%
\paragraph{Momento de inércia de um disco/cilindro}
%%%%%%%%%%%%%%%%%%%%%%%%%%%%%%%%%%%%%%%%%%%%%%%%%%%

Para determinarmos o momento de inércia de um disco/cilindro de massa $M$, raio $R$ e altura $L$, em torno do eixo que passa pelo centro de massa, perpendicularmente à face plana, basta empregarmos a Equação~\eqref{Eq:MomInerciaDistContinua} e escrever $dm$ de acordo com a Equação~\eqref{Eq:DiferencialDeMassaVolume}:
\begin{equation}
    I = \iiint_V r_\perp^2 \, \rho(\vec{r}) \, dV.
\end{equation}
%
\begin{marginfigure}[-4cm]
    \centering
    \begin{tikzpicture}[>=Stealth]
        
    % topo
    \draw (0,0) ellipse (1.5 and 0.7);
    
    \draw[<->] (0,0) -- node[above]{$R$} (-20:1.25);
    
    % laterais
    \draw (-1.5,0) -- (-1.5,-1.5);
    \draw[|<->|] (-1.75,0) -- node[left]{$L$} (-1.75,-1.5);
    \draw (1.5,-1.5) -- (1.5,0);  
    
    % fundo
    \draw[dotted] (-1.5,-1.5) arc (180:360:1.5 and -0.7);
    \draw (-1.5,-1.5) arc (180:360:1.5 and 0.7);
    
    % Eixo z
    \draw[dashdotted,->] (0,-0.5) -- (0,1.5) node[below left]{$z$};
    \draw[dotted] (0,-0.5) -- (0,-2.2);
    \draw[dashdotted] (0,-2.2) -- (0,-2.9);
    
    \draw[fill] (0,-0.8) circle (1pt) node[below right] {CM};
    \draw[fill] (0,0) circle (0.5pt);
    \draw[fill] (0, -1.5) circle (0.5pt);
    
    \draw[->] (0,1.05) [partial ellipse=-225:60:0.3cm and 0.125cm];
    
    \end{tikzpicture}
    \caption{Cilíndro de massa $M$, raio $R$, e altura $L$. \label{Fig:MomInerciaDiscoCilindro}}
\end{marginfigure}
%
Se a densidade do corpo é homogênea, podemos escrever $\rho(\vec{r}) \equiv \rho$. O volume $V$ sobre o qual a integração é realizada é o próprio volume determinado pelo cilindro. Claramente uma integração em coordenadas cilíndricas é mais conveniente, logo, reescrevemos $dV$ como $dV = r \, dr \, d\theta \, dz$ e utilizamos os limites $r = (0, R)$, $\theta = (0, 2\pi)$ e $z = (0,L)$. Note ainda que a coordenada $r$ das coordenadas cilíndricas é a própria distância ao eixo de rotação e podemos omitir o índice $\perp$. Assim:
\begin{align}
    I &= \int_0^L \int_0^{2\pi} \int_0^R r^3 \, \rho \, dr \, d\theta \, dz \\
    &= 2\pi \cdot L \cdot \rho \cdot \frac{R^4}{4} \\
    &= [\pi R^2 \cdot L \cdot \rho] \cdot \frac{R^2}{2}.
\end{align}
%
Note que o termo entre colchetes nada mais é que o volume do cilindro multiplicado pela sua densidade, ou seja, corresponde à sua massa. Consequentemente, podemos escrever:
\begin{equation}
    I = \frac{MR^2}{2}.\mathnote{Momento de inércia de um disco/cilíndro em torno de um eixo perpendicular à face plana e que passa pelo centro de massa.}
\end{equation}

%%%%%%%%%%%%%%%%%%%%%%%%%%%%%%%%%%%%%%%%%%%%%%%%%%%%
\paragraph{Momento de inércia de um tubo cilíndrico}
%%%%%%%%%%%%%%%%%%%%%%%%%%%%%%%%%%%%%%%%%%%%%%%%%%%%

\begin{marginfigure}[8cm]
\centering
\begin{tikzpicture}[>=Stealth]
    
% topo
\draw (0,0) ellipse (1.25 and 0.5);
\draw (0,0) ellipse (1.5 and 0.7);

\draw[<->] (0,0) -- node[above]{$R_i$} (-1.25,0);
\draw[<->] (0,0) -- node[above]{$R_e$} (-20:1.25);

% laterais
\draw[dotted] (-1.25,0) -- (-1.25,-1.5);
\draw[dotted] (1.25,-1.5) -- (1.25,0);
\draw (-1.5,0) -- (-1.5,-1.5);
\draw[|<->|] (-1.75,0) -- node[left]{$L$} (-1.75,-1.5);
\draw (1.5,-1.5) -- (1.5,0);  

% fundo
\draw[dotted] (-1.25,-1.5) arc (180:360:1.25 and 0.5);
\draw[dotted] (-1.25,-1.5) arc (180:360:1.25 and -0.5);
\draw[dotted] (-1.5,-1.5) arc (180:360:1.5 and -0.7);
\draw (-1.5,-1.5) arc (180:360:1.5 and 0.7);

% Eixo z
\draw[dashdotted,->] (0,-0.5) -- (0,1.5) node[below left]{$z$};
\draw[dotted] (0,-0.5) -- (0,-2.2);
\draw[dashdotted] (0,-2.2) -- (0,-2.9);

\draw[fill] (0,-0.8) circle (1pt) node[below right] {CM};
\draw[fill] (0,0) circle (0.5pt);
\draw[fill] (0, -1.5) circle (0.5pt);

\draw[->] (0,1.05) [partial ellipse=-225:60:0.3cm and 0.125cm];

\end{tikzpicture}
\caption{Tubo cilíndrico de massa $M$, altura $L$ e raios interno e externo $R_i$ e $R_e$, respectivamente. \label{Fig:MomInerciaTuboCilindrico}}
\end{marginfigure}

Para um tubo cilíndrico de massa $M$, altura $L$ e raios interno e externo $R_i$ e $R_e$, e que gira em torno do eixo que passa pelo centro de massa, coincidindo com o eixo do tubo, podemos seguir o mesmo procedimento que o caso de um cilíndro. Será necessário, no entanto, substituir os limites de integração em $r$ por $r = (R_i, R_e)$:
\begin{align}
    I &= \iiint_V r_\perp^2 \, \rho(\vec{r}) \, dV \\
    &= \int_0^L \int_0^{2\pi} \int_{R_i}^{R_e} r^3 \, \rho \, dr \, d\theta \, dz \\
    &= 2\pi \cdot L \cdot \rho \cdot \frac{R_e^4 - R_i^4}{4}.
\end{align}
%
Notando que 
\begin{equation}
    R_e^4 - R_i^4 = (R_e^2 - R_i^2)\cdot(R_e^2 + R_i^2),
\end{equation}
%
podemos reecrever a expressão para o momento de inércia como
\begin{equation}
    I = [\pi \cdot (R_e^2 - R_i^2) \cdot L \cdot \rho] \cdot \frac{R_e^2 + R_i^2}{2}.
\end{equation}
%
Como no caso do cilíndro, verificamos que o termo entre colchetes corresponde ao produto do volume do corpo e de sua densidade, equivalendo à massa do corpo. Consequentemente, podemos escrever para o momento de inércia
\begin{equation}
    I = M \; \frac{R_e^2 + R_i^2}{2}. \mathnote{Momento de inércia de um tubo cilíndrico em torno do eixo central do tubo.}
\end{equation}

%%%%%%%%%%%%%%%%%%%%%%%%%
\section{Roda de Maxwell}
%%%%%%%%%%%%%%%%%%%%%%%%%

% Pra essa figura ficar boa, precisaria fazer
% os traços sobre uma foto, assim a perspectiva
% ficaria correta
\begin{marginfigure}[3cm]
\centering
\begin{tikzpicture}[>=Stealth, rotate = -95,
     interface/.style={
        % superfície
        postaction={draw,decorate,decoration={border,angle=-45,
                    amplitude=0.2cm,segment length=2mm}}},
    ]

%%% Teto

    \draw[interface] (-2.6,2.25) -- (-2.45,-2.5);
    
%%% Disco

\draw (0,0) ellipse (1.25 and 0.75);
\draw (0,0) ellipse (1 and 0.5);
\begin{scope}
    \clip (0,0) ellipse (1 and 0.5);
    \draw (-1,-0.1) arc (180:360:1 and -0.5);
\end{scope}

\draw (-1.25,0) -- (-1.25,-0.4);
\draw (1.25,-0.4) -- (1.25,0);  

\draw (-1.25,-0.4) arc (180:360:1.25 and 0.75);
%\draw[dotted] (-1.25,-0.4) arc (180:360:1.25 and -0.75);

%%% Semieixo direito
\fill[white] (0.125,-0.075) rectangle (-0.125,2);
\draw (0.125,-0.075) -- +(0,2.075);
\draw (-0.125,-0.075) -- +(0,2.075);
\draw (0,2) ellipse (0.125 and 0.075);
\draw (-0.125,-0.075) arc[start angle = 180, end angle = 360, x radius = 0.125, y radius = 0.075];

% Corda
\draw (-0.125,1.75) arc[start angle = 180, end angle = 360, x radius = 0.125, y radius = 0.075];
\draw (-0.125,1.70) arc[start angle = 180, end angle = 360, x radius = 0.125, y radius = 0.075];
\draw (-0.125,1.65) arc[start angle = 180, end angle = 360, x radius = 0.125, y radius = 0.075];
\draw (-0.125,1.60) arc[start angle = 180, end angle = 360, x radius = 0.125, y radius = 0.075];
\draw (-0.125,1.55) arc[start angle = 180, end angle = 360, x radius = 0.125, y radius = 0.075];
\draw (-0.125,1.50) arc[start angle = 180, end angle = 360, x radius = 0.125, y radius = 0.075];
\draw (-0.125,1.45) arc[start angle = 180, end angle = 360, x radius = 0.125, y radius = 0.075];
\draw (-0.125,1.40) arc[start angle = 180, end angle = 360, x radius = 0.125, y radius = 0.075];
\draw (-0.125,1.8) -- (-2.58,1.55);


%%% Semieixo esquerdo
%\draw[dotted] (0,-0.4) ellipse (0.125 and 0.075);
%\draw[dotted] (0.125, -0.4) -- (0.125,-1.15);
%\draw[dotted] (-0.125, -0.4) -- (-0.125,-1.15);
\draw (0.125, -1.15) -- (0.125,-2.13);
\draw (-0.125, -1.15) -- (-0.125,-2.13);
\draw (-0.125,-2.13) arc[start angle = 180, end angle = 360, x radius = 0.125, y radius = 0.075];
% Corda
\draw (-0.125,-1.85) arc[start angle = 180, end angle = 360, x radius = 0.125, y radius = 0.075];
\draw (-0.125,-1.80) arc[start angle = 180, end angle = 360, x radius = 0.125, y radius = 0.075];
\draw (-0.125,-1.75) arc[start angle = 180, end angle = 360, x radius = 0.125, y radius = 0.075];
\draw (-0.125,-1.70) arc[start angle = 180, end angle = 360, x radius = 0.125, y radius = 0.075];
\draw (-0.125,-1.65) arc[start angle = 180, end angle = 360, x radius = 0.125, y radius = 0.075];
\draw (-0.125,-1.60) arc[start angle = 180, end angle = 360, x radius = 0.125, y radius = 0.075];
\draw (-0.125,-1.55) arc[start angle = 180, end angle = 360, x radius = 0.125, y radius = 0.075];
\draw (-0.125,-1.50) arc[start angle = 180, end angle = 360, x radius = 0.125, y radius = 0.075];
\draw (-0.125,-1.9) -- (-2.45,-2.1);

\end{tikzpicture}
\caption{Roda de Maxwell.\label{Fig:RodaDeMaxwell}}
\end{marginfigure}

\begin{marginfigure}[1.5cm]
\centering
\begin{tikzpicture}[>=Stealth, scale = 0.8,
     interface/.style={
        % superfície
        postaction={draw,decorate,decoration={border,angle=-45,
                    amplitude=0.2cm,segment length=2mm}}},
    ]

    \draw[fill] (0,0) circle (0.75pt);
    
    \draw (0,0) circle (2cm);
    \draw (0,0) circle (1.5cm);
    \draw (0,0) circle (0.25cm);
    
    \draw (-0.25,0) -- (-0.25,3);
    \draw[interface] (2,3) -- (-2,3);
    
    \draw[dashdotted] (0,0) -- (3.75,0);
    
    \draw[dotted] (0,0.25) -- (0.75,0.25);
    \draw[|<-] (0.75,0.25) -- (0.75,0.75);
    \draw[|<-] (0.75,0) -- node[below right]{$r$} (0.75,-0.5);
    
    \draw[dotted] (0,2) -- (3.5,2);
    \draw[|<->|] (2.5,1.5) -- node[right]{$R_i$} (2.5,0);

    \draw[dotted] (0,1.5) -- (2.5,1.5);
    \draw[|<->|] (3.5,2) -- node[right]{$R_e$} (3.5,0);

\end{tikzpicture}
\caption{Roda de Maxwell, visão lateral.\label{Fig:RodaDeMaxwellVisaoLateral}}
\end{marginfigure}

Um dos sistemas mais simples que exibe rotação e translação é o que denominamos como ``Roda de Maxwell'', que consiste em um disco ligado a um eixo suspenso por cordas. As cordas são enroladas no eixo e o sistema é liberado a partir do repouso, deixando o disco descer e desenrolar a corda, de maneira similar a um ioiô. Apesar de o disco exibir uma rotação e uma translação, esses movimentos não são independentes, uma vez que a distância percorrida pelo centro de massa está ligada ao comprimento de corda que é desenrolado do eixo. A relação entre o deslocamento do centro de massa e o deslocamento angular, que determina a quantidade $\ell$ de corda desenrolada, é
\begin{equation}
    \Delta y_{\text{CM}} \equiv \ell = r \Delta \theta.
\end{equation}
%
A partir dessa relação, também temos
\begin{align}
    v_{\text{CM}} &= r \omega \\
    a_{\text{CM}} &= r \alpha.
\end{align}
%
Note que as relações acima podem ou não ter um sinal negativo, dependendo da escolha do sistema de referência. Se adotarmos um eixo vertical apontando para cima, as expressões acima estão adequadas.

\begin{marginfigure}[7.5cm]
    \centering
    \begin{tikzpicture}[>=Stealth, scale = 0.8]
    
        \draw[pattern = north west lines] (0,2) -- (0.5, 2) -- (0.5,1.5) -- (0.25,1.5) -- (0.25,0.25) -- (3,0.25) -- (3,-0.25) -- (0.25,-0.25) -- (0.25, -1.5) -- (0.5, -1.5) -- (0.5, -2) -- (-0.5, -2) -- (-0.5, -1.5) -- (-0.25, -1.5) -- (-0.25, -0.25) -- (-3, -0.25) -- (-3, 0.25) -- (-0.25, 0.25) -- (-0.25,1.5) --(-0.5,1.5) -- (-0.5, 2) -- cycle;
        
        \draw[|<->|] (-3,0.875) -- node[above]{$L$} (-0.25,0.875);
        \draw[|<->|] (3,0.875) -- node[above]{$L$} (0.25,0.875);
        \draw[dotted] (-3,0.875) -- (-3,0.25);
        \draw[dotted] (3,0.875) -- (3,0.25);
        
        \draw[|<->|] (-0.5, 2.5) -- node[above]{$\ell_e$} (0.5,2.5);
        \draw[dotted] (-0.5, 2) -- (-0.5, 2.5);
        \draw[dotted] (0.5, 2) -- (0.5, 2.5);
        
        \draw[->] (-1,-0.875) -- node[below]{$\ell_i$} (-0.25,-0.875);
        \draw[->] (1,-0.875) -- (0.25,-0.875);
    
    \end{tikzpicture}
    \caption{Roda de Maxwell, seção frontal.\label{Fig:RodaDeMaxwellSecaoFrontal}}
    \end{marginfigure}

Aplicando a Segunda Lei de Newton para translação ao centro de massa, obtemos
\begin{description}
\item[Eixo $y$:]
\begin{align}
    F_R^y &= m_t a_{\text{CM}, y} \\
    T - P &= m_t a,
\end{align}
\end{description}
%
onde adotamos $a \equiv a_{\text{CM}, y}$ e $m_t$ representa a massa total do sistema (soma da massa do disco e dos dois semieixos).

Aplicando a Segunda Lei de Newton para a rotação, temos
\begin{align}
    \tau &= I\alpha \\
    -Tr &= I\alpha.
\end{align}

Através dos resultados obtidos acima, podemos montar um sistema de equações
\begin{equation}
\begin{system}
    T - P &= m_t a \\
    -Tr &= I\alpha \\
    a &= r \alpha,
\end{system}
\end{equation}
%
cuja solução é
\begin{align}
    a &= \frac{m_t g}{m_t + I / r^2} \\
    &= \frac{1}{1+\frac{I}{m_t r^2}} \cdot g. \label{Eq:AcelRodaDeMaxwellGenerica}
\end{align}

Para determinar o momento de inércia total, podemos dividir a roda de Maxwell em três partes: os dois semieixos, o disco central, e o ``tubo'' externo. Como não sabemos a massa de cada uma das partes separadamente, teremos que determiná-la através da densidade e do volume de cada uma das partes. Assim, obtemos
\begin{align}
    I_{\text{eixos}} &= 2\cdot\frac{1}{2} MR^2 \\
    &=2\cdot \frac{1}{2} \rho V_e \cdot r^2 \\
    I_{\text{disco}} &= \frac{1}{2} \cdot MR^2 \\
    &= \frac{1}{2} \cdot\rho V_d \cdot R_i^2 \\
    I_{\text{tubo}} &= \frac{1}{2} \cdot M (R_e^2 - R_i^2) \\
    &= \frac{1}{2} \cdot \rho V_t \cdot (R_e^2 - R_i^2),
\end{align}
%
onde os volumes são dados por
\begin{align}
    V_e &= \pi r^2 L \\
    V_d &= \pi R_i^2 \ell_i \\
    V_t &= \pi (R_e^2 - R_i^2)\ell_e.
\end{align}
%
Somando as expressões acima, obtemos o momento de inércia total:
\begin{equation}\label{eq:MomentoDeInerciaTotal}
    I = \frac{1}{2} \rho [V_e r^2 + V_d R_i^2 + V_t (R_e^2 - R_i^2)].
\end{equation}

%%%%%%%%%%%%%%%%%%%%%%%%%%%%%%%%%%%%%%%%%%%%%%%%%%%%%%%%%%%%%%%%%%%%%%%%%%%%%%%
\section{Experimento}
%%%%%%%%%%%%%%%%%%%%%%%%%%%%%%%%%%%%%%%%%%%%%%%%%%%%%%%%%%%%%%%%%%%%%%%%%%%%%%%

%%%%%%%%%%%%%%%%%%%%%%
\subsection{Objetivos}
\label{Sec:ObjetivosRodaDeMaxwell}
%%%%%%%%%%%%%%%%%%%%%%

\begin{itemize}
	\item Determinar a densidade do material que compõe a roda de Maxwell;
	\item Determinar o momento de inércia e a aceleração do centro de massa através das Equações~\eqref{eq:MomentoDeInerciaTotal} e~\eqref{Eq:AcelRodaDeMaxwellGenerica};
	\item Determinar a aceleração do centro de massa experimentalmente, através da cinemática;
    \item Determinar o erro percentual entre o valor teórico e o valor experimental para a aceleração do centro de massa;
    \item Verificar a validade das expressões para o momento de inércia e para a segunda lei de Newton para a rotação.
\end{itemize}

%%%%%%%%%%%%%%%%%%%%%%%%%%%%%%%%%%%%%%%%%%%%%%%%%%%%%%%%%%%%%%%%%%%%%%%%%%%%%%%
\section{Material Necessário}
%%%%%%%%%%%%%%%%%%%%%%%%%%%%%%%%%%%%%%%%%%%%%%%%%%%%%%%%%%%%%%%%%%%%%%%%%%%%%%%

\begin{itemize}
	\item Aparato da roda de Maxwell;
    \item Réguas;
    \item Paquímetro;
	\item Balança digital.
\end{itemize}

%%%%%%%%%%%%%%%%%%%%%%%%%%%%%%%%%%%%%%%%%%%%%%%%%%%%%%%%%%%%%%%%%%%%%%%%%%%%%%%
\section{Procedimento Experimental}
%%%%%%%%%%%%%%%%%%%%%%%%%%%%%%%%%%%%%%%%%%%%%%%%%%%%%%%%%%%%%%%%%%%%%%%%%%%%%%%

%%%%%%%%%%%%%%%%%%%%%
\subsection{Determinação da densidade}
%%%%%%%%%%%%%%%%%%%%%
\begin{enumerate}
	\item Afira o valor da massa total $m_t$ da roda de Maxwell usando a balança. Anote o valor obtido na Tabela~\ref{DadosRodaDeMaxwell};
	\item Meça os valores de $r$, $R_i$, $R_e$, $L$, $\ell_i$, e $\ell_e$ e os anote na Tabela~\ref{DadosRodaDeMaxwell};
	\item Determine os resultados para os volumes $V_e$, $V_d$, e $V_t$. Anote os resultados na Tabela~\ref{DadosRodaDeMaxwell}.
\end{enumerate}

%%%%%%%%%%%%%%%%%%%%%
\subsection{Determinação da aceleração do centro de massa através da cinemática}
%%%%%%%%%%%%%%%%%%%%%
\begin{enumerate}
	\item Ajuste o indicador de posição inicial para que ele coincida com a posição inicial do eixo da roda de Maxwell. Anote o valor na Tabela~\ref{DadosRodaDeMaxwell};
	\item Ajuste a posição do sensor para uma distância de \np[cm]{5,00} abaixo da posição inicial do sensor, anotando tal valor na Tabela~\ref{DadosRodaDeMaxwell};
	\item Gire a roda manualmente de forma que os fios de sustentação enrolem sobre os eixos até que a roda toque o eletroímã e ligue a chave de acionamento do eletroímã para que a roda fique presa.
	\item Acionando a chave, libere a roda para que ela possa se mover. Anote o valor registrado pelo cronômetro na Tabela~\ref{DadosRodaDeMaxwell}. Não pare a roda após ela passar pelo sensor, pois ao chegar ao final do curso permitido pelos fios que a sustentam, ela voltará a subir e facilitará o processo de ajuste para a tomada do próximo valor de tempo. Termine de enrolar os fios girando a roda manualmente até que ela volte a tocar o eletroímã e acione a chave para que ela fique presa.
	\item Mova o sensor para uma posição \np[cm]{5,00} abaixo da posição anterior. Libere a roda para que ela possa se mover e anote o resuldado registrado pelo cronômetro na Tabela~\ref{DadosRodaDeMaxwell}.
	\item Repita os dois itens anteriores até que a posição do sensor exceda o limite máximo de movimento da roda.
\end{enumerate}

%%%%%%%%%%%%%%%%%%%%%%%%%%%%%%%%%%%%%%%%%%%%%%%%%%%%%%%%%%%%%%%%%%%%%%%%%%%%%%%
%%%%%%%%%%%%%%%%%%%%%%%%%%%%%%%%%%%%%%%%%%%%%%%%%%%%%%%%%%%%%%%%%%%%%%%%%%%%%%%
%%%%%%%%%%%%%%%%%%%%%%%%%%%%%%%%%%%%%%%%%%%%%%%%%%%%%%%%%%%%%%%%%%%%%%%%%%%%%%%
%%%%%%%%%%%%%%%%%%%%%%%%%%%%%%%%%%%%%%%%%%%%%%%%%%%%%%%%%%%%%%%%%%%%%%%%%%%%%%%
\cleardoublepage

\noindent{}{\huge\textit{Roda de Maxwell}}

\vspace{15mm}

\begin{fullwidth}
\noindent{}\makebox[0.6\linewidth]{Turma:\enspace\hrulefill}\makebox[0.4\textwidth]{  Data:\enspace\hrulefill}
\vspace{5mm}

\noindent{}\makebox[0.6\linewidth]{Aluno(a):\enspace\hrulefill}\makebox[0.4\textwidth]{  Matrícula:\enspace\hrulefill}

\noindent{}\makebox[0.6\linewidth]{Aluno(a):\enspace\hrulefill}\makebox[0.4\textwidth]{  Matrícula:\enspace\hrulefill}

\noindent{}\makebox[0.6\linewidth]{Aluno(a):\enspace\hrulefill}\makebox[0.4\textwidth]{  Matrícula:\enspace\hrulefill}

\noindent{}\makebox[0.6\linewidth]{Aluno(a):\enspace\hrulefill}\makebox[0.4\textwidth]{  Matrícula:\enspace\hrulefill}

\noindent{}\makebox[0.6\linewidth]{Aluno(a):\enspace\hrulefill}\makebox[0.4\textwidth]{  Matrícula:\enspace\hrulefill}
\end{fullwidth}

\vspace{5mm}

%%%%%%%%%%%%%%%%%%%%%%%%%%%%%%%%%%%%%%%%%%%%%%%%%%%%%%%%%%%%%%%%%%%%%%%%%%%%%%%
\section{Questionário}
%%%%%%%%%%%%%%%%%%%%%%%%%%%%%%%%%%%%%%%%%%%%%%%%%%%%%%%%%%%%%%%%%%%%%%%%%%%%%%%

\begin{question}[type={exam}]{2}
Preencha as colunas de dados experimentais das tabelas com o número adequado de algarismos significativos e unidades.
\end{question}

\begin{question}[type={exam}]{2}
Determine os volumes $V_e$, $V_c$ e $V_t$ e os dados calculados $\Delta y$ e $t^2$, completando a Tabela~\ref{DadosRodaDeMaxwell}. Utilize o número adequado de algarismos significativos.
\end{question}

\begin{question}[type={exam}]{2}
\begin{enumerate}[label=\roman*.]
    \item Elabore um gráfico de $\Delta y \times t^2$ com os dados obtidos.
    \item Faça uma regressão linear e a adicione ao gráfico.
\end{enumerate}
\end{question}

\begin{question}[type={exam}]{2}
\begin{enumerate}[label=\roman*.]
    \item Determine o significado físico do coeficiente angular, isto é, determine o que tal coeficiente representa.
    \item Determine o valor da aceleração do centro de massa usando o coeficiente angular e calcule o erro percentual em relação ao valor determinado através das Equações~\eqref{Eq:AcelRodaDeMaxwellGenerica} e~\eqref{eq:MomentoDeInerciaTotal}.
\end{enumerate}
\end{question}

\begin{question}[type={exam}]{2}
Considerando os objetivos do experimento, listados na Seção~\ref{Sec:ObjetivosRodaDeMaxwell}, e os resultados obtidos nas questões anteriores, discuta quais objetivos foram atingidos com sucesso, justificando suas conclusões. Se algum objetivo não foi atingido, discuta quais são os possíveis motivos do insucesso e que providências podem ser tomadas para que eles sejam alcançados.
\end{question}

\vfill
%%%%%%%%%%%%%%%%%%%%%%%%%%%%%%%%%%%%%%%%%%%%%%%%%%%%%%%%%%%%%%%%%%%%%%%%%%%%%%%
\pagebreak
\section{Tabelas}
%%%%%%%%%%%%%%%%%%%%%%%%%%%%%%%%%%%%%%%%%%%%%%%%%%%%%%%%%%%%%%%%%%%%%%%%%%%%%%%

\begin{table*}[!ht]
    \centering
    \begin{tabular}{lp{25mm}p{25mm}p{25mm}l}
    \toprule
        &\multicolumn{4}{l}{\textbf{Raios e Comprimentos}} \\
        \cmidrule{2-3}
        & $r$ \cellcolor[gray]{0.89} & \cellcolor[gray]{0.92} \\
        & $R_i$ \cellcolor[gray]{0.95} & \cellcolor[gray]{0.97} \\
        & $R_e$ \cellcolor[gray]{0.89} & \cellcolor[gray]{0.92} \\
        & $L$ \cellcolor[gray]{0.95} & \cellcolor[gray]{0.97} \\
        & $\ell_i$ \cellcolor[gray]{0.89} & \cellcolor[gray]{0.92} \\
        & $\ell_e$ \cellcolor[gray]{0.95} & \cellcolor[gray]{0.97} \\
        \cmidrule{2-3}
        \\
        &\multicolumn{4}{l}{\textbf{Volumes}} \\
        \cmidrule{2-3}
        & $V_e$ \cellcolor[gray]{0.89} & \cellcolor[gray]{0.92} \\
        & $V_d$ \cellcolor[gray]{0.95} & \cellcolor[gray]{0.97} \\
        & $V_t$ \cellcolor[gray]{0.89} & \cellcolor[gray]{0.92} \\
        \cmidrule{2-3}
        \\
        &\multicolumn{4}{l}{\textbf{Dados Experimentais}} \\
        \cmidrule{2-4}
        & $y_i$ & $y_f$ & $t$ & \\
        \cmidrule{2-4}
        & \cellcolor[gray]{0.89} & \cellcolor[gray]{0.92} & \cellcolor[gray]{0.89} \\
        & \cellcolor[gray]{0.95} & \cellcolor[gray]{0.97} & \cellcolor[gray]{0.95} \\
        & \cellcolor[gray]{0.89} & \cellcolor[gray]{0.92} & \cellcolor[gray]{0.89} \\
        & \cellcolor[gray]{0.95} & \cellcolor[gray]{0.97} & \cellcolor[gray]{0.95} \\
        & \cellcolor[gray]{0.89} & \cellcolor[gray]{0.92} & \cellcolor[gray]{0.89} \\
        & \cellcolor[gray]{0.95} & \cellcolor[gray]{0.97} & \cellcolor[gray]{0.95} \\
        & \cellcolor[gray]{0.89} & \cellcolor[gray]{0.92} & \cellcolor[gray]{0.89} \\
        & \cellcolor[gray]{0.95} & \cellcolor[gray]{0.97} & \cellcolor[gray]{0.95} \\
        & \cellcolor[gray]{0.89} & \cellcolor[gray]{0.92} & \cellcolor[gray]{0.89} \\
        & \cellcolor[gray]{0.95} & \cellcolor[gray]{0.97} & \cellcolor[gray]{0.95} \\
        \cmidrule{2-4}
    \\
        & \multicolumn{3}{l}{\textbf{Dados calculados}} \\
        \cmidrule{2-3}
        & $\Delta y$ & $t^2$ \\
        \cmidrule{2-3}
        & \cellcolor[gray]{0.89} & \cellcolor[gray]{0.92} \\ 
        & \cellcolor[gray]{0.95} & \cellcolor[gray]{0.97} \\ 
        & \cellcolor[gray]{0.89} & \cellcolor[gray]{0.92} \\ 
        & \cellcolor[gray]{0.95} & \cellcolor[gray]{0.97} \\ 
        & \cellcolor[gray]{0.89} & \cellcolor[gray]{0.92} \\ 
        & \cellcolor[gray]{0.95} & \cellcolor[gray]{0.97} \\ 
        & \cellcolor[gray]{0.89} & \cellcolor[gray]{0.92} \\ 
        & \cellcolor[gray]{0.95} & \cellcolor[gray]{0.97} \\ 
        & \cellcolor[gray]{0.89} & \cellcolor[gray]{0.92} \\ 
        & \cellcolor[gray]{0.95} & \cellcolor[gray]{0.97} \\ 
        \cmidrule{2-3}
    \bottomrule
    \end{tabular}
    \caption[][5mm]{Dados para a Roda de Maxwell}
    \label{DadosRodaDeMaxwell}
    \end{table*}
