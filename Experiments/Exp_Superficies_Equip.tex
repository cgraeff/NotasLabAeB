%%%%%%%%%%%%%%%%%%%%%%%%%%%%%%%%%%%%%%%%%%%%%%%%%%%%%%%%%%%%%%%%%%%%%%%%%%%%%%%
\chapter{Superfícies Equipotenciais} % Sem "Experiência 01" ou qualquer outro número
\label{Chap:SupEquip}        % para poder trocar a ordem com facilidade
%%%%%%%%%%%%%%%%%%%%%%%%%%%%%%%%%%%%%%%%%%%%%%%%%%%%%%%%%%%%%%%%%%%%%%%%%%%%%%%

\begin{fullwidth}\it
	Faremos um experimentos para determinar as superfícies equipotenciais de um arranjo de dipolar cargas elétricas. Através delas, seremos capazes de determinar a direção das linhas de campo elétrico.
\end{fullwidth}

%%%%%%%%%%%%%%%%%%%%%%%%%%%%%%%%%%%%%%%%%%%%%%%%%%%%%%%%%%%%%%%%%%%%%%%%%%%%%%%
\section{Potencial elétrico de um arranjo dipolar de cargas elétricas}
%%%%%%%%%%%%%%%%%%%%%%%%%%%%%%%%%%%%%%%%%%%%%%%%%%%%%%%%%%%%%%%%%%%%%%%%%%%%%%%

O trabalho realizado por uma força $\vec{F}$ pode ser calculado no caso mais geral através da \emph{integral de linha}\footnote{Também chamada de \emph{integral de caminho}.}
\begin{equation}
    W = \int_C \vec{F}(\vec{r}) \cdot d\vec{r},
\end{equation}
%
onde $C$ representa um caminho no espaço. Além disso, se a força é conservativa, podemos calcular o trabalho como o negativo da diferença de valores de uma função escalar $U(\vec{r})$ conhecida como energia potencial:
\begin{align}
    W &= \Delta U(\vec{r}) \\
    &= U(\vec{r}_f) - U(\vec{r}_i).
\end{align}

No caso particular da força eletrostática, é conveniente denotar a força $F(\vec{r})$ como sendo o produto entre o campo elétrico e o valor $q_0$ da carga que se desloca e cujo trabalho estamos interessados em determinar. Assim, podemos escrever
\begin{equation}\label{Eq:EnergiaPotencialParaDeslEmCampoEletrico}
    \Delta U(\vec{r}) = - \int_C q_0 \cdot \vec{E}(\vec{r}) \cdot d\vec{r}.
\end{equation}

Em fenômenos eletromagnéticos é bastante comum que não haja um interesse particular na quantidade de trabalho realizado em uma carga específica, mas sim na quantidade de trabalho realizada \emph{por unidade de carga}. Dessa maneira, definimos uma nova grandeza, conhecida como \emph{potencial elétrico},\footnote{Geralmente só dizemos \emph{potencial}, deixando o `elétrico' subentendido.} definida como
\begin{equation}
    V(\vec{r}) \equiv \frac{U(\vec{r})}{q_0}.
\end{equation}
%
Substituindo essa definição na expressão~\eqref{Eq:EnergiaPotencialParaDeslEmCampoEletrico}, obtemos
\begin{equation}\label{Eq:ExprPotencialEletrico}
    \Delta V(\vec{r}) = - \int_C \vec{E}(\vec{r}) \cdot d\vec{r}. \mathnote{Definição de potencial elétrico.}
\end{equation}

O potencial elétrico $V(\vec{r})$ é um campo escalar cujo valor é de especial interesse em aplicações práticas. Devido a características dos sistemas elétricos que empregamos no dia a dia, raramente nos preocupamos com sua dependência na posição $\vec{r}$ e estamos simplesmente interessados na \emph{diferença} de potencial $\Delta V$ entre dois pontos quaisquer de um circuito elétrico. Essa diferença é o que medimos com um multímetro e denominamos como \emph{tensão}, ou \emph{diferença de potencial}, ou ainda \emph{voltagem} (os três termos são equivalentes). Suas unidades são $\rm{J}/\rm{C}$ (`joule por coulomb'). 

Note que, da mesma maneira que o trabalho está relacionado à \emph{diferença de energia potencial} entre dois pontos, a tensão está relacionada à \emph{diferença de potencial elétrico}, isto é, não há interesse no valor particular de potencial elétrico, mas sim na diferença de valores entre dois pontos quaisquer de um circuito.\footnote{Isso é um reflexo do fato de que se o potencial é igual entre dois pontos, então a energia potencial também é, o que faz com que não haja trabalho algum realizado no deslocamento de uma carga entre as duas posições.} Quando tratamos um sistema de cargas de um ponto de vista da Física, costumamos escolher $\vec{r} = \infty$ como sendo o ponto onde o potencial é nulo. Em circuitos, é comum adotar o polo negativo de baterias como sendo o ponto de potencial nulo, ou no caso de sistemas mais complexos, \emph{o potencial do solo} é considerado nulo: essa é a razão pela qual dizemos que há um \emph{aterramento} de um sistema.

%%%%%%%%%%%%%%%%%%%%%%%%%%%%%%%%%%%%%%
\paragraph{Superfícies equipotenciais}
%%%%%%%%%%%%%%%%%%%%%%%%%%%%%%%%%%%%%%

Se considerarmos uma configuração qualquer de cargas, existe uma superfície contígua formada por pontos onde o potencial tem o mesmo valor. Tal superfície é o que denominamos de \emph{superfície equipotencial}. No caso de uma carga pontual, por exemplo, é relativamente simples entender que as superfícies equipotenciais são esferas concentricas e em cujos centros se encontra a carga elétrica em questão: como a força eletrostática e, consequentemente, o campo elétrico são funções somente da distância $r$ entre a carga que gera o campo e a carga de teste, o trabalho por unidade de carga efetuado no deslocamento entre uma posição qualquer e qualquer dos pontos de uma superfície esférica centrada na carga que gera o campo é sempre o mesmo. Logo, todos os pontos dessa superfície se encontram sob o mesmo valor de potencial. Realizando a integração para esse caso particular, obtemos
\begin{align}
    \Delta V(\vec{r}) &= - \int_C \vec{E}(\vec{r}) \cdot d\vec{r} \\
    &= - \int_C \frac{kq}{r^2} \hat{r} \cdot d\vec{r}.
\end{align}
%
Como a força depende somente da distância entre as cargas, podemos tratar a integral acima como unidimensional, o que pode ser verificado matematicamente pelo produto escalar $\hat{r}\cdot d\vec{r} = dr$. Considerando um caminho que leve de um ponto $\vec{r}_A = \infty$ a um ponto qualquer $r_B = r$, temos\footnote{Estamos considerando que a carga que gera o campo se encontra na origem do sistema de coordenadas.} 
\begin{align}
    \Delta V(\vec{r}) &= - \int_{r_A}^{r_B} \frac{kq}{r^2} \,dr \\
    V(r_B) - V(r_A) &= - kq \int_{\infty}^{r} (r')^{-2} \,dr' \\
    &= kq [(r')^{-1} + C]_{\infty}^{r} \\
    &= \frac{kq}{r}.
\end{align}
%
Note que o potencial na posição $r_A = \infty$ é nulo, assim podemos escrever que o potencial elétrico devido a uma carga é dado por
\begin{equation}
    V(r) = \frac{kq}{r}.
\end{equation}

No caso uma configuração de cargas mais complexa, a visualização do potencial elétrico não é simples. Nesse caso precisamos calcular o potencial através da Equação~\eqref{Eq:ExprPotencialEletrico}, porém somando sobre todas as cargas elétricas:
\begin{equation}\label{Eq:ExprPotencialEletricoConjCargas}
    \Delta V_T(\vec{r}) = - \sum_{i=1}^{N}\int_C \vec{E}_i(\vec{r}_i) \cdot d\vec{r}. \mathnote{Potencial elétrico de um conjunto de cargas.}
\end{equation}
%
Note que o potencial elétrico $V_T$ devido a uma distribuição de várias cargas pode ser calculado através de uma simples soma, pois ele é definido como sendo o trabalho total realizado sobre a carga de teste devido às diferentes cargas da distrituição e \emph{o trabalho é uma grandeza aditiva}. Alternativamente, podemos calcular o campo elétrico total $E_T(\vec{r}) = \sum_{i = 1}^{N}$ e depois determinar o resultado da integral.

Para o caso de um \emph{dipolo elétrico}, isto é, uma distribuição de cargas elétricas formada por uma carga positiva e uma negativa ---~ambas de mesmo valor $q$~---, temos a soma dos potenciais de duas cargas pontuais. Considerando que as cargas estão nas posições $a$ e $-a$ de um eixo $x$ cuja origem se encontra no ponto médio entre as cargas, é possível escrever uma expressão simples para o potencial ao longo do eixo:
\begin{align}
    V(x) &= \frac{k\cdot(-q)}{x-a} + \frac{k\cdot q}{x + a} \\
    &= -\frac{2kqa}{x^2 - a}.
\end{align}


%%%%%%%%%%%%%%%%%%%%%%%%%%%%%%%%%%%%%%%%%%%%%
%\subsection{Se necessário, usar subsections}
%%%%%%%%%%%%%%%%%%%%%%%%%%%%%%%%%%%%%%%%%%%%%

%%%%%%%%%%%%%%%%%%%%%%%%%%%%%%%%%%%%%%%%%%%%%%%%%%%%%%%%%%%%%%%%%%%%%%%%%%%%%%%
\section{Experimento}
%%%%%%%%%%%%%%%%%%%%%%%%%%%%%%%%%%%%%%%%%%%%%%%%%%%%%%%%%%%%%%%%%%%%%%%%%%%%%%%

%%%%%%%%%%%%%%%%%%%%%%
\subsection{Objetivos}
%%%%%%%%%%%%%%%%%%%%%%

\begin{itemize}
	\item Obter as linhas equipotenciais de um dipolo elétrico em um plano que contém as cargas;
	\item Obter as linhas de campo elétrico a partir das linhas de potencial elétrico;
	\item Obter um gráfico do potencial ao longo da linha reta que une as cargas.
\end{itemize}

%%%%%%%%%%%%%%%%%%%%%%%%%%%%%%%%%%%%%%%%%%%%%%%%%%%%%%%%%%%%%%%%%%%%%%%%%%%%%%%
\section{Material Necessário}
%%%%%%%%%%%%%%%%%%%%%%%%%%%%%%%%%%%%%%%%%%%%%%%%%%%%%%%%%%%%%%%%%%%%%%%%%%%%%%%

\begin{itemize}
	\item Papel milimetrado;
	\item Fonte regulável;
	\item Multímetro;
	\item Placa de Petri com água;
	\item Eletrodos com suporte;
	\item Cabos de ligação (1 ponta de prova, 3 banana-jacaré);
	\item Becker com água.
\end{itemize}

%%%%%%%%%%%%%%%%%%%%%%%%%%%%%%%%%%%%%%%%%%%%%%%%%%%%%%%%%%%%%%%%%%%%%%%%%%%%%%%
\section{Procedimento Experimental}
%%%%%%%%%%%%%%%%%%%%%%%%%%%%%%%%%%%%%%%%%%%%%%%%%%%%%%%%%%%%%%%%%%%%%%%%%%%%%%%

%%%%%%%%%%%%%%%%%%%%%
%\subsection{Parte A} % Se necessário
%%%%%%%%%%%%%%%%%%%%%
\begin{enumerate}
	\item Tome duas folhas de papel milimetrado e desenhe sistemas de referência cujas origens se encontrem no centro e cujas marcas estejam distanciadas de 1 em \np[cm]{1,00}. Ambos os eixos devem variar de \np[cm]{-5,00} a \np[cm]{+5,00};
	\item Disponha a placa de Petri sobre o papel milimetrado de forma que seu centro coincida com a origem do sistema de referência;
	\item Ligue os eletrodos à fonte usando os cabos fornecidos;
	\item Disponha os eletrodos na placa de Petri de forma que eles estejam localizados sobre o eixo $x$, a uma distância de \np[cm]{5,00} da origem;
	\item Ajuste o multímetro para a função de voltímetro e ligue o terminal comum ao eletrodo negativo usando o cabo fornecido;
	\item Ligue a ponta de prova ao terminal marcado com \texttt{V$\mathdirectcurrent$} do multímetro;
	\item Ligue a fonte regulável e a ajuste para uma tensão de \np[V]{4,0};
	\item Coloque água na placa de Petri;
	\item Com a ponta de prova do multímetro, afira a tensão na origem do sistema de coordenadas e a cada \np[cm]{1,00} ao longo do eixo $y$ (tanto para valores positivos, quanto negativos). Na outra folha de papel milimetrado, marque um ponto na posição correspondente à da medida e anote os valores de tensão ao lado dos pontos;
	\item Afira a tensão ao longo do eixo $x$ desde o eletrodo negativo até o positivo a cada \np[cm]{1,00}.\footnote{Use as próprias marcas do eixo $x$ e os pontos intermediários entre elas, assim os valores ficarão mais simples e a origem do sistema de referência também estará incluída nos dados.} Anote os valores obtidos na Tabela~\ref{Tab:ValoresPotencialEletricoEixoX};
	\item Para a coluna $x = \np[cm]{-4,00}$ inicie com a ponta de prova em $y = \np[cm]{-5,00}$ e busque os valores de tensão dos pontos aferidos sobre o eixo $y$;
	\item Repita o procedimento do item acima varrendo todas as colunas até $x = \np[cm]{+4,0}$.
\end{enumerate}

%%%%%%%%%%%%%%%%%%%%%%%%%%%%%%%%%%%%%%%%%%%%%%%%%%%%%%%%%%%%%%%%%%%%%%%%%%%%%%%
%%%%%%%%%%%%%%%%%%%%%%%%%%%%%%%%%%%%%%%%%%%%%%%%%%%%%%%%%%%%%%%%%%%%%%%%%%%%%%%
%%%%%%%%%%%%%%%%%%%%%%%%%%%%%%%%%%%%%%%%%%%%%%%%%%%%%%%%%%%%%%%%%%%%%%%%%%%%%%%
%%%%%%%%%%%%%%%%%%%%%%%%%%%%%%%%%%%%%%%%%%%%%%%%%%%%%%%%%%%%%%%%%%%%%%%%%%%%%%%
\cleardoublepage

\noindent{}{\huge\textit{Superfícies equipotenciais}}

\vspace{15mm}

\begin{fullwidth}
\noindent{}\makebox[0.6\linewidth]{Turma:\enspace\hrulefill}\makebox[0.4\textwidth]{  Data:\enspace\hrulefill}
\vspace{5mm}

\noindent{}\makebox[0.6\linewidth]{Aluno(a):\enspace\hrulefill}\makebox[0.4\textwidth]{  Matrícula:\enspace\hrulefill}

\noindent{}\makebox[0.6\linewidth]{Aluno(a):\enspace\hrulefill}\makebox[0.4\textwidth]{  Matrícula:\enspace\hrulefill}

\noindent{}\makebox[0.6\linewidth]{Aluno(a):\enspace\hrulefill}\makebox[0.4\textwidth]{  Matrícula:\enspace\hrulefill}

\noindent{}\makebox[0.6\linewidth]{Aluno(a):\enspace\hrulefill}\makebox[0.4\textwidth]{  Matrícula:\enspace\hrulefill}

\noindent{}\makebox[0.6\linewidth]{Aluno(a):\enspace\hrulefill}\makebox[0.4\textwidth]{  Matrícula:\enspace\hrulefill}
\end{fullwidth}

\vspace{5mm}

%%%%%%%%%%%%%%%%%%%%%%%%%%%%%%%%%%%%%%%%%%%%%%%%%%%%%%%%%%%%%%%%%%%%%%%%%%%%%%%
\section{Questionário}
%%%%%%%%%%%%%%%%%%%%%%%%%%%%%%%%%%%%%%%%%%%%%%%%%%%%%%%%%%%%%%%%%%%%%%%%%%%%%%%

\begin{question}[type={exam}]{2}
Preencha as colunas de dados experimentais das tabelas com o número adequado de algarismos significativos e unidades.
\end{question}

\begin{question}[type={exam}]{3}
Ligue os pontos equipotenciais obtidos de uma maneira suave. Considerando que as superfícies equipotenciais são sempre perpendiculares às linhas de campo elétrico, desenhe as linhas de campo, de uma maneira estimada.
\end{question}

\begin{question}[type={exam}]{7}
Faça um gráfico do potencial elétrico em função da posição $x$ para os dados contidos na Tabela~\ref{Tab:ValoresPotencialEletricoEixoX}.
\end{question}

\begin{question}[type={exam}]{2}
Estime o campo elétrico ao longo do eixo $x$ através de
\begin{equation}
    E = -\frac{\partial V}{\partial x} \approx \frac{\Delta V}{\Delta x}.
\end{equation}
%
Faça um gráfico de $E(x)$ para o resultado obtido.
\end{question}

\vfill
%%%%%%%%%%%%%%%%%%%%%%%%%%%%%%%%%%%%%%%%%%%%%%%%%%%%%%%%%%%%%%%%%%%%%%%%%%%%%%%
\pagebreak
\section{Tabelas}
%%%%%%%%%%%%%%%%%%%%%%%%%%%%%%%%%%%%%%%%%%%%%%%%%%%%%%%%%%%%%%%%%%%%%%%%%%%%%%%


\begin{table}
\label{Tab:ValoresPotencialEletricoEixoX}
	\begin{center}
		\begin{tabular}{cp{25mm}p{25mm}c}
		\toprule
		&\multicolumn{2}{l}{\textbf{Dados para o potencial.}} \\
		\cmidrule{2-3}
		& $x$ & $V$ \\
		\cmidrule{2-3}
		& \cellcolor[gray]{0.89} & \cellcolor[gray]{0.92} & \\
		& \cellcolor[gray]{0.95} & \cellcolor[gray]{0.97} \\
		& \cellcolor[gray]{0.89} & \cellcolor[gray]{0.92} \\
		& \cellcolor[gray]{0.95} & \cellcolor[gray]{0.97} \\
		& \cellcolor[gray]{0.89} & \cellcolor[gray]{0.92} \\
		& \cellcolor[gray]{0.95} & \cellcolor[gray]{0.97} \\
		& \cellcolor[gray]{0.89} & \cellcolor[gray]{0.92} \\
		& \cellcolor[gray]{0.95} & \cellcolor[gray]{0.97} \\
		& \cellcolor[gray]{0.89} & \cellcolor[gray]{0.92} \\
		& \cellcolor[gray]{0.95} & \cellcolor[gray]{0.97} \\
		& \cellcolor[gray]{0.89} & \cellcolor[gray]{0.92} \\
		& \cellcolor[gray]{0.95} & \cellcolor[gray]{0.97} \\
		& \cellcolor[gray]{0.89} & \cellcolor[gray]{0.92} \\
		& \cellcolor[gray]{0.95} & \cellcolor[gray]{0.97} \\
		& \cellcolor[gray]{0.89} & \cellcolor[gray]{0.92} \\
		& \cellcolor[gray]{0.95} & \cellcolor[gray]{0.97} \\
		\cmidrule{2-3}
		\bottomrule
		\end{tabular}
	\end{center}
	\caption{Dados para as leituras de tensão em função da posição.}
\end{table}
