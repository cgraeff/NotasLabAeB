%%%%%%%%%%%%%%%%%%%%%%%%%%%%%%%%%%%%%%%%%%%%%%%%%%%%%%%%%%%%%%%%%%%%%%%%%%%%%%%
\chapter{Capacitores de placas paralelas} % Sem "Experiência 01" ou qualquer outro número
\label{Chap:CapacPlacasPar}        % para poder trocar a ordem com facilidade
%%%%%%%%%%%%%%%%%%%%%%%%%%%%%%%%%%%%%%%%%%%%%%%%%%%%%%%%%%%%%%%%%%%%%%%%%%%%%%%

\begin{fullwidth}\it
	Vamos realizar um experimento usando um capacitor de placas paralelas cuja distância entre as placas é regulável. Isso nos permitirá observar a relação entre a capacitância e a distância, verificanco a expressão teórica para a relação entre essas duas grandezas, bem como empregar um outro meio dielétrico entre as placas que não seja o ar, permitindo que determinemos sua constante dielétrica.
\end{fullwidth}

%%%%%%%%%%%%%%%%%%%%%%%%%%%%%%%%%%%%%%%%%%%%%%%%%%%%%%%%%%%%%%%%%%%%%%%%%%%%%%%
\section{Capacitância}
%%%%%%%%%%%%%%%%%%%%%%%%%%%%%%%%%%%%%%%%%%%%%%%%%%%%%%%%%%%%%%%%%%%%%%%%%%%%%%%

Descrever o que é capacitância.

\subsection{Capacitores de placas paralelas}

Calcular a capacitância de um capacitor de placas paralelas.

\paragraph{Dielétricos}

Descrever e calcular o que acontece quando colocamos um dielétrico.

%%%%%%%%%%%%%%%%%%%%%%%%%%%%%%%%%%%%%%%%%%%%%%%%%%%%%%%%%%%%%%%%%%%%%%%%%%%%%%%
\section{Experimento}
%%%%%%%%%%%%%%%%%%%%%%%%%%%%%%%%%%%%%%%%%%%%%%%%%%%%%%%%%%%%%%%%%%%%%%%%%%%%%%%

%%%%%%%%%%%%%%%%%%%%%%
\subsection{Objetivos}
%%%%%%%%%%%%%%%%%%%%%%

\begin{itemize}
	\item Observar a relação entre a capacitância e a distância entre as placas condutoras;
	\item Determinar a constante dielétrica do material inserido entre as placas;
\end{itemize}

%%%%%%%%%%%%%%%%%%%%%%%%%%%%%%%%%%%%%%%%%%%%%%%%%%%%%%%%%%%%%%%%%%%%%%%%%%%%%%%
\section{Material Necessário}
%%%%%%%%%%%%%%%%%%%%%%%%%%%%%%%%%%%%%%%%%%%%%%%%%%%%%%%%%%%%%%%%%%%%%%%%%%%%%%%

\begin{itemize}
	\item Capacitor de placas paralelas móveis;
	\item Multímetro e cabos de conexão;
	\item Lâmina dielétricas;
	\item Micrômetro e paquímetro.
\end{itemize}

%%%%%%%%%%%%%%%%%%%%%%%%%%%%%%%%%%%%%%%%%%%%%%%%%%%%%%%%%%%%%%%%%%%%%%%%%%%%%%%
\section{Procedimento Experimental}
%%%%%%%%%%%%%%%%%%%%%%%%%%%%%%%%%%%%%%%%%%%%%%%%%%%%%%%%%%%%%%%%%%%%%%%%%%%%%%%

%%%%%%%%%%%%%%%%%%%%%
%\subsection{Parte A} % Se necessário
%%%%%%%%%%%%%%%%%%%%%
\begin{enumerate}
	\item Utilizando um paquímetro, verifique as dimensões das placas do capacitor;
	\item Utilizando um micrômetro, meça a espessura de uma lâmina dielétrica e a coloque entre as placas do capacitor de formaque que não reste ar entre as placas e a lâmina. Anote o valor obtido na Tabela~\ref{Tab:ValoresCapacitancia};\label{Item:CapPlacasPar:MedidaMicrometroLamina}
	\item Afira o valor de capacitância e o anote na Tabela~\ref{Tab:ValoresCapacitancia};
	\item Tomando cuidado para não mover as placas do capacitor, remova a lâmina dielétrica e afira o novo valor de capacitância e o anote na Tabela~\ref{Tab:ValoresCapacitancia}
	\item Repita os passos acima a partir do item \ref{Item:CapPlacasPar:MedidaMicrometroLamina}, anotando o valor de distância total entre as placas e as capacitâncias com e sem dielétrico na Tabela~\ref{Tab:ValoresCapacitancia}.
\end{enumerate}

%%%%%%%%%%%%%%%%%%%%%%%%%%%%%%%%%%%%%%%%%%%%%%%%%%%%%%%%%%%%%%%%%%%%%%%%%%%%%%%
%%%%%%%%%%%%%%%%%%%%%%%%%%%%%%%%%%%%%%%%%%%%%%%%%%%%%%%%%%%%%%%%%%%%%%%%%%%%%%%
%%%%%%%%%%%%%%%%%%%%%%%%%%%%%%%%%%%%%%%%%%%%%%%%%%%%%%%%%%%%%%%%%%%%%%%%%%%%%%%
%%%%%%%%%%%%%%%%%%%%%%%%%%%%%%%%%%%%%%%%%%%%%%%%%%%%%%%%%%%%%%%%%%%%%%%%%%%%%%%
\cleardoublepage

\noindent{}{\huge\textit{Capacitores de placas paralelas}}

\vspace{15mm}

\begin{fullwidth}
\noindent{}\makebox[0.6\linewidth]{Turma:\enspace\hrulefill}\makebox[0.4\textwidth]{  Data:\enspace\hrulefill}
\vspace{5mm}

\noindent{}\makebox[0.6\linewidth]{Aluno(a):\enspace\hrulefill}\makebox[0.4\textwidth]{  Matrícula:\enspace\hrulefill}

\noindent{}\makebox[0.6\linewidth]{Aluno(a):\enspace\hrulefill}\makebox[0.4\textwidth]{  Matrícula:\enspace\hrulefill}

\noindent{}\makebox[0.6\linewidth]{Aluno(a):\enspace\hrulefill}\makebox[0.4\textwidth]{  Matrícula:\enspace\hrulefill}

\noindent{}\makebox[0.6\linewidth]{Aluno(a):\enspace\hrulefill}\makebox[0.4\textwidth]{  Matrícula:\enspace\hrulefill}

\noindent{}\makebox[0.6\linewidth]{Aluno(a):\enspace\hrulefill}\makebox[0.4\textwidth]{  Matrícula:\enspace\hrulefill}
\end{fullwidth}

\vspace{5mm}

%%%%%%%%%%%%%%%%%%%%%%%%%%%%%%%%%%%%%%%%%%%%%%%%%%%%%%%%%%%%%%%%%%%%%%%%%%%%%%%
\section{Questionário}
%%%%%%%%%%%%%%%%%%%%%%%%%%%%%%%%%%%%%%%%%%%%%%%%%%%%%%%%%%%%%%%%%%%%%%%%%%%%%%%

\begin{question}[type={exam}]{1}
Apresente os resultados de maneira clara e organizada. Mostre os cálculos requisitados de maneira clara e sucinta, evidenciando o raciocínio desenvolvido.
\end{question}

\begin{question}[type={exam}]{1}
Preencha as colunas de dados experimentais das tabelas com o número adequado de algarismos significativos e unidades.
\end{question}

\begin{question}[type={exam}]{2}
\begin{enumerate}[label=\roman*.]
    \item Faça um gráfico do valor de capacitância em função da distância de separação entre as placas.
    \item Faça um ajuste de uma curva do tipo $y = A + B \cdot x^{-1}$ para os dados obtidos.\footnote{Nem todo software é capaz de realizar esse tipo de ajuste, porém as calculadoras geralmente têm suporte a esse procedimento. Alternativamente, vocês podem fazer uma linearização dos dados e realizar uma regressão linear.} 
    \item Determine o valor da permissividade do vácuo $\epsilon_0$ através dos valores dos  coeficientes $A$ e $B$ calculados. 
\end{enumerate}
\end{question}

\begin{question}[type={exam}]{2}
Lorem ipsum dolor sit amet, consectetuer adi-
piscing elit. Ut purus elit, vestibulum ut, placerat ac, adipiscing vitae,
felis. Curabitur dictum gravida mauris. Nam arcu libero, nonummy
eget, consectetuer id, vulputate a, magna. Donec vehicula augue
eu neque. Pellentesque habitant morbi tristique senectus et netus
et malesuada fames ac turpis egestas. Mauris ut leo. Cras viverra
metus rhoncus sem. Nulla et lectus vestibulum urna fringilla ultrices.
\end{question}



\begin{question}[type={exam}]{2}
Lorem ipsum dolor sit amet, consectetuer adi-
piscing elit. Ut purus elit, vestibulum ut, placerat ac, adipiscing vitae,
felis. Curabitur dictum gravida mauris. Nam arcu libero, nonummy
eget, consectetuer id, vulputate a, magna. Donec vehicula augue
eu neque. Pellentesque habitant morbi tristique senectus et netus
et malesuada fames ac turpis egestas. Mauris ut leo. Cras viverra
metus rhoncus sem. Nulla et lectus vestibulum urna fringilla ultrices.
\end{question}
\vfill
%%%%%%%%%%%%%%%%%%%%%%%%%%%%%%%%%%%%%%%%%%%%%%%%%%%%%%%%%%%%%%%%%%%%%%%%%%%%%%%
\pagebreak
\section{Tabelas}
%%%%%%%%%%%%%%%%%%%%%%%%%%%%%%%%%%%%%%%%%%%%%%%%%%%%%%%%%%%%%%%%%%%%%%%%%%%%%%%

\begin{table*}[!h]
\centering
\begin{tabular}{lp{25mm}p{25mm}p{25mm}l}
\toprule
	& \multicolumn{2}{l}{\textbf{Dimensões}} \\
	\cmidrule{2-3}
	& Diâmetro: \cellcolor[gray]{0.89} & \cellcolor[gray]{0.92} \\
	& Espessura: \cellcolor[gray]{0.95} & \cellcolor[gray]{0.97} \\
	\cmidrule{2-3}
\\
	& \multicolumn{4}{l}{\textbf{Dados experimentais}} \\
	\cmidrule{2-4}
	& $C$ & $C_0$ & $d$ & \\
	\cmidrule{2-4}
	& \cellcolor[gray]{0.89} & \cellcolor[gray]{0.92} & \cellcolor[gray]{0.89} & \\
	& \cellcolor[gray]{0.95} & \cellcolor[gray]{0.97} & \cellcolor[gray]{0.95} & \\
	& \cellcolor[gray]{0.89} & \cellcolor[gray]{0.92} & \cellcolor[gray]{0.89} & \\
	& \cellcolor[gray]{0.95} & \cellcolor[gray]{0.97} & \cellcolor[gray]{0.95} & \\
	& \cellcolor[gray]{0.89} & \cellcolor[gray]{0.92} & \cellcolor[gray]{0.89} & \\
	& \cellcolor[gray]{0.95} & \cellcolor[gray]{0.97} & \cellcolor[gray]{0.95} & \\
	& \cellcolor[gray]{0.89} & \cellcolor[gray]{0.92} & \cellcolor[gray]{0.89} & \\
	& \cellcolor[gray]{0.95} & \cellcolor[gray]{0.97} & \cellcolor[gray]{0.95} & \\
	& \cellcolor[gray]{0.89} & \cellcolor[gray]{0.92} & \cellcolor[gray]{0.89} & \\
	& \cellcolor[gray]{0.95} & \cellcolor[gray]{0.97} & \cellcolor[gray]{0.95} & \\
	\cmidrule{2-4}
\bottomrule
\end{tabular}
\caption[][5mm]{Dados para o movimento efetuado pela esfera sob ação da gravidade e do arrasto.}
\label{Tab:ValoresCapacitancia}
\end{table*}

