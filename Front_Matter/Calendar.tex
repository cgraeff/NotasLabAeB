\thispagestyle{plain}
\begin{fullwidth}
\begin{center}
{\noindent\LARGE\textsc{Cronograma}} \\
\end{center}
\end{fullwidth}

\vspace{1cm}
\begin{fullwidth}
\it
As aulas seguirão o planejamento abaixo. No calendário ao lado, estão circuladas as datas das provas.
\end{fullwidth}

%%%
% Laboratório de Física A
%%%

\begin{marginfigure}[4.5cm]
    \centering
    Março\\
    \begin{tikzpicture}
        \calendar (mycal)
        [
            dates=2025-03-01 to 2025-03-last,
            week list,
            day headings=gray,
            day letter headings
        ]
        if (Saturday, Sunday)
            [gray]
        if (between=2025-03-01 and 2025-03-24)
            [gray]
        ;
        \node [draw, rounded corners, minimum size = 4mm, inner sep=0pt,outer sep=0pt,thick] at (mycal-2025-03-31) {};
    \end{tikzpicture}
\end{marginfigure} %
%
\begin{marginfigure}
    \centering
    Abril\\
    \begin{tikzpicture}
        \calendar (mycal)
        [
            dates=2025-04-01 to 2025-04-last,
            week list,
            day headings=gray,
            day letter headings
        ]
        if (Saturday, Sunday)
            [gray]
        if (equals=2025-04-18,equals=2025-04-21)
            [gray]
        ;
    \end{tikzpicture}
\end{marginfigure} %
%
\begin{marginfigure}
    \centering
    Maio\\
    \begin{tikzpicture}
        \calendar (mycal)
        [
            dates=2025-05-01 to 2025-05-last,
            week list,
            day headings=gray,
            day letter headings
        ]
        if (Saturday, Sunday)
            [gray]
        if (between=2025-05-01 and 2025-05-03)
            [gray]
        ;
        \draw (mycal-2025-05-12) circle (6pt);
    \end{tikzpicture}
\end{marginfigure} %
%
\begin{marginfigure}
    \centering
    Junho\\
    \begin{tikzpicture}
        \calendar (mycal)
        [
            dates=2025-06-01 to 2025-06-last,
            week list,
            day headings=gray,
            day letter headings
        ]
        if (Saturday, Sunday)
            [gray]
        if (between=2025-06-19 and 2025-06-21)
            [gray]
        ;
        \draw (mycal-2025-06-23) circle (6pt);
        \draw[dashed] (mycal-2025-06-30) circle (6pt);
    \end{tikzpicture}
\end{marginfigure} %
%
\begin{marginfigure}
    \centering
    Julho\\
    \begin{tikzpicture}
        \calendar (mycal)
        [
            dates=2025-07-01 to 2025-07-last,
            week list,
            day headings=gray,
            day letter headings
        ]
        if (Saturday, Sunday)
            [gray]
        if (between=2025-07-12 and 2025-07-last)
            [gray]
        ;
        \node [draw, rounded corners, minimum size = 4mm, inner sep=0pt,outer sep=0pt,thick] at (mycal-2025-07-11) {};
    \end{tikzpicture}
\end{marginfigure} %
%
\vspace{1cm}
\begin{center}
\Large\textsc{Física Experimental A}
\end{center}

\begin{center}
\begin{longtable}{ccp{70mm}}
\toprule
Aula & Data & Conteúdo \\
\midrule
\endhead
\bottomrule
\endfoot
  1 & 31/03 & Apresentação da disciplina. \\
  2 & 07/04 & Exp. 1, Medidas e algarismos significativos. \\
  3 & 14/04 & Exp. 2, Movimento retilíneo uniforme (MRU) e uniformemente variado (MRUV). \\
  4 & 28/04 & Exp. 3, Leis de Newton. \\
  5 & 05/05 & Exp. 4, Lei de Hooke. \\
  6 & 12/05 & \textbf{Prova 1:} Algarismos significativos e elaboração de gráficos. \\
  7 & 19/05 & Exp. 5, Arrasto. \\
  8 & 26/05 & Exp. 6, Atrito. \\
  9 & 02/06 & Exp. 7, Roda de Maxwell. \\
 10 & 09/06 & Exp. 8, Oscilações. \\
 11 & 16/06 & Exp. 9, Ondas estacionárias. \\
 12 & 23/06 & \textbf{Prova 2:} Algarismos significativos, regressão linear, e linearização. \\
 13 & 30/06 & Recuperação.\\
 14 & 07/06 & Entrega das notas finais.
\end{longtable}
\end{center}

%\clearpage
% Calendário de outra turma, se houver

\cleardoublepage
