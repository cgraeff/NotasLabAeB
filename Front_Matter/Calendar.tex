\thispagestyle{plain}
\begin{fullwidth}
\begin{center}
{\noindent\LARGE\textsc{Cronograma}} \\
\end{center}
\end{fullwidth}

\vspace{1cm}
\begin{fullwidth}
\it
As aulas seguirão o planejamento abaixo. No calendário ao lado, estão circuladas as datas das provas.
\end{fullwidth}

%%%
% Laboratório de Física 1
%%%

\begin{marginfigure}[4cm]
    \centering
    Março\\
    \begin{tikzpicture}
        \calendar (mycal)
        [
            dates=2023-03-01 to 2023-03-last,
            week list,
            day headings=gray,
            day letter headings
        ]
        if (Saturday,Sunday)
            [gray]
        ;
    \end{tikzpicture}
\end{marginfigure} %
%
\begin{marginfigure}
    \centering
    Abril\\
    \begin{tikzpicture}
        \calendar (mycal)
        [
            dates=2023-04-01 to 2023-04-last,
            week list,
            day headings=gray,
            day letter headings
        ]
        if (Saturday,Sunday)
            [gray]
        if (equals=2023-04-07, equals=2023-04-08, equals=2023-04-21, equals=2023-04-22)
            [gray]
        ;
    \end{tikzpicture}
\end{marginfigure} %
%
\begin{marginfigure}
    \centering
    Maio\\
    \begin{tikzpicture}
        \calendar (mycal)
        [
            dates=2023-05-01 to 2023-05-last,
            week list,
            day headings=gray,
            day letter headings
        ]
        if (Saturday,Sunday)
            [gray]
        if (equals= 2023-05-01)
            [gray]
        ;
    \end{tikzpicture}
\end{marginfigure} %
%
\begin{marginfigure}
    \centering
    Junho\\
    \begin{tikzpicture}
        \calendar (mycal)
        [
            dates=2023-06-01 to 2023-06-last,
            week list,
            day headings=gray,
            day letter headings
        ]
        if (Saturday,Sunday)
            [gray]
        if (equals= 2023-06-08, equals= 2023-06-09, equals= 2023-06-10, equals= 2023-06-29, equals= 2023-06-30)
            [gray]
        ;
        \draw[densely dotted] (mycal-2023-06-20) circle (6pt);
    \end{tikzpicture}
\end{marginfigure} %
%
\begin{marginfigure}
    \centering
    Julho\\
    \begin{tikzpicture}
        \calendar (mycal)
        [
            dates=2023-07-01 to 2023-07-last,
            week list,
            day headings=gray,
            day letter headings
        ]
        if (Saturday,Sunday)
            [gray]
    	if (equals=2023-07-01)
    	    [gray]
        if (at most=2023-07-08)
            {}
        else
            [gray]
        ;
    \end{tikzpicture}
\end{marginfigure}
\vspace{1cm}
\begin{center}
\Large\textsc{Laboratório de Física 1}
\end{center}

A entrega das respostas dos questionários referentes a cada experimento serão realizadas através do Moodle no endereço \url{https://moodle.utfpr.edu.br/course/view.php?id=24890}. A chave para a inscrição no curso é \verb|pQhl&cMPjXfqI7/DKE_i|.
\begin{center}
\begin{longtable}{ccp{70mm}}
\toprule
Aula & Data & Conteúdo \\
\midrule
\endhead
\bottomrule
\endfoot
 1 & 07/03 & Apresentação da disciplina. \\
 2 & 14/03 & Turma A: Exp. 1, Medidas. \\
 3 & 21/03 & Turma B: Exp. 1, Medidas. \\
 4 & 28/03 & Turma A: Exp. 2, MRU e MRUV. \\ 
 5 & 04/04 & Turma B: Exp. 2, MRU e MRUV. \\
 6 & 11/04 & Turma A: Exp. 3, Lei de Hooke. \\
 7 & 18/04 & Turma B: Exp. 3, Lei de Hooke. \\
 8 & 25/04 & Turma A: Exp. 4, Leis de Newton. \\
 9 & 02/05 & Turma B: Exp. 4, Leis de Newton. \\
10 & 09/05 & Turma A: Exp. 5, Arrasto. \\
11 & 16/05 & Turma B: Exp. 5, Arrasto. \\
12 & 23/05 & Turma A: Exp. 6, Atrito. \\
13 & 30/05 & Turma B: Exp. 6, Atrito. \\
14 & 06/06 & Turma A: Exp. 7, Energia Mecânica. \\
15 & 13/06 & Turma B: Exp. 7, Energia Mecância. \\
16 & 20/06 & \textbf{\textit{Prova de laboratório:}} Turmas A e B. \\
17 & 27/06 & Entrega das notas da prova de laboratório. \\
18 & 04/07 & Entrega das notas finais de laboratório.
\end{longtable}
\end{center}

\clearpage

%%%
% Laboratório de Física 2
%%%

\begin{marginfigure}[4cm]
    \centering
    Março\\
    \begin{tikzpicture}
        \calendar (mycal)
        [
            dates=2023-03-01 to 2023-03-last,
            week list,
            day headings=gray,
            day letter headings
        ]
        if (Saturday,Sunday)
            [gray]
        ;
    \end{tikzpicture}
\end{marginfigure} %
%
\begin{marginfigure}[2.5mm]
    \centering
    Abril\\
    \begin{tikzpicture}
        \calendar (mycal)
        [
            dates=2023-04-01 to 2023-04-last,
            week list,
            day headings=gray,
            day letter headings
        ]
        if (Saturday,Sunday)
            [gray]
        if (equals=2023-04-07, equals=2023-04-08, equals=2023-04-21, equals=2023-04-22)
            [gray]
        ;
    \end{tikzpicture}
\end{marginfigure} %
%
\begin{marginfigure}
    \centering
    Maio\\
    \begin{tikzpicture}
        \calendar (mycal)
        [
            dates=2023-05-01 to 2023-05-last,
            week list,
            day headings=gray,
            day letter headings
        ]
        if (Saturday,Sunday)
            [gray]
        if (equals= 2023-05-01)
            [gray]
        ;
    \end{tikzpicture}
\end{marginfigure} %
%
\begin{marginfigure}
    \centering
    Junho\\
    \begin{tikzpicture}
        \calendar (mycal)
        [
            dates=2023-06-01 to 2023-06-last,
            week list,
            day headings=gray,
            day letter headings
        ]
        if (Saturday,Sunday)
            [gray]
        if (equals= 2023-06-08, equals= 2023-06-09, equals= 2023-06-10, equals= 2023-06-29, equals= 2023-06-30)
            [gray]
        ;
        \draw[densely dotted] (mycal-2023-06-22) circle (6pt);
    \end{tikzpicture}
\end{marginfigure} %
%
\begin{marginfigure}
    \centering
    Julho\\
    \begin{tikzpicture}
        \calendar (mycal)
        [
            dates=2023-07-01 to 2023-07-last,
            week list,
            day headings=gray,
            day letter headings
        ]
        if (Saturday,Sunday)
            [gray]
    	if (equals=2023-07-01)
    	    [gray]
        if (at most=2023-07-08)
            {}
        else
            [gray]
        ;
    \end{tikzpicture}
\end{marginfigure}
\vspace{1cm}
\begin{center}
\Large\textsc{Laboratório de Física 2}
\end{center}

A entrega das respostas dos questionários referentes a cada experimento serão realizadas através do Moodle no endereço \url{https://moodle.utfpr.edu.br/course/view.php?id=24891}. A chave para inscrição no curso é \verb|WZg*XKRc2w;!:zSkJ#j0|.
\begin{center}
\begin{longtable}{ccp{70mm}}
\toprule
Aula & Data & Conteúdo \\
\midrule
\endhead
\bottomrule
\endfoot
 1 & 02/03 & Apresentação da disciplina. \\
 2 & 09/03 & Turma A: Exp. 1, Elasticidade. \\
 3 & 16/03 & Turma B: Exp. 1, Elasticidade \\
 4 & 23/03 & Turma A: Exp. 2, Empuxo. \\ 
 5 & 30/03 & Turma B: Exp. 2, Empuxo. \\
 6 & 06/04 & Turma A: Exp. 3, Oscilações. \\
 7 & 13/04 & Turma B: Exp. 3, Oscilações. \\
 8 & 20/04 & Turma A: Exp. 4, Ondas Estacionárias. \\
 9 & 27/04 & Turma B: Exp. 4, Ondas Estacionárias. \\
10 & 04/05 & Turma A: Exp. 5, Dilatação linear e lei de resfriamento. \\
11 & 11/05 & Turma B: Exp. 5, Dilatação linear e lei de resfriamento. \\
12 & 18/05 & Turma A: Exp. 6, Calor específico. \\
13 & 25/05 & Turma B: Exp. 6, Calor específico. \\
14 & 01/06 & Turma A: Exp. 7, Zero absoluto. \\
-- & 08/06 & \emph{Feriado.} \\
15 & 15/06 & Turma B: Exp. 7, Zero absoluto. \\
16 & 22/06 & \textbf{\textit{Prova de laboratório:}} Turmas A e B. \\
-- & 29/06 & \emph{Feriado.} \\
17 & 06/07 & Entrega das notas finais de laboratório. \\
\end{longtable}
\end{center}

\cleardoublepage
