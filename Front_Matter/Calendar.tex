\thispagestyle{plain}
\begin{fullwidth}
\begin{center}
{\noindent\LARGE\textsc{Cronograma}} \\
\end{center}
\end{fullwidth}

\vspace{1cm}
\begin{fullwidth}
\it
As aulas seguirão o planejamento abaixo. No calendário ao lado, estão circuladas as datas das provas.
\end{fullwidth}

%%%
% Laboratório de Física A - 3MC 2024-2
%%%

\begin{marginfigure}[4.5cm]
    \centering
    Outubro\\
    \begin{tikzpicture}
        \calendar (mycal)
        [
            dates=2024-10-01 to 2024-10-last,
            week list,
            day headings=gray,
            day letter headings
        ]
        if (Saturday, Sunday)
            [gray]
        if (equals=2024-10-28)
            [gray]
        ;
        \node [draw, rounded corners, minimum size = 4mm, inner sep=0pt,outer sep=0pt,thick] at (mycal-2024-10-01) {};
    \end{tikzpicture}
\end{marginfigure} %
%
\begin{marginfigure}
    \centering
    Novembro\\
    \begin{tikzpicture}
        \calendar (mycal)
        [
            dates=2024-11-01 to 2024-11-last,
            week list,
            day headings=gray,
            day letter headings
        ]
        if (Saturday, Sunday)
            [gray]
        if (equals=2024-11-15,equals=2024-11-20)
            [gray]
        ;
    \end{tikzpicture}
\end{marginfigure} %
%
\begin{marginfigure}
    \centering
    Dezembro\\
    \begin{tikzpicture}
        \calendar (mycal)
        [
            dates=2024-12-01 to 2024-12-last,
            week list,
            day headings=gray,
            day letter headings
        ]
        if (Saturday, Sunday)
            [gray]
        if (between=2024-12-23 and 2024-12-31)
            [gray]
        ;
    \end{tikzpicture}
\end{marginfigure} %
%
\begin{marginfigure}
    \centering
    Janeiro\\
    \begin{tikzpicture}
        \calendar (mycal)
        [
            dates=2025-01-01 to 2025-01-last,
            week list,
            day headings=gray,
            day letter headings
        ]
        if (Saturday, Sunday)
            [gray]
        if (between=2025-01-01 and 2025-01-last)
            [gray]
        ;
    \end{tikzpicture}
\end{marginfigure} %
%
\begin{marginfigure}
    \centering
    Fevereiro\\
    \begin{tikzpicture}
        \calendar (mycal)
        [
            dates=2025-02-01 to 2025-02-last,
            week list,
            day headings=gray,
            day letter headings
        ]
        if (Saturday, Sunday)
            [gray]
        ;
        \draw (mycal-2025-02-10) circle (6pt);
        \draw[dashed] (mycal-2025-02-17) circle (6pt);
        \node [draw, rounded corners, minimum size = 4mm, inner sep=0pt,outer sep=0pt,thick] at (mycal-2025-02-28) {};
    \end{tikzpicture}
\end{marginfigure} %
%
\begin{marginfigure}
    \centering
    Março\\
    \begin{tikzpicture}
        \calendar (mycal)
        [
            dates=2025-03-01 to 2025-03-last,
            week list,
            day headings=gray,
            day letter headings
        ]
        if (Saturday, Sunday)
            [gray]
        if (between=2025-03-03 and 2025-03-05)
            [gray]
        if (between=2025-03-08 and 2025-03-last)
            [gray]
        ;
    \end{tikzpicture}
\end{marginfigure}
%
%
%
\vspace{1cm}
\begin{center}
\Large\textsc{Física Experimental A - Turma FS42NB-3MC}
\end{center}

\begin{center}
\begin{longtable}{ccp{70mm}}
\toprule
Aula & Data & Conteúdo \\
\midrule
\endhead
\bottomrule
\endfoot
  1 & 07/10 & Apresentação da disciplina. \\
  2 & 14/10 & Exp. 1, Medidas e algarismos significativos. \\
  3 & 21/10 & \emph{Semana acadêmica de Eng. Mecânica.}  \\
--- & 28/10 & --- \\
  4 & 04/11 & Exp. 2, Movimento retilíneo uniforme (MRU) e uniformemente variado (MRUV). \\
  5 & 11/11 & Exp. 3, Lei de Hooke. \\
  6 & 18/11 & Exp. 4, Leis de Newton. \\
  7 & 25/11 & Exp. 5, Arrasto. \\
  8 & 02/12 & Exp. 6, Roda de Maxwell. \\
  9 & 09/12 & Exp. 7, Oscilações. \\
 10 & 16/12 & Exp. 8, Ondas estacionárias. \\
--- & 23/12 & --- \\
--- & 30/12 & --- \\
--- & 06/01 & --- \\
--- & 13/01 & --- \\
--- & 20/01 & --- \\ 
--- & 27/01 & --- \\
 11 & 03/02 & Dúvidas. \\
 12 & 10/02 & Prova.\\
 13 & 17/02 & Recuperação.\\
 14 & 24/02 & Entrega das notas finais.
\end{longtable}
\end{center}

\clearpage
% Calendário de outra turma, se houver

\begin{marginfigure}[4.5cm]
    \centering
    Outubro\\
    \begin{tikzpicture}
        \calendar (mycal)
        [
            dates=2024-10-01 to 2024-10-last,
            week list,
            day headings=gray,
            day letter headings
        ]
        if (Saturday, Sunday)
            [gray]
        if (equals=2024-10-28)
            [gray]
        ;
        \node [draw, rounded corners, minimum size = 4mm, inner sep=0pt,outer sep=0pt,thick] at (mycal-2024-10-01) {};
    \end{tikzpicture}
\end{marginfigure} %
%
\begin{marginfigure}
    \centering
    Novembro\\
    \begin{tikzpicture}
        \calendar (mycal)
        [
            dates=2024-11-01 to 2024-11-last,
            week list,
            day headings=gray,
            day letter headings
        ]
        if (Saturday, Sunday)
            [gray]
        if (equals=2024-11-15,equals=2024-11-20)
            [gray]
        ;
    \end{tikzpicture}
\end{marginfigure} %
%
\begin{marginfigure}
    \centering
    Dezembro\\
    \begin{tikzpicture}
        \calendar (mycal)
        [
            dates=2024-12-01 to 2024-12-last,
            week list,
            day headings=gray,
            day letter headings
        ]
        if (Saturday, Sunday)
            [gray]
        if (between=2024-12-23 and 2024-12-31)
            [gray]
        ;
    \end{tikzpicture}
\end{marginfigure} %
%
\begin{marginfigure}
    \centering
    Janeiro\\
    \begin{tikzpicture}
        \calendar (mycal)
        [
            dates=2025-01-01 to 2025-01-last,
            week list,
            day headings=gray,
            day letter headings
        ]
        if (Saturday, Sunday)
            [gray]
        if (between=2025-01-01 and 2025-01-last)
            [gray]
        ;
    \end{tikzpicture}
\end{marginfigure} %
%
\begin{marginfigure}
    \centering
    Fevereiro\\
    \begin{tikzpicture}
        \calendar (mycal)
        [
            dates=2025-02-01 to 2025-02-last,
            week list,
            day headings=gray,
            day letter headings
        ]
        if (Saturday, Sunday)
            [gray]
        ;
        \draw (mycal-2025-02-10) circle (6pt);
        \draw[dashed] (mycal-2025-02-17) circle (6pt);
        \node [draw, rounded corners, minimum size = 4mm, inner sep=0pt,outer sep=0pt,thick] at (mycal-2025-02-28) {};
    \end{tikzpicture}
\end{marginfigure} %
%
\begin{marginfigure}
    \centering
    Março\\
    \begin{tikzpicture}
        \calendar (mycal)
        [
            dates=2025-03-01 to 2025-03-last,
            week list,
            day headings=gray,
            day letter headings
        ]
        if (Saturday, Sunday)
            [gray]
        if (between=2025-03-03 and 2025-03-05)
            [gray]
        if (between=2025-03-08 and 2025-03-last)
            [gray]
        ;
    \end{tikzpicture}
\end{marginfigure}
%
%
%
\vspace{1cm}
\begin{center}
\Large\textsc{Física Experimental A - Turma FS42NB-4CV}
\end{center}

\begin{center}
\begin{longtable}{ccp{70mm}}
\toprule
Aula & Data & Conteúdo \\
\midrule
\endhead
\bottomrule
\endfoot
  1 & 07/10 & Apresentação da disciplina. \\
  2 & 14/10 & Exp. 1, Medidas e algarismos significativos. \\
  3 & 21/10 & \emph{Semana Acadëmica de Eng. Civil} \\
--- & 28/10 & --- \\
  4 & 04/11 & Exp. 2, Movimento retilíneo uniforme (MRU) e uniformemente variado (MRUV). \\
  5 & 11/11 & Exp. 3, Lei de Hooke. \\
  6 & 18/11 & Exp. 4, Leis de Newton. \\
  7 & 25/11 & Exp. 5, Arrasto. \\
  8 & 02/12 & Exp. 6, Roda de Maxwell. \\
  9 & 09/12 & Exp. 7, Oscilações. \\
 10 & 16/12 & Exp. 8, Ondas estacionárias. \\
--- & 23/12 & --- \\
--- & 30/12 & --- \\
--- & 06/01 & --- \\
--- & 13/01 & --- \\
--- & 20/01 & --- \\ 
--- & 27/01 & --- \\
 11 & 03/02 & Dúvidas. \\
 12 & 10/02 & Prova.\\
 13 & 17/02 & Recuperação.\\
 14 & 24/02 & Entrega das notas finais.
\end{longtable}
\end{center}
\cleardoublepage
