\chapter{Regressão Linear}
\label{Chap:RegressoLinear}

\begin{fullwidth}
{\it
Apesar de podermos verificar previsões teóricas pela simples comparação com os dados experimentais obtidos (assumindo que eles sejam confiáveis), devido à dispersão dos dados não temos um bom método para extrair informações a partir das medidas. Veremos que para isso precisamos determinar a linha de tendência dos dados. A partir das informações para essa linha, conseguiremos extrair informações acerca de parâmetros físicos do sistema.
}
\end{fullwidth}

%%%%%%%%%%%%%%%%%%%%%%%%%%%%%%%%%%%%%
\section{Linhas de Tendência}
%%%%%%%%%%%%%%%%%%%%%%%%%%%%%%%%%%%%%

Quando realizamos um experimento, procuramos relacionar uma variável dependente a uma variável independente. Para visualizarmos a relação entre as duas, é interessante fazer uma representação gráfica da variável dependente em função dos valores da variável independente. Podemos assim verificar uma tendência geral dos pontos, que pode seguir padrões retilíneos, parabólicos, etc. Retomando a citação ao texto de Johann H. Lambert (e a extendendo), temos\cite{Lambert}
\begin{quote}
Se as observações experimentais fossem completamente precisas, essas ordenadas resultariam em um número de pontos através dos quais uma curva ou uma reta deveriam ser traçadas. No entanto, esse não é o caso, a curva/reta desvia pouco ou muito dos pontos observados. Portanto, ela deve ser traçada de maneira que passe tão próxima quanto possível das posições verdadeiras e vá, como se fosse, pelo meio dos pontos em questão.
\end{quote}
%
Muitas vezes tal padrão é muito claro, pois os pontos tem uma \emph{dispersão} baixa. Outras vezes a dispersão é razoavelmente alta e fica difícil verificar o padrão seguido pelos pontos.

Mesmo em casos em que podemos verificar um padrão aparente ao fazer um gráfico, determinar a forma mais adequada para a \emph{linha de tendência} que melhor descreve os pontos experimentais ---~ou mesmo afirmar que tais pontos seguem este padrão~--- pode ser complicado. Se, por exemplo, fizermos uma série de medidas que seguem um padrão parabólico, mas com medidas que se restringem a um intervalo pequeno da variável independente, o gráfico terá a aparência de uma reta (veja as Figuras~\ref{Fig:ParabolaReta} e~\ref{Fig:ParabolaReta2}).
\begin{figure*}
\centering
\forceversofloat
\caption{Gráfico de um conjunto de pontos que aparentemente seguem uma tendência linear. Veja também a Figura~\ref{Fig:ParabolaReta2}.}
\label{Fig:ParabolaReta}
\input{Graphics/Parabola_Reta/graph.tex}
\end{figure*}

\begin{figure*}
\centering
\forceversofloat
\caption{Gráfico do mesmo conjunto de pontos da Figura~\ref{Fig:ParabolaReta}, juntamente com diversos outros pertencentes ao mesmo conjunto de dados. Verifique que a tendência linear aparente no primeiro gráfico já não é mais razoável. De fato, os dados correspondem a uma distribuição em torno de uma parábola.}
\label{Fig:ParabolaReta2}
\input{Graphics/Parabola_Reta/graph2.tex}
\end{figure*}

Determinar o padrão seguido pelos pontos, é, portanto, uma tarefa que não pode ser feita a partir de um gráfico: o mais adequado é termos uma \emph{teoria acerca do fenômeno físico} que descreva qual é o padrão que os pontos devem seguir, e usar um gráfico para determinar se tal descrição é coerente. Verificaremos posteriormente como determinar se uma teoria é plausível ou não, por hora vamos nos preocupar em determinar a \emph{melhor reta} que ajusta um conjunto de dados.

Existem outros tipos de regressão, como a logarítmica ou exponencial, ou mesmo processos capazes de calcular os coeficientes para uma equação com uma forma qualquer. No entanto, nos restringiremos ao caso linear devido ao fato de que a maioria das calculadoras científicas é capaz de realizar tal processo. Além disso, uma vez conhecida a equação da reta, adicioná-la ao gráfico dos dados experimentais é uma questão de calcular dois pontos e traçar uma reta com uma régua.% \comment{se for colocar mais tipos, tem que tirar esse ``nos restringiremos''.}

%%%%%%%%%%%%%%%%%%%%%%%%%%
\section{Regressão Linear}
\label{Sec:RegressaoLinear}
%%%%%%%%%%%%%%%%%%%%%%%%%%

Sempre que tivermos um conjunto de dados, podemos calcular a melhor reta que o representa através de um processo de \emph{regressão linear}. Este processo consiste em aplicar um método matemático que tome os dados experimentais e calcule os coeficientes linear $A$ e angular $B$ para a equação da reta:
\begin{equation}
	y = A + Bx.
\end{equation}
%
A dimensão do coeficiente linear é a mesma que a da variável dependente $y$ ---~ou seja, é a mesma que do eixo $y$ em um gráfico $y \times x$~---, enquanto a dimensão do coeficiente angular é a mesma que a da razão entre a dimensão da variável dependente pela dimensão da variável independente ---~isto é, é a razão entre a dimensão do eixo $y$ e a dimensão do eixo $x$ em um gráfico $y \times x$~---. 

O método utilizado para obter tais coeficientes é o de \emph{mínimos quadrados}. Nele, são encontrados os coeficientes de forma a minimizar o quadrado da distância entre os pontos experimentais e a ``melhor reta''. É possível mostrar\cite{Taylor}, utilizando técnicas de cálculo, que para minimizarmos a soma do quadrado das distâncias entre os pontos e a melhor reta, os coeficientes linear e angular são dados por
\begin{align}
	A &= \frac{\sum x_i^2 \sum y_i - \sum x_i \sum x_iy_i}{N \sum x_i^2 - (\sum x_i)^2} \\
	B &= \frac{N\sum x_iy_i - \sum x_i \sum y_i}{N \sum x_i^2 - (\sum x_i)^2}.
\end{align}
%
onde as somas se dão sobre todas mas medidas $x_i$ e $y_i$, $N$ representa o número de medidas, e as médias das medidas para as variáveis $x$ e $y$ são representadas por $\mean{x}$ e $\mean{y}$, respectivamente. É importante lembrar que as relações acima são válidas somente para o caso em que os erros associados às medidas de $x$ e $y$ são constantes, isto é, a incerteza $\delta x$ para as medidas de $x$ são todas iguais, assim como a incerteza $\delta y$ para as medidas de $y$. Após fazer a regressão, podemos utilizar os coeficientes para traçar as \emph{retas de tendência} nos gráficos\footnote{Quando tais retas forem calculadas, adicione as equações resultantes aos gráficos.}.

Ao fazermos o processo de regressão linear, a calculadora também calculará o \emph{coeficiente de correlação linear} $r$, dado por
\begin{equation}
	r = \frac{\sum (x_i - \mean{x})(y_i-\mean{y})}{\sqrt{\sum(x_i - \mean{x})^2\sum (y_i - \mean{y})^2}}.
\end{equation}
%
Tal coeficiente pode ser interpretado como um ``índice de confiança'' e geralmente é calculado ao quadrado, pois seus valores podem variar entre $-1$ e $1$. Quanto mais próximo $r$ for de $\pm 1$, menor é a dispersão dos pontos em relação ao comportamento retilíneo, ou seja, maiores são as chances de que o fenômeno estudado e que deu origem aos dados siga uma relação linear. Este número geralmente se parece com algo como $r^2=\np{0,99998}$, $r^2=\np{0,997}$, $r^2= \np{0,990}$, etc., quando os dados são altamente lineares e relativamente abundantes. Se a dispersão em relação ao comportamento linear for grande e forem poucos os pontos, $r^2$ pode cair para \np{0,7}, ou valores menores.

%%%%%%%%%%%%%%%%%%%%%%%%%%%%%%%%%%%%%%%%%%
\paragraph{Regressão linear e linearidade}
%%%%%%%%%%%%%%%%%%%%%%%%%%%%%%%%%%%%%%%%%%

\begin{margintable}
\centering
\label{tab:TabelaDadosLin}
\begin{tabular}{cccc}
\toprule     
$x$ & $y_1$ & $y_2$ & $y_3$ \\
\midrule      
\np{0.714}  & \np{14.577}  &   \np{0.678}   &  \np{ 1.235} \\	
\np{2.693}  & \np{20.696}  &   \np{8.806}   &  \np{-0.153} \\	
\np{4.389}  & \np{25.226}  &   \np{20.988}  &  \np{-0.657} \\ 
\np{4.960}  & \np{27.449}  &   \np{28.474}  &  \np{-0.592} \\	
\np{6.245}  & \np{30.242}  &   \np{40.030}  &  \np{ 0.961} \\ 
\np{7.277}  & \np{33.378}  &   \np{55.780}  &  \np{ 1.149} \\ 
\np{7.579}  & \np{34.195}  &   \np{64.904}  &  \np{ 1.635} \\ 
\np{7.719}  & \np{35.715}  &   \np{64.394}  &  \np{ 1.476} \\ 
\np{7.912}  & \np{35.011}  &   \np{66.412}  &  \np{ 1.739} \\ 
\np{8.280}  & \np{37.529}  &   \np{74.632}  &  \np{ 1.249} \\ 
\np{9.034}  & \np{40.590}  &   \np{87.455}  &  \np{ 0.200} \\ 
\np{9.442}  & \np{39.156}  &   \np{94.785}  &  \np{ 0.146} \\ 
\np{10.306} & \np{43.238}  &   \np{113.030} &  \np{-0.231} \\	
\np{10.572} & \np{42.406}  &   \np{111.970} &  \np{-0.575} \\		
\np{11.177} & \np{44.796}  &   \np{129.481} &  \np{ 0.121} \\
\np{15.335} & \np{57.611}  &   \np{235.805} &  \np{ 0.446} \\
\np{17.023} & \np{63.832}  &   \np{294.533} &  \np{-0.780} \\
\np{18.926} & \np{68.063}  &   \np{372.048} &  \np{ 0.500} \\
\np{20.608} & \np{74.408}  &   \np{426.581} &  \np{ 1.118} \\
\np{20.876} & \np{75.083}  &   \np{456.391} &  \np{ 1.065} \\
\np{21.095} & \np{75.248}  &   \np{452.660} &  \np{ 1.085} \\
\np{22.225} & \np{77.243}  &   \np{507.765} &  \np{ 0.271} \\
\np{22.407} & \np{81.058}  &   \np{509.275} &  \np{-0.230} \\
\np{22.469} & \np{78.821}  &   \np{521.414} &  \np{ 0.394} \\
\np{23.077} & \np{80.714}  &   \np{554.175} &  \np{-0.198} \\
\np{26.421} & \np{91.433}  &   \np{700.788} &  \np{ 1.869} \\
\np{26.863} & \np{91.777}  &   \np{732.601} &  \np{ 1.724} \\
\np{27.360} & \np{93.291}  &   \np{773.226} &  \np{ 1.452} \\
\bottomrule
\end{tabular}
\vspace{1mm}
\caption{Dados de três variáveis ($y_1$, $y_2$ e $y_3$) em função de uma quarta ($x$).}
\end{margintable}

Um aspecto importante a ser considerado é se os dados realmente seguem uma tendência linear: uma distribuição qualquer de pontos, mesmo que visivelmente não linear, pode ser descrita por uma melhor reta. Caso a tendência não seja linear, o coeficiente $r^2$ resultará um valor baixo. Nas Figuras~\ref{RetasConjuntosDados1} a~\ref{RetasConjuntosDados3} mostramos três conjuntos de dados distintos  juntamente com suas respectivas melhores retas e coeficientes de dispersão $r^2$. Os dados foram gerados utilizando as expressões $y_1 = A + B(x+r_1) + r_2$, $y_2 = A(r_1+x)^2 + B + r_2$ e $y_3 = A\sen (x+r_1) + B + r_2$, onde $r_1$ e $r_2$ são números aleatórios entre zero e um, enquanto $A$ e $B$ representam constantes arbitrárias ---~veja a Tabela~\ref{tab:TabelaDadosLin}~---. Verificamos que nos três casos podemos calcular uma melhor reta, ainda que para $y_2$ e $y_3$ o comportamento não seja linear.

Dentre as três figuras, a Figura~\ref{RetasConjuntosDados2} se destaca. Nesse caso, temos um comportamento não linear, porém verificamos que o valor do coeficiente $r^2$ é relativamente alto. Ocorre que o comportamento dos dados sofre uma distorção relativamente pequena ---~apesar de visivelmente não linear~--- em relação a um comportamento linear e por isso obtemos um valor grande para o coeficiente de dispersão. Muitas vezes um comportamento verdadeiramente linear pode apresentar uma distorção pequena como esta, porém devido a flutuações aleatórias, e não devido à não-lineariedade. A distinção entre essas duas possibilidades pode ser feita realizando mais medidas: Se o comportamento for verdadeiramente linear, o valor de $r^2$ tenderá a crescer, enquanto no caso de termos um comportamento não-linear, o valor desse coeficiente deve permanecer constante, ou diminuir.

\begin{figure}[!h]\forcerectofloat
\centering
\input{Graphics/Linear_Regression/linear}
\caption{Conjunto de dados 1. A reta da regressão linear é dada por $y(x)=\np{2.984480401}x + \np{12.14046264}$, $r^2 = \np{0.998901256}$.}
\label{RetasConjuntosDados1}
\end{figure}

\begin{figure}[!h]\forcerectofloat
\centering
\input{Graphics/Linear_Regression/quadratic}
\caption{Conjunto de dados 2. A reta da regressão linear é dada por $y(x)=\np{29.97823068} x - \np{152.924429}$, $r^2=\np{0.956577278}$.}
\label{RetasConjuntosDados2}
\end{figure}

\begin{figure}[!h]\forcerectofloat
\centering
\input{Graphics/Linear_Regression/sine}
\caption{Conjunto de dados 3. A reta da regressão linear é dada por $y(x)=\np{0.019262681} x + \np{0.316038383}$, $r^2=\np{0,035985674}$.}
\label{RetasConjuntosDados3}
\end{figure}

\vfill
\pagebreak
%%%%%%%%%%%%%%%%%%%%%%%%%%%%%%%%%%%%%%%%%%%%%%%%%%%%%%%%%%%%%%%%%
\paragraph{Exemplo: Cálculo dos coeficientes da regressão linear}
%%%%%%%%%%%%%%%%%%%%%%%%%%%%%%%%%%%%%%%%%%%%%%%%%%%%%%%%%%%%%%%%%

Para determinarmos os valores das constantes $A$, $B$, e do coeficiente $r^2$ é necessário calcularmos uma série de valores intermediários. Na Tabela~\ref{Tab:TabelaExemploCalculoRegressaoLinear} (página~\pageref{Tab:TabelaExemploCalculoRegressaoLinear}) apresentamos tais valores intermediários para o caso do cálculo dos coeficientes da regressão linear para a variável $y_1$ discutida acima. Através desses valores, obtemos para o coeficiente linear
\begin{align}
    A &= \frac{\sum x_i^2 \sum y_i - \sum x_i \sum x_iy_i}{N \sum x_i^2 - (\sum x_i)^2} \\
    &= \frac{\np{7295.757} \cdot \np{1512.768} - \np{392.984} \cdot \np{26545.051}}{\np{28}\cdot\np{7295.757} - \np{392.984}^2} \\
    &= \np{12.14046264}.
\end{align}
%
Para o coeficiente angular, obtemos
\begin{align}
    B &= \frac{N\sum x_iy_i - \sum x_i \sum y_i}{N \sum x_i^2 - (\sum x_i)^2} \\
    &= \frac{\np{28} \cdot \np{26545.051} - \np{392.984}\cdot\np{1512.786}}{\np{28}\cdot\np{7295.757} - \np{392.984}^2} \\
    &= \np{2.984480401}.
\end{align}
%
Finalmente, para o coeficiente $r$ obtemos
\begin{align}
    r &= \frac{\sum (x_i - \mean{x})(y_i-\mean{y})}{\sqrt{\sum(x_i - \mean{x})^2\sum (y_i - \mean{y})^2}} \\
    &= \frac{\np{5312.883}}{\sqrt{\np{1780.170}\cdot\sum{15873.637}}} \\
    &= \np{0.999450477},
\end{align}
%
o que resulta em um $r^2$ dado por
\begin{equation}
    r^2 = \np{0.998901256}.
\end{equation}

Note que o cálculo é bastante trabalhoso se for realizado manualmente. Felizmente, esse tipo de análise ---~a regressão linear~--- é algo que é implementado na maioria dos programas de computador que permitem a análise de dados. Mesmo em planilhas de cálculo existem funções que permitem calcular as constantes $A$ e $B$, e o coeficiente $r^2$ facilmente. Funções para a determinação de tais valores também estão disponíveis em calculadoras científicas. Na seção seguinte verificaremos como proceder para inserir os dados experimentais e obter os coeficientes da regressão linear para alguns modelos comuns de calculadoras. 

\begin{table*}[tp]
\caption[][1cm]{Tabela de cálculos para a determinação das constantes $A$ e $B$, e do coeficiente $r^2$. As linhas marcadas com $\sum$ e $\mean{~}$ denotam as somas e as médias dos valores apresentados para cada variável, respectivamente. Os valores apresentados na tabela foram limitados a três casas após a vírgula, porém os cálculos foram feitos com toda a precisão disponível. \label{Tab:TabelaExemploCalculoRegressaoLinear}}
\small
\begin{tabular}{ccccccccccc}
\toprule
$i$ & $x_i$ & $y_{1,i}$ & $x_i^2$ & $x_i y_i$ & $x_i - \mean{x}$ & $y_i - \mean{y}$ & $(x_i - \mean{x})^2$ & $(y_i - \mean{y})^2$ & $(x_i - \mean{x})(y_i - \mean{y})$ \\
\midrule
1	& \np{0.714} 	& \np{14.577}	& \np{0.510}	& \np{10.408}	& \np{-13.321}	& \np{-39.451}	& \np{177.453}	& \np{1556.387}	& \np{525.533} \\
2	& \np{2.693}	& \np{20.696}	& \np{7.252}	& \np{55.734}	& \np{-11.342}	& \np{-33.332}	& \np{128.644}	& \np{1111.027}	& \np{378.057} \\
3	& \np{4.389}	& \np{25.226}	& \np{19.263}	& \np{110.717}	& \np{-9.646}	& \np{-28.802}	& \np{93.048}	& \np{829.559}	& \np{277.828} \\
4	& \np{4.960}	& \np{27.449}	& \np{24.602}	& \np{136.147}	& \np{-9.075}	& \np{-26.579}	& \np{82.358}	& \np{706.447}	& \np{241.208} \\
5	& \np{6.245}	& \np{30.242}	& \np{39.000}	& \np{188.861}	& \np{-7.790}	& \np{-23.786}	& \np{60.686}	& \np{565.777}	& \np{185.296} \\
6	& \np{7.277}	& \np{33.378}	& \np{52.955}	& \np{242.892}	& \np{-6.758}	& \np{-20.650}	& \np{45.672}	& \np{426.425}	& \np{139.556} \\
7	& \np{7.579}	& \np{34.195}	& \np{57.441}	& \np{259.164}	& \np{-6.456}	& \np{-19.833}	& \np{41.682}	& \np{393.351}	& \np{128.045} \\
8	& \np{7.719}	& \np{35.715}	& \np{59.583}	& \np{275.684}	& \np{-6.316}	& \np{-18.313}	& \np{39.894}	& \np{335.369}	& \np{115.667} \\
9	& \np{7.912}	& \np{35.011}	& \np{62.600}	& \np{277.007}	& \np{-6.123}	& \np{-19.017}	& \np{37.493}	& \np{361.649}	& \np{116.444} \\
10	& \np{8.280}	& \np{37.529}	& \np{68.558}	& \np{310.740}	& \np{-5.755}	& \np{-16.499}	& \np{33.122}	& \np{272.219}	& \np{94.955} \\
11	& \np{9.034}	& \np{40.590}	& \np{81.613}	& \np{366.690}	& \np{-5.001}	& \np{-13.438}	& \np{25.0114}	& \np{180.582}	& \np{67.206} \\
12	& \np{9.442}	& \np{39.156}	& \np{89.151}	& \np{369.711}	& \np{-4.593}	& \np{-14.872}	& \np{21.097}	& \np{221.179}	& \np{68.310} \\
13	& \np{10.306}	& \np{43.238}	& \np{106.214}	& \np{445.611}	& \np{-3.729}	& \np{-10.790}	& \np{13.907}	& \np{116.426}	& \np{40.238} \\
14	& \np{10.572}	& \np{42.406}	& \np{111.767}	& \np{448.316}	& \np{-3.463}	& \np{-11.622}	& \np{11.993}	& \np{135.073}	& \np{40.249} \\
15	& \np{11.177}	& \np{44.796}	& \np{124.925}	& \np{500.685}	& \np{-2.858}	& \np{-9.232}	& \np{8.169}	& \np{85.231}	& \np{26.387} \\
16	& \np{15.335}	& \np{57.611}	& \np{235.162}	& \np{883.465}	& \np{1.300}	& \np{3.583}	& \np{1.690}	& \np{12.837}	& \np{4.657} \\
17	& \np{17.023}	& \np{63.832}	& \np{289.783}	& \np{1086.612}	& \np{2.988}	& \np{9.804}	& \np{8.927}	& \np{96.117}	& \np{29.293} \\
18	& \np{18.926}	& \np{68.063}	& \np{358.193}	& \np{1288.160}	& \np{4.891}	& \np{14.035}	& \np{23.920}	& \np{196.979}	& \np{68.643} \\
19	& \np{20.608}	& \np{74.408}	& \np{424.690}	& \np{1533.400}	& \np{6.573}	& \np{20.380}	& \np{43.202}	& \np{415.341}	& \np{133.954} \\
20	& \np{20.876}	& \np{75.083}	& \np{435.807}	& \np{1567.433}	& \np{6.841}	& \np{21.055}	& \np{46.797}	& \np{443.310}	& \np{144.033} \\
21	& \np{21.095}	& \np{75.248}	& \np{444.999}	& \np{1587.357}	& \np{7.060}	& \np{21.220}	& \np{49.842}	& \np{450.285}	& \np{149.809} \\
22	& \np{22.225}	& \np{77.243}	& \np{493.951}	& \np{1716.726}	& \np{8.190}	& \np{23.215}	& \np{67.074}	& \np{538.933}	& \np{190.126} \\
23	& \np{22.407}	& \np{81.058}	& \np{502.074}	& \np{1816.267}	& \np{8.372}	& \np{27.030}	& \np{70.088}	& \np{730.617}	& \np{226.290} \\
24	& \np{22.469}	& \np{78.821}	& \np{504.856}	& \np{1771.029}	& \np{8.434}	& \np{24.793}	& \np{71.130}	& \np{614.689}	& \np{209.100} \\
25	& \np{23.077}	& \np{80.714}	& \np{532.548}	& \np{1862.637}	& \np{9.042}	& \np{26.686}	& \np{81.755}	& \np{712.139}	& \np{241.290} \\
26	& \np{26.421}	& \np{91.433}	& \np{698.069}	& \np{2415.751}	& \np{12.386}	& \np{37.405}	& \np{153.409}  & \np{1399.129}	& \np{463.292} \\
27	& \np{26.863}	& \np{91.777}	& \np{721.621}	& \np{2465.406}	& \np{12.828}	& \np{37.749}	& \np{164.554}  & \np{1424.982}	& \np{484.237} \\
28	& \np{27.360}	& \np{93.291}	& \np{748.570}	& \np{2552.442}	& \np{13.325}	& \np{39.263}	& \np{177.552}  & \np{1541.578}	& \np{523.172} \\
\midrule
$\sum$	& \np{392.984}	& \np{1512.786}	& \np{7295.757}	& \np{26545.051}	& 0	& 0	& \np{1780.170}	& \np{15873.637}	& \np{5312.883} \\
$\mean{~}$ & \np{14.035}	& \np{54.028} \\				
\bottomrule
\end{tabular}
\end{table*}

\FloatBarrier

%%%%%%%%%%%%%%%%%%%%%%%%%%%%%%%%%%%%%%%%%%%%%%%%%%%%%%%%%%%%%%%%%%%%%%
\subsection{Regressão linear utilizando uma calculadora}
%%%%%%%%%%%%%%%%%%%%%%%%%%%%%%%%%%%%%%%%%%%%%%%%%%%%%%%%%%%%%%%%%%%%%%

Geralmente calculadoras científicas são capazes de realizar regressões lineares e diversas outras, facilitando a obtenção da melhor reta. Apresentaremos abaixo como realizar tal cálculo em alguns modelos.

%%%%%%%%%%%%%%%%%%%%%%%%%%%
\paragraph{CASIO$^{\circledR}$ $fx$-\textit{82TL}}
%%%%%%%%%%%%%%%%%%%%%%%%%%%

\begin{itemize}
\item Pressione o botão \keystroke{~MODE~}. As informações abaixo aparecerão no visor:
\begin{center}
\begin{tabular}{p{20mm}p{20mm}p{20mm}}
COMP & SD & REG\\
1 & 2 & 3
\end{tabular}
\end{center}

\item Pressione o botão \keystroke{~3~}. Teremos no visor
\begin{center}
\begin{tabular}{p{20mm}p{20mm}p{20mm}}
Lin & Log & Exp \phantom{xxx}\ding{225} \\
1 & 2 & 3
\end{tabular}
\end{center}
Ao pressionarmos o botão \keystroke{~1~}, a calculadora estará no modo de regressão linear, indicado por \texttt{REG} no visor.

\item Podemos agora digitar o valor da variável independente $x_1$, correspondente ao primeiro ponto, seguido do botão \keystroke{~,~}. Digitamos após a vírgula o valor da variável dependente $y_1$ correspondente ao primeiro ponto. Após isso, basta pressionar \keystroke{~M+~} para inserir o par de valores na memória da calculadora. Repetiremos esse processo para cada par $x_i$, $y_i$.

\item Quando todos os valores tiverem sido inseridos, podemos recuperar os valores de $A$, $B$ e $r$ pressionando \keystroke{~SHIFT~} seguido de 
\begin{itemize}
	\item tecla \keystroke{7} e então \keystroke{=} para recuperar o valor de $A$;
	\item tecla \keystroke{8} e então \keystroke{=} para recuperar o valor de $B$;
	\item tecla \keystroke{(} e então \keystroke{=} para recuperar o valor de $r$.
\end{itemize}

Para recuperar outro valor, não é necessário inserir os pontos novamente, basta pressionar mais uma vez \keystroke{~SHIFT~} seguido da tecla correspondente à variável que desejamos.

\item Para realizar uma nova regressão, devemos antes apagar os dados da regressão anterior. Fazemos isso pressionando \keystroke{~SHIFT~} e então \keystroke{~AC/ON~} (função \texttt{Scl}). Na tela aparecerá \texttt{Scl}. Pressione \keystroke{=} para confirmar a exclusão dos dados.
\end{itemize}

%%%%%%%%%%%%%%%%%%%%%%%%%%%
\paragraph{CASIO$^{\circledR}$ $fx$-\textit{82MS}}
%%%%%%%%%%%%%%%%%%%%%%%%%%%

\begin{itemize}
\item Pressione o botão \keystroke{~MODE~}. As informações abaixo aparecerão no visor:
\begin{center}
\begin{tabular}{p{20mm}p{20mm}p{20mm}}
COMP & SD & REG\\
1 & 2 & 3
\end{tabular}
\end{center}

\item Pressione o botão \keystroke{~3~}. Teremos no visor
\begin{center}
\begin{tabular}{p{20mm}p{20mm}p{20mm}}
Lin & Log & Exp \phantom{xxx}\ding{225} \\
1 & 2 & 3
\end{tabular}
\end{center}
Ao pressionarmos o botão \keystroke{~1~}, a calculadora estará no modo de regressão linear, indicado por \texttt{REG} no visor.

\item Podemos agora digitar o valor da variável independente $x_1$, correspondente ao primeiro ponto, seguido do botão \keystroke{~,~}. Digitamos após a vírgula o valor da variável dependente $y_1$ correspondente ao primeiro ponto. Após isso, basta pressionar \keystroke{~M+~} para inserir o par de valores na memória da calculadora. Repetiremos esse processo para cada par $x_i$, $y_i$.

\item Quando todos os valores tiverem sido inseridos, podemos recuperar os valores de $A$, $B$ e $r$ pressionando \keystroke{~SHIFT~} seguido do botão \keystroke{~2~} (função \texttt{S-var}). Neste momento, aparecerá na tela
\begin{center}
\begin{tabular}{p{20mm}p{20mm}p{20mm}}
$\overline{x}$ & $x\sigma n$ & $x\sigma n-1$ \phantom{xx}\ding{225} \\
1 & 2 & 3
\end{tabular}
\end{center}
%
Se pressionarmos para a direita no botão direcional duas vezes, na tela teremos
\begin{center}
\begin{tabular}{p{20mm}p{20mm}p{20mm}}
A & B & r \\
1 & 2 & 3
\end{tabular}
\end{center}
%
Basta agora escolher qual variável desejamos, pressionar o botão correspondente ---~\keystroke{~1~}, \keystroke{~2~} ou \keystroke{~3~}~--- e pressionar \keystroke{~=~}. Para recuperar outro valor, não é necessário inserir os pontos novamente, basta pressionar mais uma vez \keystroke{~SHIFT~} seguido de \keystroke{~2~} (função \texttt{S-var}) e escolher outra variável.

\item Para realizar uma nova regressão, devemos antes apagar os dados da regressão anterior. Fazemos isso pressionando \keystroke{~SHIFT~} e então \keystroke{~MODE~} (função \texttt{CLR}). Na tela aparecerá
\begin{center}
\begin{tabular}{p{2cm}p{2cm}p{2cm}}
Scl & Mode & All \\
1 & 2 & 3
\end{tabular}
\end{center}
%
Selecione \texttt{Scl} pressionando o botão \keystroke{~1~} e pressione \keystroke{~=~}.
\end{itemize}

%%%%%%%%%%%%%%%%%%%%%%%%%%%
\paragraph{HP$^{\circledR}$ 50g}
%%%%%%%%%%%%%%%%%%%%%%%%%%%

\begin{itemize}
\item Pressione o botão \keystroke{~$\Rsh$~} e selecione a função \texttt{`STAT`} (no botão \keystroke{~5~}).

\item Selecione a opção \texttt{`Fit data \dots'} e pressione o botão \keystroke{~F6~}, correspondente à ação \texttt{`OK'}, ou o botão \keystroke{~ENTER~}.

\item Na tela mostrada, a primeira linha, correspondendo à entrada \texttt{`}$\Sigma$\texttt{DAT'} estará selecionada. O texto \texttt{`Enter statistical data'} será mostrado na linha inferior. Caso existam dados de uma regressão anterior, eles poderão ser apagados usando o botão \keystroke{~$\Lsh$~}, seguido do botão \keystroke{~$\leftarrow$~}, acessando a ação \texttt{`DEL'}. Pressione o botão \keystroke{~F1~}, que corresponde à ação de edição (\texttt{`EDIT'}).

\item A calculadora passará a mostrar a tela de edição de tabela. Digite os valores nas células correspondentes. Por padrão, a primeira coluna deverá conter os valores da variável independente $x$ e a segunda os valores da variável independente $y$. Ao finalizar a digitação do conteúdo de uma célula, pressione \keystroke{~ENTER~}. Note que digitar uma entrada da tabela é possível fazer operações como potenciação, raízes quadradas, etc., bastando digitá-las como seriam calculadas na tela inicial da calculadora. Isso pode ser útil ao fazer linearizações dos dados.

\item Ao finalizar a digitação do conteúdo da última célula, pressione \keystroke{~ENTER~} novamente, o que fechará o modo de edição de tabela. Se for necessário editar algum valor, basta repetir o item anterior para retornar ao modo de edição de tabela.

\item Se necessário, é possível alterar a coluna correspondente às variáveis independente e dependente nas entradas \texttt{`X-Col'} e \texttt{`Y-Col'}, respectivamente.

\item Na entrada \texttt{`Model'} podemos escolher o modelo de regressão. Para isso, pressione o botão \keystroke{~F2~}, correspondente à ação \texttt{`CHOOS'}, selecione o modelo desejado, e pressione \texttt{`OK'} (através do botão \keystroke{~F6~}) ou \keystroke{~ENTER~}.

\item Finalmente, selecione a ação \texttt{`OK'} (botão \keystroke{~F6~}) para realizar a regressão. A calculadora retornará à tela inicial, mostrando a equação obtida na forma $A + B \cdot x$, a correlação e a covariância. A correlação corresponde ao valor de $r$. Caso seja necessário, utilize a seta para cima para selecionar uma linha e a ação \texttt{`VIEW'} (botão \keystroke{~F2~}) para entrar na tela de visualização, o que permite rolar a tela usando as setas. Para sair da tela de visualização, basta selecionar a ação \texttt{`OK'}.

\item Finalmente, note que se a regressão desejada e as colunas estiverem corretas, é possível realizar a regressão imediatamente após entrar com os dados na tabela, bastando usar a ação \texttt{`OK'}.
\end{itemize}

%%%%%%%%%%%%%%%%%%%%%%%%%%%
\paragraph{CASIO$^{\circledR}$ $fx$-\textit{570ES} e $fx$-\textit{991ES Plus}}
%%%%%%%%%%%%%%%%%%%%%%%%%%%

\begin{itemize}
	\item Ao pressionar a tecla \keystroke{~MODE~}, aparecerão diversas opções, entre elas a opção `\texttt{3:~STAT}'. Pressione a tecla \keystroke{~3~} para selecioná-la. Uma nova tela aparecerá com opções.
	\item Selecione a opção `\texttt{2:~A+Bx}' pressionando a tecla \keystroke{~2~}. Ao selecionar esta opção, uma tabela surgirá para a entrada de dados.
	\item Coloque o valor da abscissa $x_1$ da primeira medida e pressione \keystroke{~=~}. A calculadora passará ao próximo valor de abscissa automaticamente, o que torna mais fácil a entrada de todos os valores para essa variável. Ao finalizar, navegue usando as teclas direcionais até o local de inserção da primeira ordenada. Insira o valor e tecle \keystroke{~=~}. A calculadora passará automaticamente ao próximo campo, permitindo a inserção do próximo valor de ordenada. Como a calculadora tem o comportamento automático de passar ao próximo valor da mesma variável, é mais fácil segui-lo, porém podemos inserir os pares $x,y$ se utilizarmos as teclas direcionais após pressionar \keystroke{~=~}. Qualquer erro pode ser corrigido navegando novamente ao campo e reinserindo os valores.
	\item Após inserir todos os valores, pressione a tecla \keystroke{~AC~}.
	\item Para recuperar os valores de $A$, $B$, e $r$, devemos pressionar a tecla \keystroke{~SHIFT~}, seguida da tecla \keystroke{~1~} (função \texttt{STAT}). Uma nova tela de opções surgirá.
	\item No modelo $fx$-\textit{570ES} devemos selecionar a opção `\texttt{7:~Reg}' pressionando a tecla \keystroke{~7~}. No modelo $fx$-\textit{991ES Plus}, devemos pressionar a tecla \keystroke{~5~} (opção `\texttt{5:~Reg}`). Uma tela surgirá com as opções correspondentes às variáveis $A$, $B$ e $r$. Selecione a variável que deseja recuperar através das teclas numéricas correspondentes e então pressione \keystroke{~=~}. Para recuperar outra variável, basta repetir os passos desse ítem, não sendo necessário digitar novamente os dados experimentais.
	\item Para realizar uma nova regressão, basta repetirmos os passos desde o início. Quando a tabela de inserção de dados surgir, ela estará vazia.
\end{itemize}

%%%%%%%%%%%%%%%%%%%%%%%%%%%%%%%%%%%%%%%%%%%
\subsection{Interpretação dos coeficientes}
%%%%%%%%%%%%%%%%%%%%%%%%%%%%%%%%%%%%%%%%%%%

A ideia por trás do cálculo da melhor reta é estabelecer quais seriam os coeficientes mais adequados para uma relação linear que descreve o fenômeno estudado. \emph{Esses coeficientes são importantes pois estão, em geral, ligados a constantes físicas cujo valor estamos interessados em medir}. Além disso, esse processo é mais preciso do que simplesmente calcular o valor dos coeficientes da reta associados aos pontos medidos e depois fazer uma média.

\begin{margintable}\centering
\begin{tabular}{cc}
\toprule
$(t \pm \np{0,01})~\textrm{s}$ & $(v \pm \np{0,01})~\textrm{m}/\textrm{s}$ \\
\midrule
\np{0,10} & \np{24,04} \\
\np{0,20} & \np{25,14} \\
\np{0,30} & \np{25,79} \\
\np{0,40} & \np{27,08} \\
\np{0,50} & \np{27,33} \\
\np{0,60} & \np{28,79} \\
\np{0,70} & \np{29,98} \\
\np{0,80} & \np{30,61} \\
\np{0,90} & \np{31,15} \\
\np{1,00} & \np{32,94} \\
\np{1,10} & \np{34,04} \\
\np{1,20} & \np{34,78} \\
\np{1,30} & \np{35,22} \\
\np{1,40} & \np{36,10} \\
\np{1,50} & \np{37,91} \\
\bottomrule
\end{tabular}
\vspace{1mm}
\caption{Dados medidos para a velocidade em função do tempo para um experimento hipotético.}
\label{DadosVelocidade}
\end{margintable}

Vamos considerar, por exemplo, o conjunto de dados para a velocidade em função do tempo dados na Tabela~\ref{DadosVelocidade}, representados na Figura~\ref{fig:VelocidadeTempo}. Verificamos nos dados da tabela que a velocidade se altera com o tempo. Supondo que tenhamos um movimento com aceleração constante, podemos descrever os dados em função do tempo como
\begin{equation}
	v = v_0 + at.
\end{equation}
%
Comparando esta equação com a equação da reta $y = A + Bx$, verificamos as relações
\begin{align}
	y &= v \\
	A &= v_0 \\
	B &= a \\
	x &= t.
\end{align}
%
Para determinar a relação entre as variáveis das duas equações, devemos verificar qual variável dos dados foi varrida arbitrariamente ---~e que então deve corresponder à variável independente $x$~--- e qual foi lida em resposta à primeira ---~correspondendo à variável dependente $y$~---. Vemos na Tabela~\ref{DadosVelocidade} que os valores de tempo são mais condizentes com uma variação arbitrária (valores ``redondos'' ou semi-inteiros, isto é, variados com um passo regular) do que no caso dos valores da velocidade. No caso de realizarmos um experimento, não teremos problemas em determinar quem foi a variável independente, pois realizaremos essa escolha ao idealizá-lo.

Dessa forma, se tomarmos os dados e realizarmos uma regressão linear, vamos obter os valores de $v_0$ e de $a$ ---~que são desconhecidos~--- através das constantes $A$ e $B$. Para os valores da tabela obtemos $v_0 = \np{22,734 285 714 3}$ e $a = \np{9,907 142 857 1}$. Verificaremos adiante que nem todos os digitos obtidos para as constantes $A$ e $B$ são relevantes e por isso deveremos descartar alguns deles. Por ora podemos expressá-las com o mesmo número de casas após a vírgula que a variável ($x$ ou $y$) que tem menos casas após a vírgula:
\begin{align}
	A = v_0 &= \np[m/s]{22,73} \\
	B = a &= \np[m/s^2]{9,91}.
\end{align}

\begin{figure*}[!htbt]
\centering
\caption{Gráfico dos dados da Tabela~\ref{DadosVelocidade}.} 
\label{fig:VelocidadeTempo}
\begin{tikzpicture}[gnuplot]
%% generated with GNUPLOT 5.4p9 (Lua 5.4; terminal rev. Jun 2020, script rev. 118)
%% sex 18 out 2024 08:55:30
\path (0.000,0.000) rectangle (14.000,9.000);
\gpcolor{color=gp lt color border}
\gpsetlinetype{gp lt border}
\gpsetdashtype{gp dt solid}
\gpsetlinewidth{1.00}
\draw[gp path] (1.136,0.985)--(1.316,0.985);
\draw[gp path] (13.447,0.985)--(13.267,0.985);
\node[gp node right] at (0.952,0.985) {$20$};
\draw[gp path] (1.136,2.736)--(1.316,2.736);
\draw[gp path] (13.447,2.736)--(13.267,2.736);
\node[gp node right] at (0.952,2.736) {$25$};
\draw[gp path] (1.136,4.488)--(1.316,4.488);
\draw[gp path] (13.447,4.488)--(13.267,4.488);
\node[gp node right] at (0.952,4.488) {$30$};
\draw[gp path] (1.136,6.239)--(1.316,6.239);
\draw[gp path] (13.447,6.239)--(13.267,6.239);
\node[gp node right] at (0.952,6.239) {$35$};
\draw[gp path] (1.136,7.990)--(1.316,7.990);
\draw[gp path] (13.447,7.990)--(13.267,7.990);
\node[gp node right] at (0.952,7.990) {$40$};
\draw[gp path] (1.136,0.985)--(1.136,1.165);
\draw[gp path] (1.136,8.691)--(1.136,8.511);
\node[gp node center] at (1.136,0.677) {$0$};
\draw[gp path] (2.675,0.985)--(2.675,1.165);
\draw[gp path] (2.675,8.691)--(2.675,8.511);
\node[gp node center] at (2.675,0.677) {$0{,}2$};
\draw[gp path] (4.214,0.985)--(4.214,1.165);
\draw[gp path] (4.214,8.691)--(4.214,8.511);
\node[gp node center] at (4.214,0.677) {$0{,}4$};
\draw[gp path] (5.753,0.985)--(5.753,1.165);
\draw[gp path] (5.753,8.691)--(5.753,8.511);
\node[gp node center] at (5.753,0.677) {$0{,}6$};
\draw[gp path] (7.292,0.985)--(7.292,1.165);
\draw[gp path] (7.292,8.691)--(7.292,8.511);
\node[gp node center] at (7.292,0.677) {$0{,}8$};
\draw[gp path] (8.830,0.985)--(8.830,1.165);
\draw[gp path] (8.830,8.691)--(8.830,8.511);
\node[gp node center] at (8.830,0.677) {$1$};
\draw[gp path] (10.369,0.985)--(10.369,1.165);
\draw[gp path] (10.369,8.691)--(10.369,8.511);
\node[gp node center] at (10.369,0.677) {$1{,}2$};
\draw[gp path] (11.908,0.985)--(11.908,1.165);
\draw[gp path] (11.908,8.691)--(11.908,8.511);
\node[gp node center] at (11.908,0.677) {$1{,}4$};
\draw[gp path] (13.447,0.985)--(13.447,1.165);
\draw[gp path] (13.447,8.691)--(13.447,8.511);
\node[gp node center] at (13.447,0.677) {$1{,}6$};
\draw[gp path] (1.136,8.691)--(1.136,0.985)--(13.447,0.985)--(13.447,8.691)--cycle;
\node[gp node center,rotate=-270] at (0.108,4.838) {$v$~(m/s)};
\node[gp node center] at (7.291,0.215) {$t$~(s)};
\node[gp node left] at (2.604,8.280) {Dados experimentais};
\gpcolor{rgb color={0.000,0.000,0.000}}
\gpsetlinewidth{2.00}
\gpsetpointsize{4.00}
\gp3point{gp mark 7}{}{(1.905,2.050)}
\gp3point{gp mark 7}{}{(2.675,2.785)}
\gp3point{gp mark 7}{}{(3.444,3.013)}
\gp3point{gp mark 7}{}{(4.214,3.465)}
\gp3point{gp mark 7}{}{(4.983,3.552)}
\gp3point{gp mark 7}{}{(5.753,4.064)}
\gp3point{gp mark 7}{}{(6.522,4.481)}
\gp3point{gp mark 7}{}{(7.292,4.701)}
\gp3point{gp mark 7}{}{(8.061,4.891)}
\gp3point{gp mark 7}{}{(8.830,5.518)}
\gp3point{gp mark 7}{}{(9.600,5.903)}
\gp3point{gp mark 7}{}{(10.369,6.162)}
\gp3point{gp mark 7}{}{(11.139,6.316)}
\gp3point{gp mark 7}{}{(11.908,6.624)}
\gp3point{gp mark 7}{}{(12.678,7.258)}
\gp3point{gp mark 7}{}{(1.962,8.280)}
\gpcolor{color=gp lt color border}
\node[gp node left] at (2.604,7.818) {$y(x) = \np{22.7342857143} + \np{9.9071428571}\cdot x$, $r^2 = \np{0.9924824357}$};
\gpcolor{rgb color={0.000,0.000,0.000}}
\draw[gp path] (1.504,7.818)--(2.420,7.818);
\draw[gp path] (1.905,2.290)--(1.911,2.292)--(1.916,2.295)--(1.922,2.297)--(1.927,2.299)%
  --(1.932,2.302)--(1.938,2.304)--(1.943,2.307)--(1.949,2.309)--(1.954,2.312)--(1.959,2.314)%
  --(1.965,2.316)--(1.970,2.319)--(1.975,2.321)--(1.981,2.324)--(1.986,2.326)--(1.992,2.329)%
  --(1.997,2.331)--(2.002,2.333)--(2.008,2.336)--(2.013,2.338)--(2.019,2.341)--(2.024,2.343)%
  --(2.029,2.346)--(2.035,2.348)--(2.040,2.350)--(2.045,2.353)--(2.051,2.355)--(2.056,2.358)%
  --(2.062,2.360)--(2.067,2.363)--(2.072,2.365)--(2.078,2.367)--(2.083,2.370)--(2.089,2.372)%
  --(2.094,2.375)--(2.099,2.377)--(2.105,2.380)--(2.110,2.382)--(2.115,2.385)--(2.121,2.387)%
  --(2.126,2.389)--(2.132,2.392)--(2.137,2.394)--(2.142,2.397)--(2.148,2.399)--(2.153,2.402)%
  --(2.159,2.404)--(2.164,2.406)--(2.169,2.409)--(2.175,2.411)--(2.180,2.414)--(2.186,2.416)%
  --(2.191,2.419)--(2.196,2.421)--(2.202,2.423)--(2.207,2.426)--(2.212,2.428)--(2.218,2.431)%
  --(2.223,2.433)--(2.229,2.436)--(2.234,2.438)--(2.239,2.440)--(2.245,2.443)--(2.250,2.445)%
  --(2.256,2.448)--(2.261,2.450)--(2.266,2.453)--(2.272,2.455)--(2.277,2.457)--(2.282,2.460)%
  --(2.288,2.462)--(2.293,2.465)--(2.299,2.467)--(2.304,2.470)--(2.309,2.472)--(2.315,2.474)%
  --(2.320,2.477)--(2.326,2.479)--(2.331,2.482)--(2.336,2.484)--(2.342,2.487)--(2.347,2.489)%
  --(2.352,2.491)--(2.358,2.494)--(2.363,2.496)--(2.369,2.499)--(2.374,2.501)--(2.379,2.504)%
  --(2.385,2.506)--(2.390,2.508)--(2.396,2.511)--(2.401,2.513)--(2.406,2.516)--(2.412,2.518)%
  --(2.417,2.521)--(2.422,2.523)--(2.428,2.525)--(2.433,2.528)--(2.439,2.530)--(2.444,2.533)%
  --(2.449,2.535)--(2.455,2.538)--(2.460,2.540)--(2.466,2.542)--(2.471,2.545)--(2.476,2.547)%
  --(2.482,2.550)--(2.487,2.552)--(2.493,2.555)--(2.498,2.557)--(2.503,2.559)--(2.509,2.562)%
  --(2.514,2.564)--(2.519,2.567)--(2.525,2.569)--(2.530,2.572)--(2.536,2.574)--(2.541,2.576)%
  --(2.546,2.579)--(2.552,2.581)--(2.557,2.584)--(2.563,2.586)--(2.568,2.589)--(2.573,2.591)%
  --(2.579,2.593)--(2.584,2.596)--(2.589,2.598)--(2.595,2.601)--(2.600,2.603)--(2.606,2.606)%
  --(2.611,2.608)--(2.616,2.610)--(2.622,2.613)--(2.627,2.615)--(2.633,2.618)--(2.638,2.620)%
  --(2.643,2.623)--(2.649,2.625)--(2.654,2.627)--(2.659,2.630)--(2.665,2.632)--(2.670,2.635)%
  --(2.676,2.637)--(2.681,2.640)--(2.686,2.642)--(2.692,2.644)--(2.697,2.647)--(2.703,2.649)%
  --(2.708,2.652)--(2.713,2.654)--(2.719,2.657)--(2.724,2.659)--(2.730,2.661)--(2.735,2.664)%
  --(2.740,2.666)--(2.746,2.669)--(2.751,2.671)--(2.756,2.674)--(2.762,2.676)--(2.767,2.678)%
  --(2.773,2.681)--(2.778,2.683)--(2.783,2.686)--(2.789,2.688)--(2.794,2.691)--(2.800,2.693)%
  --(2.805,2.695)--(2.810,2.698)--(2.816,2.700)--(2.821,2.703)--(2.826,2.705)--(2.832,2.708)%
  --(2.837,2.710)--(2.843,2.712)--(2.848,2.715)--(2.853,2.717)--(2.859,2.720)--(2.864,2.722)%
  --(2.870,2.725)--(2.875,2.727)--(2.880,2.729)--(2.886,2.732)--(2.891,2.734)--(2.896,2.737)%
  --(2.902,2.739)--(2.907,2.742)--(2.913,2.744)--(2.918,2.746)--(2.923,2.749)--(2.929,2.751)%
  --(2.934,2.754)--(2.940,2.756)--(2.945,2.759)--(2.950,2.761)--(2.956,2.763)--(2.961,2.766)%
  --(2.966,2.768)--(2.972,2.771)--(2.977,2.773)--(2.983,2.776)--(2.988,2.778)--(2.993,2.780)%
  --(2.999,2.783)--(3.004,2.785)--(3.010,2.788)--(3.015,2.790)--(3.020,2.793)--(3.026,2.795)%
  --(3.031,2.797)--(3.037,2.800)--(3.042,2.802)--(3.047,2.805)--(3.053,2.807)--(3.058,2.810)%
  --(3.063,2.812)--(3.069,2.814)--(3.074,2.817)--(3.080,2.819)--(3.085,2.822)--(3.090,2.824)%
  --(3.096,2.827)--(3.101,2.829)--(3.107,2.831)--(3.112,2.834)--(3.117,2.836)--(3.123,2.839)%
  --(3.128,2.841)--(3.133,2.844)--(3.139,2.846)--(3.144,2.848)--(3.150,2.851)--(3.155,2.853)%
  --(3.160,2.856)--(3.166,2.858)--(3.171,2.861)--(3.177,2.863)--(3.182,2.865)--(3.187,2.868)%
  --(3.193,2.870)--(3.198,2.873)--(3.203,2.875)--(3.209,2.878)--(3.214,2.880)--(3.220,2.882)%
  --(3.225,2.885)--(3.230,2.887)--(3.236,2.890)--(3.241,2.892)--(3.247,2.895)--(3.252,2.897)%
  --(3.257,2.899)--(3.263,2.902)--(3.268,2.904)--(3.273,2.907)--(3.279,2.909)--(3.284,2.912)%
  --(3.290,2.914)--(3.295,2.916)--(3.300,2.919)--(3.306,2.921)--(3.311,2.924)--(3.317,2.926)%
  --(3.322,2.929)--(3.327,2.931)--(3.333,2.933)--(3.338,2.936)--(3.344,2.938)--(3.349,2.941)%
  --(3.354,2.943)--(3.360,2.946)--(3.365,2.948)--(3.370,2.950)--(3.376,2.953)--(3.381,2.955)%
  --(3.387,2.958)--(3.392,2.960)--(3.397,2.963)--(3.403,2.965)--(3.408,2.967)--(3.414,2.970)%
  --(3.419,2.972)--(3.424,2.975)--(3.430,2.977)--(3.435,2.980)--(3.440,2.982)--(3.446,2.985)%
  --(3.451,2.987)--(3.457,2.989)--(3.462,2.992)--(3.467,2.994)--(3.473,2.997)--(3.478,2.999)%
  --(3.484,3.002)--(3.489,3.004)--(3.494,3.006)--(3.500,3.009)--(3.505,3.011)--(3.510,3.014)%
  --(3.516,3.016)--(3.521,3.019)--(3.527,3.021)--(3.532,3.023)--(3.537,3.026)--(3.543,3.028)%
  --(3.548,3.031)--(3.554,3.033)--(3.559,3.036)--(3.564,3.038)--(3.570,3.040)--(3.575,3.043)%
  --(3.581,3.045)--(3.586,3.048)--(3.591,3.050)--(3.597,3.053)--(3.602,3.055)--(3.607,3.057)%
  --(3.613,3.060)--(3.618,3.062)--(3.624,3.065)--(3.629,3.067)--(3.634,3.070)--(3.640,3.072)%
  --(3.645,3.074)--(3.651,3.077)--(3.656,3.079)--(3.661,3.082)--(3.667,3.084)--(3.672,3.087)%
  --(3.677,3.089)--(3.683,3.091)--(3.688,3.094)--(3.694,3.096)--(3.699,3.099)--(3.704,3.101)%
  --(3.710,3.104)--(3.715,3.106)--(3.721,3.108)--(3.726,3.111)--(3.731,3.113)--(3.737,3.116)%
  --(3.742,3.118)--(3.747,3.121)--(3.753,3.123)--(3.758,3.125)--(3.764,3.128)--(3.769,3.130)%
  --(3.774,3.133)--(3.780,3.135)--(3.785,3.138)--(3.791,3.140)--(3.796,3.142)--(3.801,3.145)%
  --(3.807,3.147)--(3.812,3.150)--(3.817,3.152)--(3.823,3.155)--(3.828,3.157)--(3.834,3.159)%
  --(3.839,3.162)--(3.844,3.164)--(3.850,3.167)--(3.855,3.169)--(3.861,3.172)--(3.866,3.174)%
  --(3.871,3.176)--(3.877,3.179)--(3.882,3.181)--(3.888,3.184)--(3.893,3.186)--(3.898,3.189)%
  --(3.904,3.191)--(3.909,3.193)--(3.914,3.196)--(3.920,3.198)--(3.925,3.201)--(3.931,3.203)%
  --(3.936,3.206)--(3.941,3.208)--(3.947,3.210)--(3.952,3.213)--(3.958,3.215)--(3.963,3.218)%
  --(3.968,3.220)--(3.974,3.223)--(3.979,3.225)--(3.984,3.227)--(3.990,3.230)--(3.995,3.232)%
  --(4.001,3.235)--(4.006,3.237)--(4.011,3.240)--(4.017,3.242)--(4.022,3.244)--(4.028,3.247)%
  --(4.033,3.249)--(4.038,3.252)--(4.044,3.254)--(4.049,3.257)--(4.054,3.259)--(4.060,3.261)%
  --(4.065,3.264)--(4.071,3.266)--(4.076,3.269)--(4.081,3.271)--(4.087,3.274)--(4.092,3.276)%
  --(4.098,3.278)--(4.103,3.281)--(4.108,3.283)--(4.114,3.286)--(4.119,3.288)--(4.124,3.291)%
  --(4.130,3.293)--(4.135,3.295)--(4.141,3.298)--(4.146,3.300)--(4.151,3.303)--(4.157,3.305)%
  --(4.162,3.308)--(4.168,3.310)--(4.173,3.312)--(4.178,3.315)--(4.184,3.317)--(4.189,3.320)%
  --(4.195,3.322)--(4.200,3.325)--(4.205,3.327)--(4.211,3.329)--(4.216,3.332)--(4.221,3.334)%
  --(4.227,3.337)--(4.232,3.339)--(4.238,3.342)--(4.243,3.344)--(4.248,3.346)--(4.254,3.349)%
  --(4.259,3.351)--(4.265,3.354)--(4.270,3.356)--(4.275,3.359)--(4.281,3.361)--(4.286,3.363)%
  --(4.291,3.366)--(4.297,3.368)--(4.302,3.371)--(4.308,3.373)--(4.313,3.376)--(4.318,3.378)%
  --(4.324,3.380)--(4.329,3.383)--(4.335,3.385)--(4.340,3.388)--(4.345,3.390)--(4.351,3.393)%
  --(4.356,3.395)--(4.361,3.397)--(4.367,3.400)--(4.372,3.402)--(4.378,3.405)--(4.383,3.407)%
  --(4.388,3.410)--(4.394,3.412)--(4.399,3.414)--(4.405,3.417)--(4.410,3.419)--(4.415,3.422)%
  --(4.421,3.424)--(4.426,3.427)--(4.432,3.429)--(4.437,3.431)--(4.442,3.434)--(4.448,3.436)%
  --(4.453,3.439)--(4.458,3.441)--(4.464,3.444)--(4.469,3.446)--(4.475,3.448)--(4.480,3.451)%
  --(4.485,3.453)--(4.491,3.456)--(4.496,3.458)--(4.502,3.461)--(4.507,3.463)--(4.512,3.465)%
  --(4.518,3.468)--(4.523,3.470)--(4.528,3.473)--(4.534,3.475)--(4.539,3.478)--(4.545,3.480)%
  --(4.550,3.482)--(4.555,3.485)--(4.561,3.487)--(4.566,3.490)--(4.572,3.492)--(4.577,3.495)%
  --(4.582,3.497)--(4.588,3.499)--(4.593,3.502)--(4.598,3.504)--(4.604,3.507)--(4.609,3.509)%
  --(4.615,3.512)--(4.620,3.514)--(4.625,3.516)--(4.631,3.519)--(4.636,3.521)--(4.642,3.524)%
  --(4.647,3.526)--(4.652,3.529)--(4.658,3.531)--(4.663,3.533)--(4.668,3.536)--(4.674,3.538)%
  --(4.679,3.541)--(4.685,3.543)--(4.690,3.546)--(4.695,3.548)--(4.701,3.550)--(4.706,3.553)%
  --(4.712,3.555)--(4.717,3.558)--(4.722,3.560)--(4.728,3.563)--(4.733,3.565)--(4.739,3.567)%
  --(4.744,3.570)--(4.749,3.572)--(4.755,3.575)--(4.760,3.577)--(4.765,3.580)--(4.771,3.582)%
  --(4.776,3.584)--(4.782,3.587)--(4.787,3.589)--(4.792,3.592)--(4.798,3.594)--(4.803,3.597)%
  --(4.809,3.599)--(4.814,3.602)--(4.819,3.604)--(4.825,3.606)--(4.830,3.609)--(4.835,3.611)%
  --(4.841,3.614)--(4.846,3.616)--(4.852,3.619)--(4.857,3.621)--(4.862,3.623)--(4.868,3.626)%
  --(4.873,3.628)--(4.879,3.631)--(4.884,3.633)--(4.889,3.636)--(4.895,3.638)--(4.900,3.640)%
  --(4.905,3.643)--(4.911,3.645)--(4.916,3.648)--(4.922,3.650)--(4.927,3.653)--(4.932,3.655)%
  --(4.938,3.657)--(4.943,3.660)--(4.949,3.662)--(4.954,3.665)--(4.959,3.667)--(4.965,3.670)%
  --(4.970,3.672)--(4.975,3.674)--(4.981,3.677)--(4.986,3.679)--(4.992,3.682)--(4.997,3.684)%
  --(5.002,3.687)--(5.008,3.689)--(5.013,3.691)--(5.019,3.694)--(5.024,3.696)--(5.029,3.699)%
  --(5.035,3.701)--(5.040,3.704)--(5.046,3.706)--(5.051,3.708)--(5.056,3.711)--(5.062,3.713)%
  --(5.067,3.716)--(5.072,3.718)--(5.078,3.721)--(5.083,3.723)--(5.089,3.725)--(5.094,3.728)%
  --(5.099,3.730)--(5.105,3.733)--(5.110,3.735)--(5.116,3.738)--(5.121,3.740)--(5.126,3.742)%
  --(5.132,3.745)--(5.137,3.747)--(5.142,3.750)--(5.148,3.752)--(5.153,3.755)--(5.159,3.757)%
  --(5.164,3.759)--(5.169,3.762)--(5.175,3.764)--(5.180,3.767)--(5.186,3.769)--(5.191,3.772)%
  --(5.196,3.774)--(5.202,3.776)--(5.207,3.779)--(5.212,3.781)--(5.218,3.784)--(5.223,3.786)%
  --(5.229,3.789)--(5.234,3.791)--(5.239,3.793)--(5.245,3.796)--(5.250,3.798)--(5.256,3.801)%
  --(5.261,3.803)--(5.266,3.806)--(5.272,3.808)--(5.277,3.810)--(5.282,3.813)--(5.288,3.815)%
  --(5.293,3.818)--(5.299,3.820)--(5.304,3.823)--(5.309,3.825)--(5.315,3.827)--(5.320,3.830)%
  --(5.326,3.832)--(5.331,3.835)--(5.336,3.837)--(5.342,3.840)--(5.347,3.842)--(5.353,3.844)%
  --(5.358,3.847)--(5.363,3.849)--(5.369,3.852)--(5.374,3.854)--(5.379,3.857)--(5.385,3.859)%
  --(5.390,3.861)--(5.396,3.864)--(5.401,3.866)--(5.406,3.869)--(5.412,3.871)--(5.417,3.874)%
  --(5.423,3.876)--(5.428,3.878)--(5.433,3.881)--(5.439,3.883)--(5.444,3.886)--(5.449,3.888)%
  --(5.455,3.891)--(5.460,3.893)--(5.466,3.895)--(5.471,3.898)--(5.476,3.900)--(5.482,3.903)%
  --(5.487,3.905)--(5.493,3.908)--(5.498,3.910)--(5.503,3.912)--(5.509,3.915)--(5.514,3.917)%
  --(5.519,3.920)--(5.525,3.922)--(5.530,3.925)--(5.536,3.927)--(5.541,3.929)--(5.546,3.932)%
  --(5.552,3.934)--(5.557,3.937)--(5.563,3.939)--(5.568,3.942)--(5.573,3.944)--(5.579,3.946)%
  --(5.584,3.949)--(5.590,3.951)--(5.595,3.954)--(5.600,3.956)--(5.606,3.959)--(5.611,3.961)%
  --(5.616,3.963)--(5.622,3.966)--(5.627,3.968)--(5.633,3.971)--(5.638,3.973)--(5.643,3.976)%
  --(5.649,3.978)--(5.654,3.980)--(5.660,3.983)--(5.665,3.985)--(5.670,3.988)--(5.676,3.990)%
  --(5.681,3.993)--(5.686,3.995)--(5.692,3.997)--(5.697,4.000)--(5.703,4.002)--(5.708,4.005)%
  --(5.713,4.007)--(5.719,4.010)--(5.724,4.012)--(5.730,4.014)--(5.735,4.017)--(5.740,4.019)%
  --(5.746,4.022)--(5.751,4.024)--(5.756,4.027)--(5.762,4.029)--(5.767,4.031)--(5.773,4.034)%
  --(5.778,4.036)--(5.783,4.039)--(5.789,4.041)--(5.794,4.044)--(5.800,4.046)--(5.805,4.048)%
  --(5.810,4.051)--(5.816,4.053)--(5.821,4.056)--(5.826,4.058)--(5.832,4.061)--(5.837,4.063)%
  --(5.843,4.065)--(5.848,4.068)--(5.853,4.070)--(5.859,4.073)--(5.864,4.075)--(5.870,4.078)%
  --(5.875,4.080)--(5.880,4.082)--(5.886,4.085)--(5.891,4.087)--(5.897,4.090)--(5.902,4.092)%
  --(5.907,4.095)--(5.913,4.097)--(5.918,4.099)--(5.923,4.102)--(5.929,4.104)--(5.934,4.107)%
  --(5.940,4.109)--(5.945,4.112)--(5.950,4.114)--(5.956,4.116)--(5.961,4.119)--(5.967,4.121)%
  --(5.972,4.124)--(5.977,4.126)--(5.983,4.129)--(5.988,4.131)--(5.993,4.133)--(5.999,4.136)%
  --(6.004,4.138)--(6.010,4.141)--(6.015,4.143)--(6.020,4.146)--(6.026,4.148)--(6.031,4.150)%
  --(6.037,4.153)--(6.042,4.155)--(6.047,4.158)--(6.053,4.160)--(6.058,4.163)--(6.063,4.165)%
  --(6.069,4.167)--(6.074,4.170)--(6.080,4.172)--(6.085,4.175)--(6.090,4.177)--(6.096,4.180)%
  --(6.101,4.182)--(6.107,4.184)--(6.112,4.187)--(6.117,4.189)--(6.123,4.192)--(6.128,4.194)%
  --(6.133,4.197)--(6.139,4.199)--(6.144,4.202)--(6.150,4.204)--(6.155,4.206)--(6.160,4.209)%
  --(6.166,4.211)--(6.171,4.214)--(6.177,4.216)--(6.182,4.219)--(6.187,4.221)--(6.193,4.223)%
  --(6.198,4.226)--(6.204,4.228)--(6.209,4.231)--(6.214,4.233)--(6.220,4.236)--(6.225,4.238)%
  --(6.230,4.240)--(6.236,4.243)--(6.241,4.245)--(6.247,4.248)--(6.252,4.250)--(6.257,4.253)%
  --(6.263,4.255)--(6.268,4.257)--(6.274,4.260)--(6.279,4.262)--(6.284,4.265)--(6.290,4.267)%
  --(6.295,4.270)--(6.300,4.272)--(6.306,4.274)--(6.311,4.277)--(6.317,4.279)--(6.322,4.282)%
  --(6.327,4.284)--(6.333,4.287)--(6.338,4.289)--(6.344,4.291)--(6.349,4.294)--(6.354,4.296)%
  --(6.360,4.299)--(6.365,4.301)--(6.370,4.304)--(6.376,4.306)--(6.381,4.308)--(6.387,4.311)%
  --(6.392,4.313)--(6.397,4.316)--(6.403,4.318)--(6.408,4.321)--(6.414,4.323)--(6.419,4.325)%
  --(6.424,4.328)--(6.430,4.330)--(6.435,4.333)--(6.441,4.335)--(6.446,4.338)--(6.451,4.340)%
  --(6.457,4.342)--(6.462,4.345)--(6.467,4.347)--(6.473,4.350)--(6.478,4.352)--(6.484,4.355)%
  --(6.489,4.357)--(6.494,4.359)--(6.500,4.362)--(6.505,4.364)--(6.511,4.367)--(6.516,4.369)%
  --(6.521,4.372)--(6.527,4.374)--(6.532,4.376)--(6.537,4.379)--(6.543,4.381)--(6.548,4.384)%
  --(6.554,4.386)--(6.559,4.389)--(6.564,4.391)--(6.570,4.393)--(6.575,4.396)--(6.581,4.398)%
  --(6.586,4.401)--(6.591,4.403)--(6.597,4.406)--(6.602,4.408)--(6.607,4.410)--(6.613,4.413)%
  --(6.618,4.415)--(6.624,4.418)--(6.629,4.420)--(6.634,4.423)--(6.640,4.425)--(6.645,4.427)%
  --(6.651,4.430)--(6.656,4.432)--(6.661,4.435)--(6.667,4.437)--(6.672,4.440)--(6.677,4.442)%
  --(6.683,4.444)--(6.688,4.447)--(6.694,4.449)--(6.699,4.452)--(6.704,4.454)--(6.710,4.457)%
  --(6.715,4.459)--(6.721,4.461)--(6.726,4.464)--(6.731,4.466)--(6.737,4.469)--(6.742,4.471)%
  --(6.748,4.474)--(6.753,4.476)--(6.758,4.478)--(6.764,4.481)--(6.769,4.483)--(6.774,4.486)%
  --(6.780,4.488)--(6.785,4.491)--(6.791,4.493)--(6.796,4.495)--(6.801,4.498)--(6.807,4.500)%
  --(6.812,4.503)--(6.818,4.505)--(6.823,4.508)--(6.828,4.510)--(6.834,4.512)--(6.839,4.515)%
  --(6.844,4.517)--(6.850,4.520)--(6.855,4.522)--(6.861,4.525)--(6.866,4.527)--(6.871,4.529)%
  --(6.877,4.532)--(6.882,4.534)--(6.888,4.537)--(6.893,4.539)--(6.898,4.542)--(6.904,4.544)%
  --(6.909,4.546)--(6.914,4.549)--(6.920,4.551)--(6.925,4.554)--(6.931,4.556)--(6.936,4.559)%
  --(6.941,4.561)--(6.947,4.563)--(6.952,4.566)--(6.958,4.568)--(6.963,4.571)--(6.968,4.573)%
  --(6.974,4.576)--(6.979,4.578)--(6.984,4.580)--(6.990,4.583)--(6.995,4.585)--(7.001,4.588)%
  --(7.006,4.590)--(7.011,4.593)--(7.017,4.595)--(7.022,4.597)--(7.028,4.600)--(7.033,4.602)%
  --(7.038,4.605)--(7.044,4.607)--(7.049,4.610)--(7.055,4.612)--(7.060,4.614)--(7.065,4.617)%
  --(7.071,4.619)--(7.076,4.622)--(7.081,4.624)--(7.087,4.627)--(7.092,4.629)--(7.098,4.631)%
  --(7.103,4.634)--(7.108,4.636)--(7.114,4.639)--(7.119,4.641)--(7.125,4.644)--(7.130,4.646)%
  --(7.135,4.648)--(7.141,4.651)--(7.146,4.653)--(7.151,4.656)--(7.157,4.658)--(7.162,4.661)%
  --(7.168,4.663)--(7.173,4.665)--(7.178,4.668)--(7.184,4.670)--(7.189,4.673)--(7.195,4.675)%
  --(7.200,4.678)--(7.205,4.680)--(7.211,4.682)--(7.216,4.685)--(7.221,4.687)--(7.227,4.690)%
  --(7.232,4.692)--(7.238,4.695)--(7.243,4.697)--(7.248,4.699)--(7.254,4.702)--(7.259,4.704)%
  --(7.265,4.707)--(7.270,4.709)--(7.275,4.712)--(7.281,4.714)--(7.286,4.716)--(7.291,4.719)%
  --(7.297,4.721)--(7.302,4.724)--(7.308,4.726)--(7.313,4.729)--(7.318,4.731)--(7.324,4.733)%
  --(7.329,4.736)--(7.335,4.738)--(7.340,4.741)--(7.345,4.743)--(7.351,4.746)--(7.356,4.748)%
  --(7.362,4.750)--(7.367,4.753)--(7.372,4.755)--(7.378,4.758)--(7.383,4.760)--(7.388,4.763)%
  --(7.394,4.765)--(7.399,4.767)--(7.405,4.770)--(7.410,4.772)--(7.415,4.775)--(7.421,4.777)%
  --(7.426,4.780)--(7.432,4.782)--(7.437,4.784)--(7.442,4.787)--(7.448,4.789)--(7.453,4.792)%
  --(7.458,4.794)--(7.464,4.797)--(7.469,4.799)--(7.475,4.801)--(7.480,4.804)--(7.485,4.806)%
  --(7.491,4.809)--(7.496,4.811)--(7.502,4.814)--(7.507,4.816)--(7.512,4.819)--(7.518,4.821)%
  --(7.523,4.823)--(7.528,4.826)--(7.534,4.828)--(7.539,4.831)--(7.545,4.833)--(7.550,4.836)%
  --(7.555,4.838)--(7.561,4.840)--(7.566,4.843)--(7.572,4.845)--(7.577,4.848)--(7.582,4.850)%
  --(7.588,4.853)--(7.593,4.855)--(7.599,4.857)--(7.604,4.860)--(7.609,4.862)--(7.615,4.865)%
  --(7.620,4.867)--(7.625,4.870)--(7.631,4.872)--(7.636,4.874)--(7.642,4.877)--(7.647,4.879)%
  --(7.652,4.882)--(7.658,4.884)--(7.663,4.887)--(7.669,4.889)--(7.674,4.891)--(7.679,4.894)%
  --(7.685,4.896)--(7.690,4.899)--(7.695,4.901)--(7.701,4.904)--(7.706,4.906)--(7.712,4.908)%
  --(7.717,4.911)--(7.722,4.913)--(7.728,4.916)--(7.733,4.918)--(7.739,4.921)--(7.744,4.923)%
  --(7.749,4.925)--(7.755,4.928)--(7.760,4.930)--(7.765,4.933)--(7.771,4.935)--(7.776,4.938)%
  --(7.782,4.940)--(7.787,4.942)--(7.792,4.945)--(7.798,4.947)--(7.803,4.950)--(7.809,4.952)%
  --(7.814,4.955)--(7.819,4.957)--(7.825,4.959)--(7.830,4.962)--(7.835,4.964)--(7.841,4.967)%
  --(7.846,4.969)--(7.852,4.972)--(7.857,4.974)--(7.862,4.976)--(7.868,4.979)--(7.873,4.981)%
  --(7.879,4.984)--(7.884,4.986)--(7.889,4.989)--(7.895,4.991)--(7.900,4.993)--(7.906,4.996)%
  --(7.911,4.998)--(7.916,5.001)--(7.922,5.003)--(7.927,5.006)--(7.932,5.008)--(7.938,5.010)%
  --(7.943,5.013)--(7.949,5.015)--(7.954,5.018)--(7.959,5.020)--(7.965,5.023)--(7.970,5.025)%
  --(7.976,5.027)--(7.981,5.030)--(7.986,5.032)--(7.992,5.035)--(7.997,5.037)--(8.002,5.040)%
  --(8.008,5.042)--(8.013,5.044)--(8.019,5.047)--(8.024,5.049)--(8.029,5.052)--(8.035,5.054)%
  --(8.040,5.057)--(8.046,5.059)--(8.051,5.061)--(8.056,5.064)--(8.062,5.066)--(8.067,5.069)%
  --(8.072,5.071)--(8.078,5.074)--(8.083,5.076)--(8.089,5.078)--(8.094,5.081)--(8.099,5.083)%
  --(8.105,5.086)--(8.110,5.088)--(8.116,5.091)--(8.121,5.093)--(8.126,5.095)--(8.132,5.098)%
  --(8.137,5.100)--(8.142,5.103)--(8.148,5.105)--(8.153,5.108)--(8.159,5.110)--(8.164,5.112)%
  --(8.169,5.115)--(8.175,5.117)--(8.180,5.120)--(8.186,5.122)--(8.191,5.125)--(8.196,5.127)%
  --(8.202,5.129)--(8.207,5.132)--(8.213,5.134)--(8.218,5.137)--(8.223,5.139)--(8.229,5.142)%
  --(8.234,5.144)--(8.239,5.146)--(8.245,5.149)--(8.250,5.151)--(8.256,5.154)--(8.261,5.156)%
  --(8.266,5.159)--(8.272,5.161)--(8.277,5.163)--(8.283,5.166)--(8.288,5.168)--(8.293,5.171)%
  --(8.299,5.173)--(8.304,5.176)--(8.309,5.178)--(8.315,5.180)--(8.320,5.183)--(8.326,5.185)%
  --(8.331,5.188)--(8.336,5.190)--(8.342,5.193)--(8.347,5.195)--(8.353,5.197)--(8.358,5.200)%
  --(8.363,5.202)--(8.369,5.205)--(8.374,5.207)--(8.379,5.210)--(8.385,5.212)--(8.390,5.214)%
  --(8.396,5.217)--(8.401,5.219)--(8.406,5.222)--(8.412,5.224)--(8.417,5.227)--(8.423,5.229)%
  --(8.428,5.231)--(8.433,5.234)--(8.439,5.236)--(8.444,5.239)--(8.450,5.241)--(8.455,5.244)%
  --(8.460,5.246)--(8.466,5.248)--(8.471,5.251)--(8.476,5.253)--(8.482,5.256)--(8.487,5.258)%
  --(8.493,5.261)--(8.498,5.263)--(8.503,5.265)--(8.509,5.268)--(8.514,5.270)--(8.520,5.273)%
  --(8.525,5.275)--(8.530,5.278)--(8.536,5.280)--(8.541,5.282)--(8.546,5.285)--(8.552,5.287)%
  --(8.557,5.290)--(8.563,5.292)--(8.568,5.295)--(8.573,5.297)--(8.579,5.299)--(8.584,5.302)%
  --(8.590,5.304)--(8.595,5.307)--(8.600,5.309)--(8.606,5.312)--(8.611,5.314)--(8.616,5.316)%
  --(8.622,5.319)--(8.627,5.321)--(8.633,5.324)--(8.638,5.326)--(8.643,5.329)--(8.649,5.331)%
  --(8.654,5.333)--(8.660,5.336)--(8.665,5.338)--(8.670,5.341)--(8.676,5.343)--(8.681,5.346)%
  --(8.686,5.348)--(8.692,5.350)--(8.697,5.353)--(8.703,5.355)--(8.708,5.358)--(8.713,5.360)%
  --(8.719,5.363)--(8.724,5.365)--(8.730,5.367)--(8.735,5.370)--(8.740,5.372)--(8.746,5.375)%
  --(8.751,5.377)--(8.757,5.380)--(8.762,5.382)--(8.767,5.384)--(8.773,5.387)--(8.778,5.389)%
  --(8.783,5.392)--(8.789,5.394)--(8.794,5.397)--(8.800,5.399)--(8.805,5.401)--(8.810,5.404)%
  --(8.816,5.406)--(8.821,5.409)--(8.827,5.411)--(8.832,5.414)--(8.837,5.416)--(8.843,5.418)%
  --(8.848,5.421)--(8.853,5.423)--(8.859,5.426)--(8.864,5.428)--(8.870,5.431)--(8.875,5.433)%
  --(8.880,5.436)--(8.886,5.438)--(8.891,5.440)--(8.897,5.443)--(8.902,5.445)--(8.907,5.448)%
  --(8.913,5.450)--(8.918,5.453)--(8.923,5.455)--(8.929,5.457)--(8.934,5.460)--(8.940,5.462)%
  --(8.945,5.465)--(8.950,5.467)--(8.956,5.470)--(8.961,5.472)--(8.967,5.474)--(8.972,5.477)%
  --(8.977,5.479)--(8.983,5.482)--(8.988,5.484)--(8.993,5.487)--(8.999,5.489)--(9.004,5.491)%
  --(9.010,5.494)--(9.015,5.496)--(9.020,5.499)--(9.026,5.501)--(9.031,5.504)--(9.037,5.506)%
  --(9.042,5.508)--(9.047,5.511)--(9.053,5.513)--(9.058,5.516)--(9.064,5.518)--(9.069,5.521)%
  --(9.074,5.523)--(9.080,5.525)--(9.085,5.528)--(9.090,5.530)--(9.096,5.533)--(9.101,5.535)%
  --(9.107,5.538)--(9.112,5.540)--(9.117,5.542)--(9.123,5.545)--(9.128,5.547)--(9.134,5.550)%
  --(9.139,5.552)--(9.144,5.555)--(9.150,5.557)--(9.155,5.559)--(9.160,5.562)--(9.166,5.564)%
  --(9.171,5.567)--(9.177,5.569)--(9.182,5.572)--(9.187,5.574)--(9.193,5.576)--(9.198,5.579)%
  --(9.204,5.581)--(9.209,5.584)--(9.214,5.586)--(9.220,5.589)--(9.225,5.591)--(9.230,5.593)%
  --(9.236,5.596)--(9.241,5.598)--(9.247,5.601)--(9.252,5.603)--(9.257,5.606)--(9.263,5.608)%
  --(9.268,5.610)--(9.274,5.613)--(9.279,5.615)--(9.284,5.618)--(9.290,5.620)--(9.295,5.623)%
  --(9.301,5.625)--(9.306,5.627)--(9.311,5.630)--(9.317,5.632)--(9.322,5.635)--(9.327,5.637)%
  --(9.333,5.640)--(9.338,5.642)--(9.344,5.644)--(9.349,5.647)--(9.354,5.649)--(9.360,5.652)%
  --(9.365,5.654)--(9.371,5.657)--(9.376,5.659)--(9.381,5.661)--(9.387,5.664)--(9.392,5.666)%
  --(9.397,5.669)--(9.403,5.671)--(9.408,5.674)--(9.414,5.676)--(9.419,5.678)--(9.424,5.681)%
  --(9.430,5.683)--(9.435,5.686)--(9.441,5.688)--(9.446,5.691)--(9.451,5.693)--(9.457,5.695)%
  --(9.462,5.698)--(9.467,5.700)--(9.473,5.703)--(9.478,5.705)--(9.484,5.708)--(9.489,5.710)%
  --(9.494,5.712)--(9.500,5.715)--(9.505,5.717)--(9.511,5.720)--(9.516,5.722)--(9.521,5.725)%
  --(9.527,5.727)--(9.532,5.729)--(9.537,5.732)--(9.543,5.734)--(9.548,5.737)--(9.554,5.739)%
  --(9.559,5.742)--(9.564,5.744)--(9.570,5.746)--(9.575,5.749)--(9.581,5.751)--(9.586,5.754)%
  --(9.591,5.756)--(9.597,5.759)--(9.602,5.761)--(9.608,5.763)--(9.613,5.766)--(9.618,5.768)%
  --(9.624,5.771)--(9.629,5.773)--(9.634,5.776)--(9.640,5.778)--(9.645,5.780)--(9.651,5.783)%
  --(9.656,5.785)--(9.661,5.788)--(9.667,5.790)--(9.672,5.793)--(9.678,5.795)--(9.683,5.797)%
  --(9.688,5.800)--(9.694,5.802)--(9.699,5.805)--(9.704,5.807)--(9.710,5.810)--(9.715,5.812)%
  --(9.721,5.814)--(9.726,5.817)--(9.731,5.819)--(9.737,5.822)--(9.742,5.824)--(9.748,5.827)%
  --(9.753,5.829)--(9.758,5.831)--(9.764,5.834)--(9.769,5.836)--(9.774,5.839)--(9.780,5.841)%
  --(9.785,5.844)--(9.791,5.846)--(9.796,5.848)--(9.801,5.851)--(9.807,5.853)--(9.812,5.856)%
  --(9.818,5.858)--(9.823,5.861)--(9.828,5.863)--(9.834,5.865)--(9.839,5.868)--(9.844,5.870)%
  --(9.850,5.873)--(9.855,5.875)--(9.861,5.878)--(9.866,5.880)--(9.871,5.882)--(9.877,5.885)%
  --(9.882,5.887)--(9.888,5.890)--(9.893,5.892)--(9.898,5.895)--(9.904,5.897)--(9.909,5.899)%
  --(9.915,5.902)--(9.920,5.904)--(9.925,5.907)--(9.931,5.909)--(9.936,5.912)--(9.941,5.914)%
  --(9.947,5.916)--(9.952,5.919)--(9.958,5.921)--(9.963,5.924)--(9.968,5.926)--(9.974,5.929)%
  --(9.979,5.931)--(9.985,5.933)--(9.990,5.936)--(9.995,5.938)--(10.001,5.941)--(10.006,5.943)%
  --(10.011,5.946)--(10.017,5.948)--(10.022,5.950)--(10.028,5.953)--(10.033,5.955)--(10.038,5.958)%
  --(10.044,5.960)--(10.049,5.963)--(10.055,5.965)--(10.060,5.967)--(10.065,5.970)--(10.071,5.972)%
  --(10.076,5.975)--(10.081,5.977)--(10.087,5.980)--(10.092,5.982)--(10.098,5.984)--(10.103,5.987)%
  --(10.108,5.989)--(10.114,5.992)--(10.119,5.994)--(10.125,5.997)--(10.130,5.999)--(10.135,6.001)%
  --(10.141,6.004)--(10.146,6.006)--(10.151,6.009)--(10.157,6.011)--(10.162,6.014)--(10.168,6.016)%
  --(10.173,6.018)--(10.178,6.021)--(10.184,6.023)--(10.189,6.026)--(10.195,6.028)--(10.200,6.031)%
  --(10.205,6.033)--(10.211,6.036)--(10.216,6.038)--(10.222,6.040)--(10.227,6.043)--(10.232,6.045)%
  --(10.238,6.048)--(10.243,6.050)--(10.248,6.053)--(10.254,6.055)--(10.259,6.057)--(10.265,6.060)%
  --(10.270,6.062)--(10.275,6.065)--(10.281,6.067)--(10.286,6.070)--(10.292,6.072)--(10.297,6.074)%
  --(10.302,6.077)--(10.308,6.079)--(10.313,6.082)--(10.318,6.084)--(10.324,6.087)--(10.329,6.089)%
  --(10.335,6.091)--(10.340,6.094)--(10.345,6.096)--(10.351,6.099)--(10.356,6.101)--(10.362,6.104)%
  --(10.367,6.106)--(10.372,6.108)--(10.378,6.111)--(10.383,6.113)--(10.388,6.116)--(10.394,6.118)%
  --(10.399,6.121)--(10.405,6.123)--(10.410,6.125)--(10.415,6.128)--(10.421,6.130)--(10.426,6.133)%
  --(10.432,6.135)--(10.437,6.138)--(10.442,6.140)--(10.448,6.142)--(10.453,6.145)--(10.459,6.147)%
  --(10.464,6.150)--(10.469,6.152)--(10.475,6.155)--(10.480,6.157)--(10.485,6.159)--(10.491,6.162)%
  --(10.496,6.164)--(10.502,6.167)--(10.507,6.169)--(10.512,6.172)--(10.518,6.174)--(10.523,6.176)%
  --(10.529,6.179)--(10.534,6.181)--(10.539,6.184)--(10.545,6.186)--(10.550,6.189)--(10.555,6.191)%
  --(10.561,6.193)--(10.566,6.196)--(10.572,6.198)--(10.577,6.201)--(10.582,6.203)--(10.588,6.206)%
  --(10.593,6.208)--(10.599,6.210)--(10.604,6.213)--(10.609,6.215)--(10.615,6.218)--(10.620,6.220)%
  --(10.625,6.223)--(10.631,6.225)--(10.636,6.227)--(10.642,6.230)--(10.647,6.232)--(10.652,6.235)%
  --(10.658,6.237)--(10.663,6.240)--(10.669,6.242)--(10.674,6.244)--(10.679,6.247)--(10.685,6.249)%
  --(10.690,6.252)--(10.695,6.254)--(10.701,6.257)--(10.706,6.259)--(10.712,6.261)--(10.717,6.264)%
  --(10.722,6.266)--(10.728,6.269)--(10.733,6.271)--(10.739,6.274)--(10.744,6.276)--(10.749,6.278)%
  --(10.755,6.281)--(10.760,6.283)--(10.766,6.286)--(10.771,6.288)--(10.776,6.291)--(10.782,6.293)%
  --(10.787,6.295)--(10.792,6.298)--(10.798,6.300)--(10.803,6.303)--(10.809,6.305)--(10.814,6.308)%
  --(10.819,6.310)--(10.825,6.312)--(10.830,6.315)--(10.836,6.317)--(10.841,6.320)--(10.846,6.322)%
  --(10.852,6.325)--(10.857,6.327)--(10.862,6.329)--(10.868,6.332)--(10.873,6.334)--(10.879,6.337)%
  --(10.884,6.339)--(10.889,6.342)--(10.895,6.344)--(10.900,6.346)--(10.906,6.349)--(10.911,6.351)%
  --(10.916,6.354)--(10.922,6.356)--(10.927,6.359)--(10.932,6.361)--(10.938,6.363)--(10.943,6.366)%
  --(10.949,6.368)--(10.954,6.371)--(10.959,6.373)--(10.965,6.376)--(10.970,6.378)--(10.976,6.380)%
  --(10.981,6.383)--(10.986,6.385)--(10.992,6.388)--(10.997,6.390)--(11.002,6.393)--(11.008,6.395)%
  --(11.013,6.397)--(11.019,6.400)--(11.024,6.402)--(11.029,6.405)--(11.035,6.407)--(11.040,6.410)%
  --(11.046,6.412)--(11.051,6.414)--(11.056,6.417)--(11.062,6.419)--(11.067,6.422)--(11.073,6.424)%
  --(11.078,6.427)--(11.083,6.429)--(11.089,6.431)--(11.094,6.434)--(11.099,6.436)--(11.105,6.439)%
  --(11.110,6.441)--(11.116,6.444)--(11.121,6.446)--(11.126,6.448)--(11.132,6.451)--(11.137,6.453)%
  --(11.143,6.456)--(11.148,6.458)--(11.153,6.461)--(11.159,6.463)--(11.164,6.465)--(11.169,6.468)%
  --(11.175,6.470)--(11.180,6.473)--(11.186,6.475)--(11.191,6.478)--(11.196,6.480)--(11.202,6.482)%
  --(11.207,6.485)--(11.213,6.487)--(11.218,6.490)--(11.223,6.492)--(11.229,6.495)--(11.234,6.497)%
  --(11.239,6.499)--(11.245,6.502)--(11.250,6.504)--(11.256,6.507)--(11.261,6.509)--(11.266,6.512)%
  --(11.272,6.514)--(11.277,6.516)--(11.283,6.519)--(11.288,6.521)--(11.293,6.524)--(11.299,6.526)%
  --(11.304,6.529)--(11.310,6.531)--(11.315,6.533)--(11.320,6.536)--(11.326,6.538)--(11.331,6.541)%
  --(11.336,6.543)--(11.342,6.546)--(11.347,6.548)--(11.353,6.550)--(11.358,6.553)--(11.363,6.555)%
  --(11.369,6.558)--(11.374,6.560)--(11.380,6.563)--(11.385,6.565)--(11.390,6.567)--(11.396,6.570)%
  --(11.401,6.572)--(11.406,6.575)--(11.412,6.577)--(11.417,6.580)--(11.423,6.582)--(11.428,6.584)%
  --(11.433,6.587)--(11.439,6.589)--(11.444,6.592)--(11.450,6.594)--(11.455,6.597)--(11.460,6.599)%
  --(11.466,6.601)--(11.471,6.604)--(11.476,6.606)--(11.482,6.609)--(11.487,6.611)--(11.493,6.614)%
  --(11.498,6.616)--(11.503,6.618)--(11.509,6.621)--(11.514,6.623)--(11.520,6.626)--(11.525,6.628)%
  --(11.530,6.631)--(11.536,6.633)--(11.541,6.635)--(11.546,6.638)--(11.552,6.640)--(11.557,6.643)%
  --(11.563,6.645)--(11.568,6.648)--(11.573,6.650)--(11.579,6.653)--(11.584,6.655)--(11.590,6.657)%
  --(11.595,6.660)--(11.600,6.662)--(11.606,6.665)--(11.611,6.667)--(11.617,6.670)--(11.622,6.672)%
  --(11.627,6.674)--(11.633,6.677)--(11.638,6.679)--(11.643,6.682)--(11.649,6.684)--(11.654,6.687)%
  --(11.660,6.689)--(11.665,6.691)--(11.670,6.694)--(11.676,6.696)--(11.681,6.699)--(11.687,6.701)%
  --(11.692,6.704)--(11.697,6.706)--(11.703,6.708)--(11.708,6.711)--(11.713,6.713)--(11.719,6.716)%
  --(11.724,6.718)--(11.730,6.721)--(11.735,6.723)--(11.740,6.725)--(11.746,6.728)--(11.751,6.730)%
  --(11.757,6.733)--(11.762,6.735)--(11.767,6.738)--(11.773,6.740)--(11.778,6.742)--(11.783,6.745)%
  --(11.789,6.747)--(11.794,6.750)--(11.800,6.752)--(11.805,6.755)--(11.810,6.757)--(11.816,6.759)%
  --(11.821,6.762)--(11.827,6.764)--(11.832,6.767)--(11.837,6.769)--(11.843,6.772)--(11.848,6.774)%
  --(11.853,6.776)--(11.859,6.779)--(11.864,6.781)--(11.870,6.784)--(11.875,6.786)--(11.880,6.789)%
  --(11.886,6.791)--(11.891,6.793)--(11.897,6.796)--(11.902,6.798)--(11.907,6.801)--(11.913,6.803)%
  --(11.918,6.806)--(11.924,6.808)--(11.929,6.810)--(11.934,6.813)--(11.940,6.815)--(11.945,6.818)%
  --(11.950,6.820)--(11.956,6.823)--(11.961,6.825)--(11.967,6.827)--(11.972,6.830)--(11.977,6.832)%
  --(11.983,6.835)--(11.988,6.837)--(11.994,6.840)--(11.999,6.842)--(12.004,6.844)--(12.010,6.847)%
  --(12.015,6.849)--(12.020,6.852)--(12.026,6.854)--(12.031,6.857)--(12.037,6.859)--(12.042,6.861)%
  --(12.047,6.864)--(12.053,6.866)--(12.058,6.869)--(12.064,6.871)--(12.069,6.874)--(12.074,6.876)%
  --(12.080,6.878)--(12.085,6.881)--(12.090,6.883)--(12.096,6.886)--(12.101,6.888)--(12.107,6.891)%
  --(12.112,6.893)--(12.117,6.895)--(12.123,6.898)--(12.128,6.900)--(12.134,6.903)--(12.139,6.905)%
  --(12.144,6.908)--(12.150,6.910)--(12.155,6.912)--(12.161,6.915)--(12.166,6.917)--(12.171,6.920)%
  --(12.177,6.922)--(12.182,6.925)--(12.187,6.927)--(12.193,6.929)--(12.198,6.932)--(12.204,6.934)%
  --(12.209,6.937)--(12.214,6.939)--(12.220,6.942)--(12.225,6.944)--(12.231,6.946)--(12.236,6.949)%
  --(12.241,6.951)--(12.247,6.954)--(12.252,6.956)--(12.257,6.959)--(12.263,6.961)--(12.268,6.963)%
  --(12.274,6.966)--(12.279,6.968)--(12.284,6.971)--(12.290,6.973)--(12.295,6.976)--(12.301,6.978)%
  --(12.306,6.980)--(12.311,6.983)--(12.317,6.985)--(12.322,6.988)--(12.327,6.990)--(12.333,6.993)%
  --(12.338,6.995)--(12.344,6.997)--(12.349,7.000)--(12.354,7.002)--(12.360,7.005)--(12.365,7.007)%
  --(12.371,7.010)--(12.376,7.012)--(12.381,7.014)--(12.387,7.017)--(12.392,7.019)--(12.397,7.022)%
  --(12.403,7.024)--(12.408,7.027)--(12.414,7.029)--(12.419,7.031)--(12.424,7.034)--(12.430,7.036)%
  --(12.435,7.039)--(12.441,7.041)--(12.446,7.044)--(12.451,7.046)--(12.457,7.048)--(12.462,7.051)%
  --(12.468,7.053)--(12.473,7.056)--(12.478,7.058)--(12.484,7.061)--(12.489,7.063)--(12.494,7.065)%
  --(12.500,7.068)--(12.505,7.070)--(12.511,7.073)--(12.516,7.075)--(12.521,7.078)--(12.527,7.080)%
  --(12.532,7.082)--(12.538,7.085)--(12.543,7.087)--(12.548,7.090)--(12.554,7.092)--(12.559,7.095)%
  --(12.564,7.097)--(12.570,7.099)--(12.575,7.102)--(12.581,7.104)--(12.586,7.107)--(12.591,7.109)%
  --(12.597,7.112)--(12.602,7.114)--(12.608,7.116)--(12.613,7.119)--(12.618,7.121)--(12.624,7.124)%
  --(12.629,7.126)--(12.634,7.129)--(12.640,7.131)--(12.645,7.133)--(12.651,7.136)--(12.656,7.138)%
  --(12.661,7.141)--(12.667,7.143)--(12.672,7.146)--(12.678,7.148);
\gpcolor{color=gp lt color border}
\gpsetlinewidth{1.00}
\draw[gp path] (1.136,8.691)--(1.136,0.985)--(13.447,0.985)--(13.447,8.691)--cycle;
%% coordinates of the plot area
\gpdefrectangularnode{gp plot 1}{\pgfpoint{1.136cm}{0.985cm}}{\pgfpoint{13.447cm}{8.691cm}}
\end{tikzpicture}
%% gnuplot variables

\end{figure*}


%%%%%%%%%%%%%%%%%%%%%%%%%%%%%%%%%%%%%
\section{Linearização}
%%%%%%%%%%%%%%%%%%%%%%%%%%%%%%%%%%%%%

É comum realizarmos um experimento e obtermos um conjunto de dados que não segue uma tendência linear. Em um experimento de queda livre, por exemplo, a distância percorrida pelo objeto que cai está ligada ao tempo através de
\begin{equation}
	\Delta y = v_0^y t + \frac{a_yt^2}{2}.
\end{equation}
%
Como em geral estamos interessados em extrair dos dados experimentais informações acerca de constantes físicas, podemos recorrer a um processo de regressão. No entanto, não podemos utilizar uma regressão linear neste caso, pois os dados claramente não seguirão uma tendência linear. Podemos recorrer a uma regressão quadrática ou realizar uma \emph{linearização}.

Realizar uma linearização nada mais é do que fazer uma mudança de variáveis. No entanto, nem todos os casos são passíveis de serem linearizados. No caso da queda livre, por exemplo, precisamos garantir que $v_0$ seja muito próxiimo de zero. Nesse caso, podemos escrever
\begin{equation}
	\Delta y = \frac{a_yt^2}{2}.
\end{equation}
%
Fazendo agora a mudança de variáveis $\tau = t^2$, obtemos
\begin{equation}
	\Delta y = \frac{a_y}{2}\tau.
\end{equation}

Comparando a equação acima com a equação da reta $y = A + Bx$, vefificamos as seguintes relações, se considerarmos que o tempo é a variável independente:\footnote{Veja que aqui estamos usando $y$ para duas coisas distintas: um eixo vertical através do qual calculamos o deslocamento $\Delta y$ de um corpo sujeito à gravidade, e o eixo da variável dependente de um gráfico (que é representado no sentido ``da parte inferior da página, para a parte superior'').}
\begin{align}
	y &= \Delta y \\
	A &= 0 \\
	B &= a/2 \\
	x &= \tau.
\end{align}
%
Podemos, portanto, utilizar o processo de linearização associado ao processo de regressão linear para obtermos informações a respeito de constantes físicas mesmo que o conjunto de dados experimentais obtidos não siga uma tendência linear.

%%%%%%%%%%%%%%%%%%%%%%%%%%%%%%%%%%%%%
\section{Linearização e teste de hipóteses}
%%%%%%%%%%%%%%%%%%%%%%%%%%%%%%%%%%%%%

Nos casos em que um fenômeno não segue uma tendência linear, se tivermos um modelo do fenômeno físico considerado, podemos empregar o processo da linearização com o intuito de transformar o conjunto de dados de forma que ele passe a representar uma reta. O sucesso desse procedimento, no entanto, depende de a teoria usada como referência estar correta. Muitas vezes ocorre que não temos conhecimento acerca da teoria que descreve um fenômeno ou temos mais de uma teoria, sem ter certeza de qual é a mais adequada. Podemos então utilizar uma regressão linear como um método de teste, validando uma ou outra teoria com base nos valores obtidos para o coeficiente de dispersão linear $r^2$.

\begin{margintable}
\centering
\begin{tabular}{cc}
\toprule
$\rho$ & $\xi$     \\
\midrule
0.00    & 0.5909   \\ 
0.20    & 0.8310   \\
0.40    & 0.5301   \\
0.60    & 1.0066   \\
0.80    & 1.0345   \\
1.00    & 1.6364   \\
1.20    & 1.5547   \\
1.40    & 2.0911   \\
1.60    & 2.7331   \\
1.80    & 2.9338   \\
2.00    & 3.4851   \\
2.20    & 4.5948   \\
2.40    & 5.3332   \\
2.60    & 6.6722   \\
2.80    & 8.2156   \\
3.00    & 9.8453   \\
3.20    & 12.1483  \\
3.40    & 14.8377  \\
3.60    & 18.2303  \\
\bottomrule
\end{tabular}
\vspace{1mm}
\caption{Dados medidos para $\xi$ em função e $\rho$. \label{TabelaMedidasIniciais}}
\end{margintable}

Digamos que ao fazer uma série de medidas de uma grandeza $\xi$ em função de outra grandeza $\rho$ (Tabela~\ref{TabelaMedidasIniciais}), tenhamos um conjunto que segue uma forma que não é uma reta (Figura~\ref{GraficoMedidas}). Além disso, temos duas teorias, cada uma com as seguintes previsões:
\begin{equation}
	\xi =\begin{cases} \alpha e^{\beta\rho}, \qquad\textrm{Teoria 1} \\
	\alpha\rho^2 + \beta, \qquad\textrm{Teoria 2.} \end{cases}
\end{equation}

\noindent{}Podemos realizar a regressão linear em ambos os casos e comparar os coeficientes $r$ para verificar qual das duas teorias descreve melhor os dados obtidos. Tomando o logaritmo no primeiro caso, temos
\begin{align}
	\ln(\xi) &= \ln(\alpha e^{\beta\rho}) \\
	\ln(\xi) &= \ln(\alpha) + \beta\rho
\end{align}

\begin{figure}
\centering
\caption{Dados medidos de $\xi$ em função de $\rho$. Visivelmente o comportamento é não linear.}
\label{GraficoMedidas}
\input{Graphics/Linearization/DadosMedidos}
\end{figure}

\noindent{}Logo,
\begin{align}
	y &= \ln(\xi) \\
	x &= \rho \\
	A &= \beta \\
	B &= \ln(\alpha)
\end{align}

No segundo caso, temos uma identificação mais simples:
\begin{align}
	y &= \xi \\
	x &= \rho^2 \\
	A &= \alpha \\
	B &= \beta.
\end{align}

Podemos então usar os resultados dessas linearizações para transformar a tabela inicial em duas outras tabelas, cada uma considerando as previsões de cada uma das Teorias e com a esperança de que alguma delas siga o comportamento linear, indicando que a Teoria correspondente tem fundamento.

Baseando-se nas medidas e nos valores transformados dados na Tabela~\ref{TabelaMedidas}, podemos realizar as regressões lineares. Os resultados podem ser vistos nas Figuras~\ref{GraficoTeoria1} e~\ref{GraficoTeoria2}. Concluímos, portanto, que a Teoria 1 descreve melhor os resultados medidos.

\begin{table}\forcerectofloat
\centering
\begin{tabular}{ccccc}
\toprule
\multicolumn{2}{c}{Teoria 1} && \multicolumn{2}{c}{Teoria 2} \\
\cmidrule(lr){1-2} \cmidrule(lr){4-5}
$\rho$  & $\ln(\xi)$    && $\rho^2$   & $\xi$ \\
\midrule
0.00    & -0.5260      &&   0.00     & 0.5909  \\ 
0.20    & -0.1850      &&   0.04     & 0.8310  \\
0.40    & -0.6346      &&   0.16     & 0.5301  \\
0.60    & 0.0065783    &&   0.36     & 1.0066  \\
0.80    & 0.033918     &&   0.64     & 1.0345  \\
1.00    & 0.49249      &&   1.00     & 1.6364  \\
1.20    & 0.44133      &&   1.44     & 1.5547  \\
1.40    & 0.73770      &&   1.96     & 2.0911  \\
1.60    & 1.0055       &&   2.56     & 2.7331  \\
1.80    & 1.0763       &&   3.24     & 2.9338  \\
2.00    & 1.2485       &&   4.00     & 3.4851  \\
2.20    & 1.5249       &&   4.84     & 4.5948  \\
2.40    & 1.6740       &&   5.76     & 5.3332  \\
2.60    & 1.8980       &&   6.76     & 6.6722  \\
2.80    & 2.1060       &&   7.84     & 8.2156  \\
3.00    & 2.2870       &&   9.00     & 9.8453  \\
3.20    & 2.49719      &&   10.24    & 12.1483 \\
3.40    & 2.69717      &&   11.56    & 14.8377 \\
3.60    & 2.90308      &&   12.96    & 18.2303 \\
\bottomrule
\end{tabular}
\vspace{1mm}
\caption{Dados medidos para $\xi$ em função e $\rho$ e os resultados das transformações baseadas nas Teorias 1 e 2. \label{TabelaMedidas}}
\end{table}

%%% Gráficos

\begin{figure}\forcerectofloat
\centering
\input{Graphics/Linearization/DadosTeoria1}
\caption{Dados ajustados segundo as previsões da Teoria 1.\label{GraficoTeoria1}}
\end{figure}

\begin{figure}
\centering
\input{Graphics/Linearization/DadosTeoria2}
\caption{Dados ajustados segundo a Teoria 2.\label{GraficoTeoria2}}
\end{figure}


%%%%%%%%%%%%%%%%%%%%%%%%%%%%%%%%%%%%%%%%%%%%%%%%%%%%%%
\section{Erros nos parâmetros de uma regressão linear}
%%%%%%%%%%%%%%%%%%%%%%%%%%%%%%%%%%%%%%%%%%%%%%%%%%%%%%
\label{Chap:ErrosCoefAB}

Diferentemente da parte teórica, onde assumimos o conhecimento de constantes como a aceleração da gravidade $g$ e calculamos o deslocamento de uma partícula com o passar do tempo, no laboratório podemos medir o tempo e o deslocamento facilmente. O valor da gravidade, por outro lado, é mais difícil de ser determinado. Utilizando uma regressão linear, podemos identificar a relação entre o coeficiente $B$ da equação da reta e $g$. Sabemos, no entanto, que a toda medida temos um erro associado. Qual seria, nesse caso, o erro associado ao valor de $g$? 

Utilizando algumas considerações\cite{Taylor} acerca da distribuição das medidas $y_i$ em torno de seus valores ideais correspondentes, temos que o erro associado aos parâmetros $A$ e $B$ são dados por
\begin{align}
	\delta A &= \xi_y\sqrt{\frac{\sum x_i^2}{\Delta}} \label{Eq:ErrosCoeficientesRegLinear1}\\
	\delta B &= \xi_y\sqrt{\frac{N}{\Delta}}
\end{align}
%
onde $N$ é o número de pontos experimentais e a soma se dá sobre todos os pontos experimentais. Além disso,
\begin{align}
	\xi_y &= \sqrt{\frac{1}{N-2}\sum(y_i-A-B\,x_i)^2} \\
	\Delta &= N\sum x_i^2 - \left(\sum x_i\right)^2. \label{Eq:ErrosCoeficientesRegLinear4}
\end{align}

Podemos então calcular não só o melhor valor associado a uma constante física através dos coeficientes $A$ e $B$ da melhor reta, mas também determinar qual o erro associado a cada uma dessas constantes. Nem sempre temos uma correspondência simples entre uma constante física e um dos parâmetros da equação da reta, nesses casos, ao calcular o valor da constante física, precisamos calcular o erro propagado através da fórmula geral discutida no Capítulo~\ref{Chap:Erros}.

%%%%%%%%%%%%%%%%%%%%%%%%%%%%%%%%%%%%%%%%%%%%%%%%%%%%%%%%%%%%%%%%%%%
\paragraph{Exemplo: Cálculo dos erros dos coeficientes $A$ e $B$}
%%%%%%%%%%%%%%%%%%%%%%%%%%%%%%%%%%%%%%%%%%%%%%%%%%%%%%%%%%%%%%%%%%%

\begin{margintable}[-7cm]
\centering
\begin{tabular}{cccc}
\toprule
$i$	& $x_i$	& $y_{1,i}$	& $(\Xi_i)^2$ \\
\midrule
1	&	\np{0.714}	&	\np{14.577}	&	\np{0.0934} \\ 
2	&	\np{2.693}	&	\np{20.696}	&	\np{0.2687} \\ 
3	&	\np{4.389}	&	\np{25.226}	&	\np{0.0002} \\ 
4	&	\np{4.960}	&	\np{27.449}	&	\np{0.2555} \\ 
5	&	\np{6.245}	&	\np{30.242}	&	\np{0.2879} \\ 
6	&	\np{7.277}	&	\np{33.378}	&	\np{0.2309} \\ 
7	&	\np{7.579}	&	\np{34.195}	&	\np{0.3190} \\ 
8	&	\np{7.719}	&	\np{35.715}	&	\np{0.2887} \\ 
9	&	\np{7.912}	&	\np{35.011}	&	\np{0.5516} \\ 
10	&	\np{8.280}	&	\np{37.529}	&	\np{0.4584} \\ 
11	&	\np{9.034}	&	\np{40.590}	&	\np{2.2134} \\ 
12	&	\np{9.442}	&	\np{39.156}	&	\np{1.3547} \\ 
13	&	\np{10.306}	&	\np{43.238}	&	\np{0.1152} \\ 
14	&	\np{10.572}	&	\np{42.406}	&	\np{1.6548} \\ 
15	&	\np{11.177}	&	\np{44.796}	&	\np{0.4928} \\ 
16	&	\np{15.335}	&	\np{57.611}	&	\np{0.0879} \\ 
17	&	\np{17.023}	&	\np{63.832}	&	\np{0.7863} \\ 
18	&	\np{18.926}	&	\np{68.063}	&	\np{0.3156} \\ 
19	&	\np{20.608}	&	\np{74.408}	&	\np{0.5827} \\ 
20	&	\np{20.876}	&	\np{75.083}	&	\np{0.4077} \\ 
21	&	\np{21.095}	&	\np{75.248}	&	\np{0.0225} \\ 
22	&	\np{22.225}	&	\np{77.243}	&	\np{1.5069} \\ 
23	&	\np{22.407}	&	\np{81.058}	&	\np{4.1791} \\ 
24	&	\np{22.469}	&	\np{78.821}	&	\np{0.1427} \\ 
25	&	\np{23.077}	&	\np{80.714}	&	\np{0.0896} \\ 
26	&	\np{26.421}	&	\np{91.433}	&	\np{0.1932} \\ 
27	&	\np{26.863}	&	\np{91.777}	&	\np{0.2868} \\ 
28	&	\np{27.360}	&	\np{93.291}	&	\np{0.2549} \\
\midrule
\multicolumn{3}{r}{$\sum_i (y_{1,i} - A - B x_i)^2$:} & \np{17.4411} \\ 
\bottomrule
\end{tabular}
\vspace{1mm}
\caption{Tabela com os valores de $(\Xi_i)^2 = (y_{1,i} - A - B x_i)^2$ e da soma $\sum_i (y_{1,i} - A - B x_i)^2$. \label{Tab:DadosDeterminacaoErrosCoeficientesAeB}}
\end{margintable}

Vamos determinar os erros associados aos coeficientes da regressão linear apresentada no exemplo da Seção~\ref{Sec:RegressaoLinear}. Note que os termos $\sum x_i$ e $\sum x_i^2$ que aparecem nas Equações~\eqref{Eq:ErrosCoeficientesRegLinear1} e~\eqref{Eq:ErrosCoeficientesRegLinear4} estão disponíveis na Tabela~\eqref{Tab:TabelaExemploCalculoRegressaoLinear}, porém na prática utilizamos uma calculadora ou um programa de computador para realizar o cálculo dos coeficientes da regressão linear, e por isso não temos de antemão tais valores\footnote{Algumas calculadoras disponibilizam os valores de $\sum x_i$ e $\sum x_i^2$ juntamente com os valores de $A$, $B$, e $r$.}. Nesse caso, será necessário determinar os valores de $\sum x_i$ e de $\sum x_i^2$, além do valor de $\sum (y_{1,i} - A - Bx_i)^2$ ---~novamente, a elaboração de uma tabela ajuda na determinação de tal valor, como mostrado na Tabela~\ref{Tab:DadosDeterminacaoErrosCoeficientesAeB}. Uma vez determinados tais valores, determinar os erros $\delta A$ e $\delta B$ é relativamente simples:
\begin{align}
	\xi_y &= \sqrt{\frac{1}{N-2}\sum(y_i-A-B\,x_i)^2} \\
	&= \sqrt{\frac{\np{17.4411}}{28-2}} \\
	&= \np{0.05814}, \\
	\Delta &= N\sum x_i^2 - \left(\sum x_i\right)^2 \\
	&= 28\cdot\np{7295,757} - (\np{392,984})^2 \\
	&= \np{49844.766}, \\
	\delta A &= \xi_y\sqrt{\frac{\sum x_i^2}{\Delta}} \\
	&= \np{0.05814} \cdot \sqrt{\frac{\np{7295,757}}{\np{49844.766}}} \\
	&= \np{0.022244}, \\
	\delta B &= \xi_y\sqrt{\frac{N}{\Delta}} \\
	&= \np{0.05814} \cdot \sqrt{\frac{28}{\np{49844.766}}} \\
	&= \np{0.001378}.
\end{align}

Assim, podemos denotar os coeficientes linear e angular com seus devidos erros:
\begin{align}
	A &= (\np{12.140462642} \pm \np{0.022244})~{\rm{Un}}_y\\
	B &= (\np{2.984480401} \pm \np{0.001378})~{\rm{Un}}_{\nicefrac{x}{y}}.
\end{align}
%
Note que, como discutido anteriormente, a unidade do coeficiente angular é a mesma que a da variável dependente, enquanto a unidade do coeficiente angular é dada pela razão entre as unidades da variável dependente e da variável independente. Acima denotamos conceitualmente tais unidades como ${\rm{Un}}_{y}$ e ${\rm{Un}}_{\nicefrac{x}{y}}$, respectivamente. Resta ainda notar que não devemos carregar dígitos cuja ordem de grandeza é menor que a do erro, assim obtemos finalmente
\begin{align}
	A &= (\np{12.14} \pm \np{0.02})~{\rm{Un}}_y\\
	B &= (\np{2.984} \pm \np{0.0014})~{\rm{Un}}_{\nicefrac{x}{y}}.
\end{align}

